\PassOptionsToPackage{table}{xcolor}
\documentclass[justified,openany,nofonts]{tufte-book}
\usepackage{etex}\reserveinserts{56}
\usepackage{mooculus}

\usepackage{showkeys} %% Useful for debugging

\setcounter{secnumdepth}{2}




% Prints the month name (e.g., January) and the year (e.g., 2008)
\newcommand{\monthyear}{%
  \ifcase\month\or January\or February\or March\or April\or May\or June\or
  July\or August\or September\or October\or November\or
  December\fi\space\number\year
}

% Generates the index
\usepackage{makeidx}
\makeindex

\newcommand{\xrefn}[1]{\ref{#1}}



\newenvironment{lemma}{\subsection*{Lemma}}{}
\newenvironment{remark}[1]{\subsection*{Remark: #1}}{}


%\def\exam{\null}
\def\pagerdef{\null}

\def\dfont{\bf}
\def\em{\it}           % for emphasis

\newcommand{\ds}{\displaystyle}

\let\ssk\smallskip \let\msk\medskip \let\bsk\bigskip

\usepackage{multicol}
\def\twocol{\begin{multicols}{2}}
\def\endtwocol{\end{multicols}}

\title{Calculus 1\\ For Math 2250\\
University of Georgia}
\author{Danny Krashen}
\publisher{This document was typeset on \today.}
%\newcommand{\blankpage}{\newpage\hbox{}\thispagestyle{empty}\newpage}

%% % Prints an epigraph and speaker in sans serif, all-caps type.
%% \newcommand{\openepigraph}[2]{%
%%   %\sffamily\fontsize{14}{16}\selectfont
%%   \begin{fullwidth}
%%   \sffamily\large
%%   \begin{doublespace}
%%   \noindent\allcaps{#1}\\% epigraph
%%   \noindent\allcaps{#2}% author
%%   \end{doublespace}
%%   \end{fullwidth}
%% }




\begin{document}
\maketitle

% v.4 copyright page


\begin{fullwidth}
~\vfill
\thispagestyle{empty}
\setlength{\parindent}{0pt}
\setlength{\parskip}{\baselineskip}
Copyright \copyright\ \the\year\ Daniel Krashen 

This work is licensed under the Creative Commons
Attribution-NonCommercial-ShareAlike License. To view a copy of this
license, visit
\url{http://creativecommons.org/licenses/by-nc-sa/3.0/}~or send a
letter to Creative Commons, 543 Howard Street, 5th Floor, San
Francisco, California, 94105, USA. If you distribute this work or a
derivative, include the history of the document. 

The source code is available
at: \url{https://github.com/ASCTech/mooculus/tree/master/public/textbook}

\noindent
This text is based on Jim Fowler and Bart Snapp's Mooculus calculus text,
which is based on David Guichard's open-source calculus text which
in turn is a modification and expansion of notes written by Neal
Koblitz at the University of Washington. David Guichard's text is
available at \url{http://www.whitman.edu/mathematics/calculus/}~under a Creative Commons license.

\noindent The book includes some exercises and examples from {\it
  Elementary Calculus: An Approach Using Infinitesimals}, by H.~Jerome
Keisler, available at
\url{http://www.math.wisc.edu/~keisler/calc.html}~under a Creative
Commons license. 


\noindent This book is typeset in the Kerkis font, 
Kerkis \copyright~Department of Mathematics, University of the Aegean.

\end{fullwidth}


\tableofcontents


%% \renewcommand{\listtheoremname}{List of Main Theorems}
%% \setcounter{tocdepth}{1}
%% \listoftheorems[numwidth=4em,ignoreall,show={mainTheorem}]

\chapter*{List of Main Theorems}% uses ntheorem
\theoremlisttype{opt} 
\listtheorems{mainTheorem}

\chapter*{How to Read Mathematics}

Reading mathematics is \textbf{not} the same as reading a
novel. To read mathematics you need: 
\begin{enumerate}
\item A pen. 
\item Plenty of blank paper. 
\item A willingness to write things down.
\end{enumerate}
As you read mathematics, you must work alongside the text
itself. You must \textbf{write} down each expression, \textbf{sketch}
each graph, and \textbf{think} about what you are doing. You should
work examples and fill in the details. This is not an easy task; it is
in fact \textbf{hard} work. However, mathematics is not a passive
endeavor. You, the reader, must become a doer of mathematics.



\setcounter{chapter}{-1}

%%
% Start the main matter (normal chapters)
%\mainmatter
%\tikzexternaldisable
\input functions
\input limits
\input infinityAndContinuity
\input basicsOfDerivatives
\input derivativesOfTrigFunctions
\input productRuleAndQuotientRule
\input chainRule
\input applicationsOfDifferentiation
\input curveSketching
\input limitsAtInfinityAndLHopital
\input optimization
\input approximation
\input antiderivatives
\input integrals
\input fundamentalTheoremOfCalculus
\input techniquesOfIntegration
\input applicationsOfIntegration

\appendix
\input inverseTrigFunctions



%\bibliography{sample-handout}
%\bibliographystyle{plainnat}

\finalizeanswers
\chapter*{Answers to Exercises}
\small
\addcontentsline{toc}{chapter}{Answers to Exercises}
\IfFileExists{answers.tex}{\subsection *{Answers for 0.1}
\hypertarget {a:0.1.1}{\hyperlink {e:0.1.1}{\bfseries 1.}} \mdseries $2$\qquad 
\hypertarget {a:0.1.2}{\hyperlink {e:0.1.2}{\bfseries 2.}} \mdseries $-3$\qquad 
\hypertarget {a:0.1.3}{\hyperlink {e:0.1.3}{\bfseries 3.}} \mdseries Yes. Every input has exactly one output.\qquad 
\hypertarget {a:0.1.4}{\hyperlink {e:0.1.4}{\bfseries 4.}} \mdseries Yes. Every input has exactly one output.\qquad 
\hypertarget {a:0.1.5}{\hyperlink {e:0.1.5}{\bfseries 5.}} \mdseries $x=2$\qquad 
\hypertarget {a:0.1.6}{\hyperlink {e:0.1.6}{\bfseries 6.}} \mdseries No. These points define a function as every input has a unique output.\qquad 
\hypertarget {a:0.1.7}{\hyperlink {e:0.1.7}{\bfseries 7.}} \mdseries If $x$ were one of $-1$, $-3$, $5$, or $8$.\qquad 
\hypertarget {a:0.1.8}{\hyperlink {e:0.1.8}{\bfseries 8.}} \mdseries $4$\qquad 
\hypertarget {a:0.1.9}{\hyperlink {e:0.1.9}{\bfseries 9.}} \mdseries $21$\qquad 
\hypertarget {a:0.1.10}{\hyperlink {e:0.1.10}{\bfseries 10.}} \mdseries $3$\qquad 
\hypertarget {a:0.1.11}{\hyperlink {e:0.1.11}{\bfseries 11.}} \mdseries $\sqrt {w^2+w+1}$\qquad 
\hypertarget {a:0.1.12}{\hyperlink {e:0.1.12}{\bfseries 12.}} \mdseries $\sqrt {(x+h)^2+(x+h)+1}$\qquad 
\hypertarget {a:0.1.13}{\hyperlink {e:0.1.13}{\bfseries 13.}} \mdseries $\sqrt {(x+h)^2+(x+h)+1} - \sqrt {x^2+x+1}$\qquad 
\hypertarget {a:0.1.14}{\hyperlink {e:0.1.14}{\bfseries 14.}} \mdseries $5$\qquad 
\hypertarget {a:0.1.15}{\hyperlink {e:0.1.15}{\bfseries 15.}} \mdseries $4+x+h$\qquad 
\hypertarget {a:0.1.16}{\hyperlink {e:0.1.16}{\bfseries 16.}} \mdseries $x=8$, $y=24$\qquad 
\hypertarget {a:0.1.17}{\hyperlink {e:0.1.17}{\bfseries 17.}} \mdseries $x=7/8$, $y=6$\qquad 
\hypertarget {a:0.1.18}{\hyperlink {e:0.1.18}{\bfseries 18.}} \mdseries $x=6$, $y=-7$\qquad 
\subsection *{Answers for 0.2}
\hypertarget {a:0.2.1}{\hyperlink {e:0.2.1}{\bfseries 1.}} \mdseries $\l ^{-1}(t) = \frac {3t}{8} - 3$, this function gives the number of months required to grow hair to a given length.\qquad 
\hypertarget {a:0.2.2}{\hyperlink {e:0.2.2}{\bfseries 2.}} \mdseries $m^{-1}(t) = \frac {t}{900} - \frac {1}{3}$, this function gives the number of months required to acquire a given amount of money.\qquad 
\hypertarget {a:0.2.3}{\hyperlink {e:0.2.3}{\bfseries 3.}} \mdseries $h^{-1}(t) = 1 \mp \sqrt {1.4-0.2t}$. Either function gives the time in terms of a height the cap reaches.\qquad 
\hypertarget {a:0.2.4}{\hyperlink {e:0.2.4}{\bfseries 4.}} \mdseries $n^{-1}(t)= \frac {10}{17} \pm \frac {\sqrt {68t-47200}}{34}$, where either function describes the temperature it takes to reach a certain number of bacteria.\qquad 
\hypertarget {a:0.2.5}{\hyperlink {e:0.2.5}{\bfseries 5.}} \mdseries $h^{-1}(20)=7$. This means that a height of 20 meters is achieved at 7 seconds in the restricted interval. In fact, it turns out that $h^{-1}(t) = 7 \cdot (\pi - \arcsin ((t-20)/18))/\pi $ when $h$ is restricted to the given interval.\qquad 
\hypertarget {a:0.2.6}{\hyperlink {e:0.2.6}{\bfseries 6.}} \mdseries $v^{-1}(4000) = 4.1$. This means that it takes approximately 4.1 years for the car's value to reach 4000 dollars.\qquad 
\hypertarget {a:0.2.7}{\hyperlink {e:0.2.7}{\bfseries 7.}} \mdseries $d^{-1}(85) = 3.2 \cdot 10^{8} \cdot I_0$ or approximately 320 million times the threshold sound.\qquad 
\hypertarget {a:0.2.8}{\hyperlink {e:0.2.8}{\bfseries 8.}} \mdseries $f^{-1}(x)$ is the inverse function (if it exists) of $f(x)$; $f(x)^{-1}$ is $1/f(x)$, the multiplicative inverse.\qquad 
\hypertarget {a:0.2.9}{\hyperlink {e:0.2.9}{\bfseries 9.}} \mdseries Group A: $\sin ^2x$, $\sin (x)^2$, $(\sin x)^2$, $(\sin x)(\sin x)$; Group B: $\sin (x^2)$, $\sin x^2$\qquad 
\hypertarget {a:0.2.10}{\hyperlink {e:0.2.10}{\bfseries 10.}} \mdseries Group A: $\arcsin (x)$, $\sin ^{-1}(x)$; Group B: $\frac {1}{\sin (x)}$, $(\sin x)^{-1}$\qquad 
\hypertarget {a:0.2.11}{\hyperlink {e:0.2.11}{\bfseries 11.}} \mdseries No. Consider $x = -1$. $\sqrt {(-1)^2} = \sqrt {1} = 1$. However, $\sqrt [3]{(-1)^3} = \sqrt [3]{-1} = -1$.\qquad 
\subsection *{Answers for 1.1}
\hypertarget {a:1.1.1}{\hyperlink {e:1.1.1}{\bfseries 1.}} \mdseries (a) $8$, (b) $6$, (c) DNE, (d) $-2$, (e) $-1$, (f) $8$, (g) $7$, (h) $6$, (i) $3$, (j) $-3/2$, (k) $6$, (l) $2$\qquad 
\hypertarget {a:1.1.2}{\hyperlink {e:1.1.2}{\bfseries 2.}} \mdseries $1$\qquad 
\hypertarget {a:1.1.3}{\hyperlink {e:1.1.3}{\bfseries 3.}} \mdseries $2$\qquad 
\hypertarget {a:1.1.4}{\hyperlink {e:1.1.4}{\bfseries 4.}} \mdseries $3$\qquad 
\hypertarget {a:1.1.5}{\hyperlink {e:1.1.5}{\bfseries 5.}} \mdseries $3/5$\qquad 
\hypertarget {a:1.1.6}{\hyperlink {e:1.1.6}{\bfseries 6.}} \mdseries $0.6931\approx \ln (2)$\qquad 
\hypertarget {a:1.1.7}{\hyperlink {e:1.1.7}{\bfseries 7.}} \mdseries $2.718 \approx e$\qquad 
\hypertarget {a:1.1.8}{\hyperlink {e:1.1.8}{\bfseries 8.}} \mdseries Consider what happens when $x$ is near zero and positive, as compared to when $x$ is near zero and negative.\qquad 
\hypertarget {a:1.1.9}{\hyperlink {e:1.1.9}{\bfseries 9.}} \mdseries The limit does not exist, so it is not surprising that the resulting values are so different.\qquad 
\hypertarget {a:1.1.10}{\hyperlink {e:1.1.10}{\bfseries 10.}} \mdseries When $v$ approaches $c$ from below, then $t_v$ approaches zero---meaning that one second to the stationary observations seems like very little time at all for our traveler.\qquad 
\subsection *{Answers for 1.2}
\hypertarget {a:1.2.1}{\hyperlink {e:1.2.1}{\bfseries 1.}} \mdseries For these problems, there are many possible values of $\delta $, so we provide an inequality that $\delta $ must satisfy when $\epsilon = 0.1$.\hspace {1em} (a)~$\delta < 1/30$, (b)~$\delta < \frac {\sqrt {110}}{10} - 1 \approx 0.0488$, (c)~$\delta < \arcsin \left ( 1/10 \right ) \approx 0.1002$, (d)~$\delta < \arctan \left ( 1/10 \right ) \approx 0.0997$ (e)~$\delta < 13/100$, (f)~$\delta < 59/400$\qquad 
\hypertarget {a:1.2.2}{\hyperlink {e:1.2.2}{\bfseries 2.}} \mdseries Let $\epsilon >0$. Set $\delta = \epsilon $. If $0<|x-0| <\delta $, then $|x\cdot 1|<\epsilon $, since $\sin \left (\frac {1}{x}\right )\le 1$, $|x\sin \left ( {1\over x}\right )-0|< \epsilon $.\qquad 
\hypertarget {a:1.2.3}{\hyperlink {e:1.2.3}{\bfseries 3.}} \mdseries Let $\epsilon > 0$. Set $\delta = \epsilon /2$. If $0 < |x - 4| < \delta $, then $|2x - 8| < 2 \delta = \epsilon $, and then because $|2x - 8| = |(2x - 5) - 3|$, we conclude $|(2x - 5) - 3| < \epsilon $.\qquad 
\hypertarget {a:1.2.4}{\hyperlink {e:1.2.4}{\bfseries 4.}} \mdseries Let $\epsilon > 0$. Set $\delta = \epsilon /4$. If $0 < |x - (-3)| < \delta $, then $|-4x - 12| < 4 \delta = \epsilon $, and then because $|-4x - 12| = |(-4x - 11) - 1|$, we conclude $|(-4x - 11) - 1| < \epsilon $.\qquad 
\hypertarget {a:1.2.5}{\hyperlink {e:1.2.5}{\bfseries 5.}} \mdseries Let $\epsilon > 0$. No matter what I choose for $\delta $, if $x$ is within $\delta $ of $-2$, then $\pi $ is within $\epsilon $ of $\pi $.\qquad 
\hypertarget {a:1.2.6}{\hyperlink {e:1.2.6}{\bfseries 6.}} \mdseries As long as $x \neq -2$, we have $\frac {x^2-4}{x+2} = x-2$, and the limit is not sensitive to the value of the function at the point $-2$; the limit only depends on nearby values, so we really want to compute $\lim _{x \to -2} (x-2)$. Let $\epsilon > 0$. Set $\delta = \epsilon $. Then if $0 < |x - (-2)| < \delta $, we have $|(x - 2) - (-4)| < \epsilon $.\qquad 
\hypertarget {a:1.2.7}{\hyperlink {e:1.2.7}{\bfseries 7.}} \mdseries Let $\epsilon > 0$. Pick $\delta $ so that $\delta < 1$ and $\delta < \frac {\epsilon }{61}$. Suppose $0 < |x-4| < \delta $. Then $4 - \delta < x < 4 + \delta $. Cube to get $\left ( 4 - \delta \right )^3 < x^3 < \left ( 4 + \delta \right )^3$. Expanding the right-side inequality, we get $x^3 < \delta ^3 + 12\cdot \delta ^2 + 48\cdot \delta + 64 < \delta + 12 \delta + 48 \delta + 64 = 64 + \epsilon $. The other inequality is similar.\qquad 
\hypertarget {a:1.2.8}{\hyperlink {e:1.2.8}{\bfseries 8.}} \mdseries Let $\epsilon > 0$. Pick $\delta $ small enough so that $\delta < \epsilon / 6$ and $\delta < 1$. Assume $|x - 1| < \delta $, so $6 \cdot |x - 1| < \epsilon $. Since $x$ is within $\delta < 1$ of $1$, we know $0 < x < 2$. So $|x+4| < 6$. Putting it together, $|x+4| \cdot |x-1| < \epsilon $, so $|x^2 + 3x - 4| < \epsilon $, and therefore $|(x^2 + 3x - 1) - 3| < \epsilon $.\qquad 
\hypertarget {a:1.2.9}{\hyperlink {e:1.2.9}{\bfseries 9.}} \mdseries Let $\epsilon > 0$. Set $\delta = 3\epsilon $. Assume $0 < |x-9| < \delta $. Divide both sides by $3$ to get $\frac {|x-9|}{3} < \epsilon $. Note that $\sqrt {x}+3 > 3$, so $\frac {|x-9|}{\sqrt {x} + 3} < \epsilon $. This can be rearranged to conclude $\left |\frac {x-9}{\sqrt {x} - 3} - 6\right | < \epsilon $.\qquad 
\hypertarget {a:1.2.10}{\hyperlink {e:1.2.10}{\bfseries 10.}} \mdseries Let $\epsilon > 0$. Set $\delta $ to be the minimum of $2\epsilon $ and $1$. Assume $x$ is within $\delta $ of $2$, so $|x - 2| < 2 \epsilon $ and $1 < x < 3$. So $\left | \frac {x-2}{2} \right | < \epsilon $. Since $1 < x < 3$, we also have $2x > 2$, so $\left | \frac {x-2}{2x} \right | < \epsilon $. Simplifying, $\left | \frac {1}{2} - \frac {1}{x} \right | < \epsilon $, which is what we wanted.\qquad 
\subsection *{Answers for 1.3}
\hypertarget {a:1.3.1}{\hyperlink {e:1.3.1}{\bfseries 1.}} \mdseries $7$\qquad 
\hypertarget {a:1.3.2}{\hyperlink {e:1.3.2}{\bfseries 2.}} \mdseries $5$\qquad 
\hypertarget {a:1.3.3}{\hyperlink {e:1.3.3}{\bfseries 3.}} \mdseries $0$\qquad 
\hypertarget {a:1.3.4}{\hyperlink {e:1.3.4}{\bfseries 4.}} \mdseries DNE\qquad 
\hypertarget {a:1.3.5}{\hyperlink {e:1.3.5}{\bfseries 5.}} \mdseries $1/6$\qquad 
\hypertarget {a:1.3.6}{\hyperlink {e:1.3.6}{\bfseries 6.}} \mdseries $0$\qquad 
\hypertarget {a:1.3.7}{\hyperlink {e:1.3.7}{\bfseries 7.}} \mdseries $3$\qquad 
\hypertarget {a:1.3.8}{\hyperlink {e:1.3.8}{\bfseries 8.}} \mdseries $172$\qquad 
\hypertarget {a:1.3.9}{\hyperlink {e:1.3.9}{\bfseries 9.}} \mdseries $0$\qquad 
\hypertarget {a:1.3.10}{\hyperlink {e:1.3.10}{\bfseries 10.}} \mdseries $2$\qquad 
\hypertarget {a:1.3.11}{\hyperlink {e:1.3.11}{\bfseries 11.}} \mdseries DNE\qquad 
\hypertarget {a:1.3.12}{\hyperlink {e:1.3.12}{\bfseries 12.}} \mdseries $\sqrt 2$\qquad 
\hypertarget {a:1.3.13}{\hyperlink {e:1.3.13}{\bfseries 13.}} \mdseries $3a^2$\qquad 
\hypertarget {a:1.3.14}{\hyperlink {e:1.3.14}{\bfseries 14.}} \mdseries $512$\qquad 
\hypertarget {a:1.3.15}{\hyperlink {e:1.3.15}{\bfseries 15.}} \mdseries $-4$\qquad 
\subsection *{Answers for 2.1}
\hypertarget {a:2.1.1}{\hyperlink {e:2.1.1}{\bfseries 1.}} \mdseries $f(x)$ is continuous at $x=4$ but it is not continuous on $\R $.\qquad 
\hypertarget {a:2.1.2}{\hyperlink {e:2.1.2}{\bfseries 2.}} \mdseries $f(x)$ is continuous at $x=3$ but it is not continuous on $\R $.\qquad 
\hypertarget {a:2.1.3}{\hyperlink {e:2.1.3}{\bfseries 3.}} \mdseries $f(x)$ is not continuous at $x=1$ and it is not continuous on $\R $.\qquad 
\hypertarget {a:2.1.4}{\hyperlink {e:2.1.4}{\bfseries 4.}} \mdseries $f(x)$ is not continuous at $x=5$ and it is not continuous on $\R $.\qquad 
\hypertarget {a:2.1.5}{\hyperlink {e:2.1.5}{\bfseries 5.}} \mdseries $f(x)$ is continuous at $x=-5$ and it is also continuous on $\R $.\qquad 
\hypertarget {a:2.1.6}{\hyperlink {e:2.1.6}{\bfseries 6.}} \mdseries $\R $\qquad 
\hypertarget {a:2.1.7}{\hyperlink {e:2.1.7}{\bfseries 7.}} \mdseries $(-\infty ,-4)\cup (-4,\infty )$\qquad 
\hypertarget {a:2.1.8}{\hyperlink {e:2.1.8}{\bfseries 8.}} \mdseries $(-\infty ,-3)\cup (-3,3)\cup (3,\infty )$\qquad 
\hypertarget {a:2.1.9}{\hyperlink {e:2.1.9}{\bfseries 9.}} \mdseries $x=-0.48$, $x=1.31$, or $x=3.17$\qquad 
\hypertarget {a:2.1.10}{\hyperlink {e:2.1.10}{\bfseries 10.}} \mdseries $x=0.20$, or $x=1.35$\qquad 
\subsection *{Answers for 2.2}
\hypertarget {a:2.2.1}{\hyperlink {e:2.2.1}{\bfseries 1.}} \mdseries $-\infty $\qquad 
\hypertarget {a:2.2.2}{\hyperlink {e:2.2.2}{\bfseries 2.}} \mdseries $3/14$\qquad 
\hypertarget {a:2.2.3}{\hyperlink {e:2.2.3}{\bfseries 3.}} \mdseries $1/2$\qquad 
\hypertarget {a:2.2.4}{\hyperlink {e:2.2.4}{\bfseries 4.}} \mdseries $-\infty $\qquad 
\hypertarget {a:2.2.5}{\hyperlink {e:2.2.5}{\bfseries 5.}} \mdseries $\infty $\qquad 
\hypertarget {a:2.2.6}{\hyperlink {e:2.2.6}{\bfseries 6.}} \mdseries $\infty $\qquad 
\hypertarget {a:2.2.7}{\hyperlink {e:2.2.7}{\bfseries 7.}} \mdseries 0\qquad 
\hypertarget {a:2.2.8}{\hyperlink {e:2.2.8}{\bfseries 8.}} \mdseries $-\infty $\qquad 
\hypertarget {a:2.2.9}{\hyperlink {e:2.2.9}{\bfseries 9.}} \mdseries $x = 1$ and $x = -3$\qquad 
\hypertarget {a:2.2.10}{\hyperlink {e:2.2.10}{\bfseries 10.}} \mdseries $x = -4$\qquad 
\subsection *{Answers for 3.1}
\hypertarget {a:3.1.1}{\hyperlink {e:3.1.1}{\bfseries 1.}} \mdseries $f(2) = 10$ and $f'(2) = 7$\qquad 
\hypertarget {a:3.1.2}{\hyperlink {e:3.1.2}{\bfseries 2.}} \mdseries $p'(x) = s(x)$ and $r'(x) = q(x)$\qquad 
\hypertarget {a:3.1.3}{\hyperlink {e:3.1.3}{\bfseries 3.}} \mdseries $f'(3)\approx 4$\qquad 
\hypertarget {a:3.1.4}{\hyperlink {e:3.1.4}{\bfseries 4.}} \mdseries $f'(-2) = 4$\qquad 
\hypertarget {a:3.1.5}{\hyperlink {e:3.1.5}{\bfseries 5.}} \mdseries $f(1.2) \approx 2.2$\qquad 
\hypertarget {a:3.1.6}{\hyperlink {e:3.1.6}{\bfseries 6.}} \mdseries (a) $(0,4.5)\cup (4.5,6)$, (b) $(0,3)\cup (3,4.5)\cup (4.5,6)$, (c) See Figure\begin {marginfigure}[0in] \begin {tikzpicture} \begin {axis}[ domain=0:6, ymax=1.5, ymin=-1.5, samples=100, axis lines =middle, xlabel=$x$, ylabel=$y$, every axis y label/.style={at=(current axis.above origin),anchor=south}, every axis x label/.style={at=(current axis.right of origin),anchor=west}, grid=both, grid style={dashed, gridColor}, xtick={0,...,6}, ytick={-1,...,1}, ] \addplot [very thick, penColor, smooth,domain=(0:3)] {-(pi/3)*cos(deg(pi*x/3)))}; \addplot [very thick, penColor, smooth,domain=(3:6)] {-(pi/3)*cos(deg(pi*(x-3)/3)))}; \addplot [color=penColor,fill=background,only marks,mark=*] coordinates{(3,pi/3)}; \addplot [color=penColor,fill=background,only marks,mark=*] coordinates{(3,-pi/3)}; \addplot [color=penColor,fill=background,only marks,mark=*] coordinates{(4.5,0)}; \end {axis} \end {tikzpicture} Answer 3.1.6: (c) a sketch of $f'(x)$. \end {marginfigure}\qquad 
\hypertarget {a:3.1.7}{\hyperlink {e:3.1.7}{\bfseries 7.}} \mdseries $f'(-3) = -6$ with tangent line $y = -6x -13$\qquad 
\hypertarget {a:3.1.8}{\hyperlink {e:3.1.8}{\bfseries 8.}} \mdseries $f'(1) = -1/9$ with tangent line $y = \frac {-1}{9}x + \frac {4}{9}$\qquad 
\hypertarget {a:3.1.9}{\hyperlink {e:3.1.9}{\bfseries 9.}} \mdseries $f'(5) = \frac {1}{2\sqrt {2}}$ with tangent line $y = \frac {1}{2\sqrt {2}}x -\frac {1}{2\sqrt {2}}$\qquad 
\hypertarget {a:3.1.10}{\hyperlink {e:3.1.10}{\bfseries 10.}} \mdseries $f'(4) = \frac {-1}{16}$ with tangent line $y = \frac {-1}{16}x +\frac {3}{4}$\qquad 
\subsection *{Answers for 3.2}
\hypertarget {a:3.2.1}{\hyperlink {e:3.2.1}{\bfseries 1.}} \mdseries $0$\qquad 
\hypertarget {a:3.2.2}{\hyperlink {e:3.2.2}{\bfseries 2.}} \mdseries $0$\qquad 
\hypertarget {a:3.2.3}{\hyperlink {e:3.2.3}{\bfseries 3.}} \mdseries $0$\qquad 
\hypertarget {a:3.2.4}{\hyperlink {e:3.2.4}{\bfseries 4.}} \mdseries $0$\qquad 
\hypertarget {a:3.2.5}{\hyperlink {e:3.2.5}{\bfseries 5.}} \mdseries $100x^{99}$\qquad 
\hypertarget {a:3.2.6}{\hyperlink {e:3.2.6}{\bfseries 6.}} \mdseries $-100x^{-101}$\qquad 
\hypertarget {a:3.2.7}{\hyperlink {e:3.2.7}{\bfseries 7.}} \mdseries $-5x^{-6}$\qquad 
\hypertarget {a:3.2.8}{\hyperlink {e:3.2.8}{\bfseries 8.}} \mdseries $\pi x^{\pi -1}$\qquad 
\hypertarget {a:3.2.9}{\hyperlink {e:3.2.9}{\bfseries 9.}} \mdseries $(3/4)x^{-1/4}$\qquad 
\hypertarget {a:3.2.10}{\hyperlink {e:3.2.10}{\bfseries 10.}} \mdseries $-(9/7)x^{-16/7}$\qquad 
\hypertarget {a:3.2.11}{\hyperlink {e:3.2.11}{\bfseries 11.}} \mdseries $15x^2+24x$\qquad 
\hypertarget {a:3.2.12}{\hyperlink {e:3.2.12}{\bfseries 12.}} \mdseries $-20x^4+6x+10/x^3$\qquad 
\hypertarget {a:3.2.13}{\hyperlink {e:3.2.13}{\bfseries 13.}} \mdseries $-30x+25$\qquad 
\hypertarget {a:3.2.14}{\hyperlink {e:3.2.14}{\bfseries 14.}} \mdseries $\frac {3}{2}x^{-1/2}-x^{-2}-ex^{e-1}$\qquad 
\hypertarget {a:3.2.15}{\hyperlink {e:3.2.15}{\bfseries 15.}} \mdseries $-5x^{-6} - x^{-3/2}/2$\qquad 
\hypertarget {a:3.2.16}{\hyperlink {e:3.2.16}{\bfseries 16.}} \mdseries $e^x$\qquad 
\hypertarget {a:3.2.17}{\hyperlink {e:3.2.17}{\bfseries 17.}} \mdseries $ex^{e-1}$\qquad 
\hypertarget {a:3.2.18}{\hyperlink {e:3.2.18}{\bfseries 18.}} \mdseries $3e^x$\qquad 
\hypertarget {a:3.2.19}{\hyperlink {e:3.2.19}{\bfseries 19.}} \mdseries $12x^3-14x +12 e^x$\qquad 
\hypertarget {a:3.2.20}{\hyperlink {e:3.2.20}{\bfseries 20.}} \mdseries $3x^2+6x-1$\qquad 
\hypertarget {a:3.2.21}{\hyperlink {e:3.2.21}{\bfseries 21.}} \mdseries $2x-1 $\qquad 
\hypertarget {a:3.2.22}{\hyperlink {e:3.2.22}{\bfseries 22.}} \mdseries $x^{-1/2}/2 $\qquad 
\hypertarget {a:3.2.23}{\hyperlink {e:3.2.23}{\bfseries 23.}} \mdseries $4x^3-4x$\qquad 
\hypertarget {a:3.2.24}{\hyperlink {e:3.2.24}{\bfseries 24.}} \mdseries $-49t/5+5$, $-49/5$\qquad 
\hypertarget {a:3.2.25}{\hyperlink {e:3.2.25}{\bfseries 25.}} \mdseries See Figure\begin {marginfigure}[0in] \begin {tikzpicture} \begin {axis}[ domain=-3:3, ymax=10, ymin=-10, samples=100, axis lines =middle, xlabel=$x$, ylabel=$y$, every axis y label/.style={at=(current axis.above origin),anchor=south}, every axis x label/.style={at=(current axis.right of origin),anchor=west}, ] \addplot [very thick, penColor, smooth,domain=(-3:3)] {x^3}; \addplot [very thick, penColor2, smooth,domain=(-3:3)] {3*x^3}; \addplot [very thick, penColor!50!background, smooth,domain=(-3:3)] {3*x^2}; \addplot [very thick, penColor2!50!background, smooth,domain=(-3:3)] {9*x^2}; \node at (axis cs:1.5,3) [anchor=west] {\color {penColor}$f(x)$}; \node at (axis cs:-1,-3) [anchor=west] {\color {penColor2}$cf(x)$}; \node at (axis cs:-1.6,3) [anchor=west] {\color {penColor!50!background}$f'(x)$}; \node at (axis cs:-.9,7) [anchor=west] {\color {penColor2!50!background}$(cf(x))'$}; \end {axis} \end {tikzpicture} Answer 3.2.25. \end {marginfigure}\qquad 
\hypertarget {a:3.2.26}{\hyperlink {e:3.2.26}{\bfseries 26.}} \mdseries $x^3/16-3x/4+4$\qquad 
\hypertarget {a:3.2.27}{\hyperlink {e:3.2.27}{\bfseries 27.}} \mdseries $y=13x/4+5$\qquad 
\hypertarget {a:3.2.28}{\hyperlink {e:3.2.28}{\bfseries 28.}} \mdseries $y=24x-48-\pi ^3$\qquad 
\hypertarget {a:3.2.29}{\hyperlink {e:3.2.29}{\bfseries 29.}} \mdseries $\ddx cf(x) = \lim _{h\to 0}\frac {cf(x+h)-cf(x)}{h} = c\lim _{h\to 0}\frac {f(x+h)-f(x)}{h} = cf'(x)$.\qquad 
\subsection *{Answers for 4.1}
\hypertarget {a:4.1.1}{\hyperlink {e:4.1.1}{\bfseries 1.}} \mdseries $\sin (\sqrt {x})\cos (\sqrt {x})/\sqrt {x}$\qquad 
\hypertarget {a:4.1.2}{\hyperlink {e:4.1.2}{\bfseries 2.}} \mdseries ${\sin (x)\over 2\sqrt x}+\sqrt {x}\cos (x)$\qquad 
\hypertarget {a:4.1.3}{\hyperlink {e:4.1.3}{\bfseries 3.}} \mdseries $ -{\cos (x)\over \sin ^2(x)}$\qquad 
\hypertarget {a:4.1.4}{\hyperlink {e:4.1.4}{\bfseries 4.}} \mdseries ${(2x +1)\sin (x) - (x^2+x)\cos (x) \over \sin ^2 (x)}$\qquad 
\hypertarget {a:4.1.5}{\hyperlink {e:4.1.5}{\bfseries 5.}} \mdseries ${-\sin (x)\cos (x)\over \sqrt {1-\sin ^2(x)}}$\qquad 
\hypertarget {a:4.1.6}{\hyperlink {e:4.1.6}{\bfseries 6.}} \mdseries $\cos ^2(x)-\sin ^2(x)$\qquad 
\hypertarget {a:4.1.7}{\hyperlink {e:4.1.7}{\bfseries 7.}} \mdseries $-\sin (x)\cos (\cos (x))$\qquad 
\hypertarget {a:4.1.8}{\hyperlink {e:4.1.8}{\bfseries 8.}} \mdseries ${\tan (x)+x\sec ^2(x)\over 2\sqrt {x\tan (x)}}$\qquad 
\hypertarget {a:4.1.9}{\hyperlink {e:4.1.9}{\bfseries 9.}} \mdseries ${\sec ^2(x)(1+\sin (x))-\tan (x) \cos (x)\over (1+\sin (x))^2}$\qquad 
\hypertarget {a:4.1.10}{\hyperlink {e:4.1.10}{\bfseries 10.}} \mdseries $ -\csc ^2(x)$\qquad 
\hypertarget {a:4.1.11}{\hyperlink {e:4.1.11}{\bfseries 11.}} \mdseries $ -\csc (x)\cot (x)$\qquad 
\hypertarget {a:4.1.12}{\hyperlink {e:4.1.12}{\bfseries 12.}} \mdseries $3x^2\sin (23x^2)+46x^4\cos (23x^2)$\qquad 
\hypertarget {a:4.1.13}{\hyperlink {e:4.1.13}{\bfseries 13.}} \mdseries $0$\qquad 
\hypertarget {a:4.1.14}{\hyperlink {e:4.1.14}{\bfseries 14.}} \mdseries $-6\cos (\cos (6x))\sin (6x)$\qquad 
\hypertarget {a:4.1.15}{\hyperlink {e:4.1.15}{\bfseries 15.}} \mdseries $\sin (\theta )/(\cos (\theta )+1)^2$\qquad 
\hypertarget {a:4.1.16}{\hyperlink {e:4.1.16}{\bfseries 16.}} \mdseries $5t^4\cos (6t)-6t^5\sin (6t)$\qquad 
\hypertarget {a:4.1.17}{\hyperlink {e:4.1.17}{\bfseries 17.}} \mdseries $3t^2(\sin (3t)+t\cos (3t))/\cos (2t)+2t^3\sin (3t)\sin (2t)/\cos ^2(2t)$\qquad 
\hypertarget {a:4.1.18}{\hyperlink {e:4.1.18}{\bfseries 18.}} \mdseries $n\pi /2$, any integer $n$\qquad 
\hypertarget {a:4.1.19}{\hyperlink {e:4.1.19}{\bfseries 19.}} \mdseries $\pi /2+n\pi $, any integer $n$\qquad 
\hypertarget {a:4.1.20}{\hyperlink {e:4.1.20}{\bfseries 20.}} \mdseries $\sqrt 3x/2+3/4-\sqrt 3\pi /6$\qquad 
\hypertarget {a:4.1.21}{\hyperlink {e:4.1.21}{\bfseries 21.}} \mdseries $8\sqrt 3x+4-8\sqrt 3\pi /3$\qquad 
\hypertarget {a:4.1.22}{\hyperlink {e:4.1.22}{\bfseries 22.}} \mdseries $3\sqrt 3x/2-\sqrt 3\pi /4$\qquad 
\hypertarget {a:4.1.23}{\hyperlink {e:4.1.23}{\bfseries 23.}} \mdseries $\pi /6+2n\pi $, $5\pi /6+2n\pi $, any integer $n$\qquad 
\subsection *{Answers for 5.1}
\hypertarget {a:5.1.1}{\hyperlink {e:5.1.1}{\bfseries 1.}} \mdseries $3x^2(x^3-5x+10)+x^3(3x^2-5)$\qquad 
\hypertarget {a:5.1.2}{\hyperlink {e:5.1.2}{\bfseries 2.}} \mdseries $(x^2+5x-3)(5x^4-18x^2+6x-7)+(2x+5)(x^5-6x^3+3x^2-7x+1)$\qquad 
\hypertarget {a:5.1.3}{\hyperlink {e:5.1.3}{\bfseries 3.}} \mdseries $2e^{2x}$\qquad 
\hypertarget {a:5.1.4}{\hyperlink {e:5.1.4}{\bfseries 4.}} \mdseries $3e^{3x}$\qquad 
\hypertarget {a:5.1.5}{\hyperlink {e:5.1.5}{\bfseries 5.}} \mdseries $6xe^{4x}+12x^2e^{4x}$\qquad 
\hypertarget {a:5.1.6}{\hyperlink {e:5.1.6}{\bfseries 6.}} \mdseries $\frac {-48e^x}{x^{17}}+\frac {3e^x}{x^{16}}$\qquad 
\hypertarget {a:5.1.7}{\hyperlink {e:5.1.7}{\bfseries 7.}} \mdseries $f'=4(2x-3)$, $y=4x-7$\qquad 
\hypertarget {a:5.1.8}{\hyperlink {e:5.1.8}{\bfseries 8.}} \mdseries $3$\qquad 
\hypertarget {a:5.1.9}{\hyperlink {e:5.1.9}{\bfseries 9.}} \mdseries $10$\qquad 
\hypertarget {a:5.1.10}{\hyperlink {e:5.1.10}{\bfseries 10.}} \mdseries $-13$\qquad 
\hypertarget {a:5.1.11}{\hyperlink {e:5.1.11}{\bfseries 11.}} \mdseries $-5$\qquad 
\hypertarget {a:5.1.12}{\hyperlink {e:5.1.12}{\bfseries 12.}} \mdseries $\ddx f(x)g(x)h(x) = \ddx f(x)(g(x)h(x)) = f(x) \ddx (g(x) h(x)) + f'(x) g(x) h(x) = f(x) (g(x) h'(x)+ g'(x)h(x)) + f'(x) g(x) h(x) = f(x)g(x)h'(x)+ f(x) g'(x)h(x)) + f'(x) g(x) h(x)$\qquad 
\subsection *{Answers for 5.2}
\hypertarget {a:5.2.1}{\hyperlink {e:5.2.1}{\bfseries 1.}} \mdseries ${3x^2\over x^3-5x+10}-{x^3(3x^2-5)\over (x^3-5x+10)^2}$\qquad 
\hypertarget {a:5.2.2}{\hyperlink {e:5.2.2}{\bfseries 2.}} \mdseries ${2x+5\over x^5-6x^3+3x^2-7x+1}- {(x^2+5x-3)(5x^4-18x^2+6x-7)\over (x^5-6x^3+3x^2-7x+1)^2}$\qquad 
\hypertarget {a:5.2.3}{\hyperlink {e:5.2.3}{\bfseries 3.}} \mdseries $\frac {2xe^x-(e^x-4)2}{4x^2}$\qquad 
\hypertarget {a:5.2.4}{\hyperlink {e:5.2.4}{\bfseries 4.}} \mdseries $\frac {(x+2)(-1-(1/2)x^{-1/2}) - (2-x-\sqrt {x})}{(x+2)^2}$\qquad 
\hypertarget {a:5.2.5}{\hyperlink {e:5.2.5}{\bfseries 5.}} \mdseries $y=17x/4-41/4$\qquad 
\hypertarget {a:5.2.6}{\hyperlink {e:5.2.6}{\bfseries 6.}} \mdseries $y=11x/16-15/16$\qquad 
\hypertarget {a:5.2.7}{\hyperlink {e:5.2.7}{\bfseries 7.}} \mdseries $y=19/169-5x/338$\qquad 
\hypertarget {a:5.2.8}{\hyperlink {e:5.2.8}{\bfseries 8.}} \mdseries $-3/16$\qquad 
\hypertarget {a:5.2.9}{\hyperlink {e:5.2.9}{\bfseries 9.}} \mdseries $8/9$\qquad 
\hypertarget {a:5.2.10}{\hyperlink {e:5.2.10}{\bfseries 10.}} \mdseries $24$\qquad 
\hypertarget {a:5.2.11}{\hyperlink {e:5.2.11}{\bfseries 11.}} \mdseries $-3$\qquad 
\hypertarget {a:5.2.12}{\hyperlink {e:5.2.12}{\bfseries 12.}} \mdseries $f(4) = 1/3, \ddx \frac {f(x)}{g(x)} = 13/18$\qquad 
\subsection *{Answers for 6.1}
\hypertarget {a:6.1.1}{\hyperlink {e:6.1.1}{\bfseries 1.}} \mdseries $4x^3-9x^2+x+7$\qquad 
\hypertarget {a:6.1.2}{\hyperlink {e:6.1.2}{\bfseries 2.}} \mdseries $3x^2-4x+2/\sqrt {x}$\qquad 
\hypertarget {a:6.1.3}{\hyperlink {e:6.1.3}{\bfseries 3.}} \mdseries $6(x^2+1)^2x$\qquad 
\hypertarget {a:6.1.4}{\hyperlink {e:6.1.4}{\bfseries 4.}} \mdseries $\sqrt {169-x^2}-x^2/\sqrt {169-x^2}$\qquad 
\hypertarget {a:6.1.5}{\hyperlink {e:6.1.5}{\bfseries 5.}} \mdseries $ (2x-4)\sqrt {25-x^2}-$\hfill \break $(x^2-4x+5)x/\sqrt {25-x^2}$\qquad 
\hypertarget {a:6.1.6}{\hyperlink {e:6.1.6}{\bfseries 6.}} \mdseries $-x/\sqrt {r^2-x^2}$\qquad 
\hypertarget {a:6.1.7}{\hyperlink {e:6.1.7}{\bfseries 7.}} \mdseries $2x^3/\sqrt {1+x^4}$\qquad 
\hypertarget {a:6.1.8}{\hyperlink {e:6.1.8}{\bfseries 8.}} \mdseries ${1\over 4\sqrt {x}(5-\sqrt {x})^{3/2}}$\qquad 
\hypertarget {a:6.1.9}{\hyperlink {e:6.1.9}{\bfseries 9.}} \mdseries $ 6+18x$\qquad 
\hypertarget {a:6.1.10}{\hyperlink {e:6.1.10}{\bfseries 10.}} \mdseries ${2 x + 1\over 1 - x }+{x^2 + x + 1\over (1 - x)^2}$\qquad 
\hypertarget {a:6.1.11}{\hyperlink {e:6.1.11}{\bfseries 11.}} \mdseries $ -1/\sqrt {25-x^2}-\sqrt {25-x^2}/x^2$\qquad 
\hypertarget {a:6.1.12}{\hyperlink {e:6.1.12}{\bfseries 12.}} \mdseries ${1\over 2}\left ({-169\over x^2}-1\right )\Big /\sqrt {{169\over x}-x}$\qquad 
\hypertarget {a:6.1.13}{\hyperlink {e:6.1.13}{\bfseries 13.}} \mdseries $ {3x^2-2x+1/x^2\over 2\sqrt {x^3-x^2-(1/x)}}$\qquad 
\hypertarget {a:6.1.14}{\hyperlink {e:6.1.14}{\bfseries 14.}} \mdseries $ {300 x \over (100-x^2)^{5/2}}$\qquad 
\hypertarget {a:6.1.15}{\hyperlink {e:6.1.15}{\bfseries 15.}} \mdseries $ { 1 + 3 x^2\over 3(x+x^3)^{2/3}}$\qquad 
\hypertarget {a:6.1.16}{\hyperlink {e:6.1.16}{\bfseries 16.}} \mdseries $ \left (4x(x^2+1)+{4x^3+4x\over 2\sqrt {1+(x^2+1)^2}}\right )\Big /$ \hfill \break $2\sqrt {(x^2+1)^2+\sqrt {1+(x^2+1)^2}}$\qquad 
\hypertarget {a:6.1.17}{\hyperlink {e:6.1.17}{\bfseries 17.}} \mdseries $5(x+8)^4$\qquad 
\hypertarget {a:6.1.18}{\hyperlink {e:6.1.18}{\bfseries 18.}} \mdseries $-3(4-x)^2$\qquad 
\hypertarget {a:6.1.19}{\hyperlink {e:6.1.19}{\bfseries 19.}} \mdseries $6x(x^2+5)^2$\qquad 
\hypertarget {a:6.1.20}{\hyperlink {e:6.1.20}{\bfseries 20.}} \mdseries $-12x(6-2x^2)^2$\qquad 
\hypertarget {a:6.1.21}{\hyperlink {e:6.1.21}{\bfseries 21.}} \mdseries $24x^2(1-4x^3)^{-3}$\qquad 
\hypertarget {a:6.1.22}{\hyperlink {e:6.1.22}{\bfseries 22.}} \mdseries $5+5/x^2$\qquad 
\hypertarget {a:6.1.23}{\hyperlink {e:6.1.23}{\bfseries 23.}} \mdseries $-8(4x-1)(2x^2-x+3)^{-3}$\qquad 
\hypertarget {a:6.1.24}{\hyperlink {e:6.1.24}{\bfseries 24.}} \mdseries $1/(x+1)^2$\qquad 
\hypertarget {a:6.1.25}{\hyperlink {e:6.1.25}{\bfseries 25.}} \mdseries $3(8x-2)/(4x^2-2x+1)^2$\qquad 
\hypertarget {a:6.1.26}{\hyperlink {e:6.1.26}{\bfseries 26.}} \mdseries $-3x^2+5x-1$\qquad 
\hypertarget {a:6.1.27}{\hyperlink {e:6.1.27}{\bfseries 27.}} \mdseries $6x(2x-4)^3+6(3x^2+1)(2x-4)^2$\qquad 
\hypertarget {a:6.1.28}{\hyperlink {e:6.1.28}{\bfseries 28.}} \mdseries $-2/(x-1)^2$\qquad 
\hypertarget {a:6.1.29}{\hyperlink {e:6.1.29}{\bfseries 29.}} \mdseries $4x/(x^2+1)^2$\qquad 
\hypertarget {a:6.1.30}{\hyperlink {e:6.1.30}{\bfseries 30.}} \mdseries $(x^2-6x+7)/(x-3)^2$\qquad 
\hypertarget {a:6.1.31}{\hyperlink {e:6.1.31}{\bfseries 31.}} \mdseries $-5/(3x-4)^2$\qquad 
\hypertarget {a:6.1.32}{\hyperlink {e:6.1.32}{\bfseries 32.}} \mdseries $60x^4+72x^3+18x^2+18x-6$\qquad 
\hypertarget {a:6.1.33}{\hyperlink {e:6.1.33}{\bfseries 33.}} \mdseries $(5-4x)/((2x+1)^2(x-3)^2)$\qquad 
\hypertarget {a:6.1.34}{\hyperlink {e:6.1.34}{\bfseries 34.}} \mdseries $1/(2(2+3x)^2)$\qquad 
\hypertarget {a:6.1.35}{\hyperlink {e:6.1.35}{\bfseries 35.}} \mdseries $56x^6+72x^5+110x^4+100x^3+60x^2+28x+6$\qquad 
\hypertarget {a:6.1.36}{\hyperlink {e:6.1.36}{\bfseries 36.}} \mdseries $y=23x/96-29/96$\qquad 
\hypertarget {a:6.1.37}{\hyperlink {e:6.1.37}{\bfseries 37.}} \mdseries $y=3-2x/3$\qquad 
\hypertarget {a:6.1.38}{\hyperlink {e:6.1.38}{\bfseries 38.}} \mdseries $y=13x/2-23/2$\qquad 
\hypertarget {a:6.1.39}{\hyperlink {e:6.1.39}{\bfseries 39.}} \mdseries $y=2x-11$\qquad 
\hypertarget {a:6.1.40}{\hyperlink {e:6.1.40}{\bfseries 40.}} \mdseries $y= {20+2\sqrt 5\over 5\sqrt {4+\sqrt 5}}\,x+{3\sqrt 5\over 5\sqrt {4+\sqrt 5}}$\qquad 
\subsection *{Answers for 6.2}
\hypertarget {a:6.2.1}{\hyperlink {e:6.2.1}{\bfseries 1.}} \mdseries $-x/y$\qquad 
\hypertarget {a:6.2.2}{\hyperlink {e:6.2.2}{\bfseries 2.}} \mdseries $x/y$\qquad 
\hypertarget {a:6.2.3}{\hyperlink {e:6.2.3}{\bfseries 3.}} \mdseries $-(2x+y)/(x+2y)$\qquad 
\hypertarget {a:6.2.4}{\hyperlink {e:6.2.4}{\bfseries 4.}} \mdseries $(2xy-3x^2-y^2)/(2xy-3y^2-x^2)$\qquad 
\hypertarget {a:6.2.5}{\hyperlink {e:6.2.5}{\bfseries 5.}} \mdseries $\frac {-2xy}{x^2-3y^2}$\qquad 
\hypertarget {a:6.2.6}{\hyperlink {e:6.2.6}{\bfseries 6.}} \mdseries $-\sqrt {y}/\sqrt {x}$\qquad 
\hypertarget {a:6.2.7}{\hyperlink {e:6.2.7}{\bfseries 7.}} \mdseries $\frac {y^{3/2}-2}{1-y^{1/2}3x/2}$\qquad 
\hypertarget {a:6.2.8}{\hyperlink {e:6.2.8}{\bfseries 8.}} \mdseries $-y^2/x^2$\qquad 
\hypertarget {a:6.2.9}{\hyperlink {e:6.2.9}{\bfseries 9.}} \mdseries $1$\qquad 
\hypertarget {a:6.2.10}{\hyperlink {e:6.2.10}{\bfseries 10.}} \mdseries $y=2x\pm 6$\qquad 
\hypertarget {a:6.2.11}{\hyperlink {e:6.2.11}{\bfseries 11.}} \mdseries $y=x/2\pm 3$\qquad 
\hypertarget {a:6.2.12}{\hyperlink {e:6.2.12}{\bfseries 12.}} \mdseries $(\sqrt 3,2\sqrt 3)$, $(-\sqrt 3,-2\sqrt 3)$, $(2\sqrt 3,\sqrt 3)$, $(-2\sqrt 3,-\sqrt 3)$\qquad 
\hypertarget {a:6.2.13}{\hyperlink {e:6.2.13}{\bfseries 13.}} \mdseries $y=7x/\sqrt 3-8/\sqrt 3$\qquad 
\hypertarget {a:6.2.14}{\hyperlink {e:6.2.14}{\bfseries 14.}} \mdseries $y=(-y_1^{1/3}x+y_1^{1/3}x_1+x_1^{1/3}y_1)/x_1^{1/3}$\qquad 
\hypertarget {a:6.2.15}{\hyperlink {e:6.2.15}{\bfseries 15.}} \mdseries $(y-y_1)/(x-x_1)=(x_1-2x_1^3-2x_1y_1^2)/(2y_1^3+2y_1x_1^2+y_1)$\qquad 
\subsection *{Answers for 6.2}
\hypertarget {a:6.2.1}{\hyperlink {e:6.2.1}{\bfseries 1.}} \mdseries ${-1\over 1+x^2}$\qquad 
\hypertarget {a:6.2.2}{\hyperlink {e:6.2.2}{\bfseries 2.}} \mdseries ${2x\over \sqrt {1-x^4}}$\qquad 
\hypertarget {a:6.2.3}{\hyperlink {e:6.2.3}{\bfseries 3.}} \mdseries ${e^x\over 1+e^{2x}}$\qquad 
\hypertarget {a:6.2.4}{\hyperlink {e:6.2.4}{\bfseries 4.}} \mdseries $-3x^2\cos (x^3)/\sqrt {1-\sin ^2(x^3)}$\qquad 
\hypertarget {a:6.2.5}{\hyperlink {e:6.2.5}{\bfseries 5.}} \mdseries ${2\over (\arcsin (x))\sqrt {1-x^2}}$\qquad 
\hypertarget {a:6.2.6}{\hyperlink {e:6.2.6}{\bfseries 6.}} \mdseries $-e^x/\sqrt {1-e^{2x}}$\qquad 
\hypertarget {a:6.2.7}{\hyperlink {e:6.2.7}{\bfseries 7.}} \mdseries $0$\qquad 
\hypertarget {a:6.2.8}{\hyperlink {e:6.2.8}{\bfseries 8.}} \mdseries ${(1+\ln x)x^x\over \ln 5(1+x^{2x})\arctan (x^x)}$\qquad 
\subsection *{Answers for 6.3}
\hypertarget {a:6.3.1}{\hyperlink {e:6.3.1}{\bfseries 1.}} \mdseries $(x+1)^3\sqrt {x^4+5}(3/(x+1) + 2x^3/(x^4+5))$\qquad 
\hypertarget {a:6.3.2}{\hyperlink {e:6.3.2}{\bfseries 2.}} \mdseries $(2/x + 5)x^2e^{5x}$\qquad 
\hypertarget {a:6.3.3}{\hyperlink {e:6.3.3}{\bfseries 3.}} \mdseries $2\ln (x)x^{\ln (x)-1}$\qquad 
\hypertarget {a:6.3.4}{\hyperlink {e:6.3.4}{\bfseries 4.}} \mdseries $(100 + 100 \ln (x))x^{100x}$\qquad 
\hypertarget {a:6.3.5}{\hyperlink {e:6.3.5}{\bfseries 5.}} \mdseries $(4+ 4\ln (3x)) (3x)^{4x}$\qquad 
\hypertarget {a:6.3.6}{\hyperlink {e:6.3.6}{\bfseries 6.}} \mdseries $((e^x)/x+ e^x\ln (x))x^{e^x}$\qquad 
\hypertarget {a:6.3.7}{\hyperlink {e:6.3.7}{\bfseries 7.}} \mdseries $\pi x^{\pi -1} + \pi ^x\ln (\pi )$\qquad 
\hypertarget {a:6.3.8}{\hyperlink {e:6.3.8}{\bfseries 8.}} \mdseries $(\ln (1+1/x) - 1/(x+1))(1+1/x)^x$\qquad 
\hypertarget {a:6.3.9}{\hyperlink {e:6.3.9}{\bfseries 9.}} \mdseries $(1/\ln (x)+\ln (\ln (x)))(\ln (x))^x$\qquad 
\hypertarget {a:6.3.10}{\hyperlink {e:6.3.10}{\bfseries 10.}} \mdseries $(f'(x)/f(x) + g'(x)/g(x) + h'(x)/h(x))f(x) g(x) h(x)$\qquad 
\subsection *{Answers for 7.1}
\hypertarget {a:7.1.1}{\hyperlink {e:7.1.1}{\bfseries 1.}} \mdseries $3/256$ m/s$^2$\qquad 
\hypertarget {a:7.1.2}{\hyperlink {e:7.1.2}{\bfseries 2.}} \mdseries on the Earth: $\approx 4.5$ s, $\approx 44$ m/s; on the Moon: $\approx 11.2$ s, $\approx 18$ m/s\qquad 
\hypertarget {a:7.1.3}{\hyperlink {e:7.1.3}{\bfseries 3.}} \mdseries average rate: $\approx -0.67$ gal/min; instantaneous rate: $\approx -0.71$ gal/min\qquad 
\hypertarget {a:7.1.4}{\hyperlink {e:7.1.4}{\bfseries 4.}} \mdseries $\approx 9.5$ s; $\approx 48$ km/h.\qquad 
\hypertarget {a:7.1.5}{\hyperlink {e:7.1.5}{\bfseries 5.}} \mdseries $p(t) = 300\cdot 3^{4t}$\qquad 
\hypertarget {a:7.1.6}{\hyperlink {e:7.1.6}{\bfseries 6.}} \mdseries $\approx -.02$ mg/ml per hour\qquad 
\hypertarget {a:7.1.7}{\hyperlink {e:7.1.7}{\bfseries 7.}} \mdseries $\approx 39$ cm/day; $\approx 0$ cm/day\qquad 
\subsection *{Answers for 7.2}
\hypertarget {a:7.2.1}{\hyperlink {e:7.2.1}{\bfseries 1.}} \mdseries $1/(16\pi )$ cm/s\qquad 
\hypertarget {a:7.2.2}{\hyperlink {e:7.2.2}{\bfseries 2.}} \mdseries $3/(1000\pi )$ meters/second\qquad 
\hypertarget {a:7.2.3}{\hyperlink {e:7.2.3}{\bfseries 3.}} \mdseries $1/4$ m/s\qquad 
\hypertarget {a:7.2.4}{\hyperlink {e:7.2.4}{\bfseries 4.}} \mdseries $6/25$ m/s\qquad 
\hypertarget {a:7.2.5}{\hyperlink {e:7.2.5}{\bfseries 5.}} \mdseries $80\pi $ mi/min\qquad 
\hypertarget {a:7.2.6}{\hyperlink {e:7.2.6}{\bfseries 6.}} \mdseries $3\sqrt 5$ ft/s\qquad 
\hypertarget {a:7.2.7}{\hyperlink {e:7.2.7}{\bfseries 7.}} \mdseries $20/(3\pi )$ cm/s\qquad 
\hypertarget {a:7.2.8}{\hyperlink {e:7.2.8}{\bfseries 8.}} \mdseries $13/20$ ft/s\qquad 
\hypertarget {a:7.2.9}{\hyperlink {e:7.2.9}{\bfseries 9.}} \mdseries $5\sqrt {10}/2$ m/s\qquad 
\hypertarget {a:7.2.10}{\hyperlink {e:7.2.10}{\bfseries 10.}} \mdseries $75/64$ m/min\qquad 
\hypertarget {a:7.2.11}{\hyperlink {e:7.2.11}{\bfseries 11.}} \mdseries tip: 6 ft/s, length: $5/2$ ft/s\qquad 
\hypertarget {a:7.2.12}{\hyperlink {e:7.2.12}{\bfseries 12.}} \mdseries tip: $20/11$ m/s, length: $9/11$ m/s\qquad 
\hypertarget {a:7.2.13}{\hyperlink {e:7.2.13}{\bfseries 13.}} \mdseries $380/\sqrt 3-150\approx 69.4$ mph\qquad 
\hypertarget {a:7.2.14}{\hyperlink {e:7.2.14}{\bfseries 14.}} \mdseries $500/\sqrt 3-200\approx 88.7$ km/hr\qquad 
\hypertarget {a:7.2.15}{\hyperlink {e:7.2.15}{\bfseries 15.}} \mdseries $136\sqrt {475}/19\approx 156$ km/hr\qquad 
\hypertarget {a:7.2.16}{\hyperlink {e:7.2.16}{\bfseries 16.}} \mdseries $-50$ m/s\qquad 
\hypertarget {a:7.2.17}{\hyperlink {e:7.2.17}{\bfseries 17.}} \mdseries $68$ m/s\qquad 
\hypertarget {a:7.2.18}{\hyperlink {e:7.2.18}{\bfseries 18.}} \mdseries $3800/\sqrt {329}\approx 210$ km/hr\qquad 
\hypertarget {a:7.2.19}{\hyperlink {e:7.2.19}{\bfseries 19.}} \mdseries \hbox {$820/\sqrt {329}+150\sqrt {57}/\sqrt {47}\approx 210$ km/hr}\qquad 
\hypertarget {a:7.2.20}{\hyperlink {e:7.2.20}{\bfseries 20.}} \mdseries $4000/49$ m/s\qquad 
\subsection *{Answers for 8.1}
\hypertarget {a:8.1.1}{\hyperlink {e:8.1.1}{\bfseries 1.}} \mdseries min at $x=1/2$\qquad 
\hypertarget {a:8.1.2}{\hyperlink {e:8.1.2}{\bfseries 2.}} \mdseries min at $x=-1$, max at $x=1$\qquad 
\hypertarget {a:8.1.3}{\hyperlink {e:8.1.3}{\bfseries 3.}} \mdseries max at $x=2$, min at $x=4$\qquad 
\hypertarget {a:8.1.4}{\hyperlink {e:8.1.4}{\bfseries 4.}} \mdseries min at $x=\pm 1$, max at $x=0$.\qquad 
\hypertarget {a:8.1.5}{\hyperlink {e:8.1.5}{\bfseries 5.}} \mdseries min at $x=1$\qquad 
\hypertarget {a:8.1.6}{\hyperlink {e:8.1.6}{\bfseries 6.}} \mdseries none\qquad 
\hypertarget {a:8.1.7}{\hyperlink {e:8.1.7}{\bfseries 7.}} \mdseries min at $x=0$, max at $x=\frac {3\pm \sqrt {17}}{2}$\qquad 
\hypertarget {a:8.1.8}{\hyperlink {e:8.1.8}{\bfseries 8.}} \mdseries none\qquad 
\hypertarget {a:8.1.9}{\hyperlink {e:8.1.9}{\bfseries 9.}} \mdseries local max at $x=5$\qquad 
\hypertarget {a:8.1.10}{\hyperlink {e:8.1.10}{\bfseries 10.}} \mdseries local min at $x=49$\qquad 
\hypertarget {a:8.1.11}{\hyperlink {e:8.1.11}{\bfseries 11.}} \mdseries local min at $x=0$\qquad 
\hypertarget {a:8.1.12}{\hyperlink {e:8.1.12}{\bfseries 12.}} \mdseries one\qquad 
\hypertarget {a:8.1.13}{\hyperlink {e:8.1.13}{\bfseries 13.}} \mdseries if $c\ge 0$, then there are no local extrema; if $c<0$ then there is a local max at $x=-\sqrt {\frac {|c|}{3}}$ and a local min at $x=\sqrt {\frac {|c|}{3}}$\qquad 
\subsection *{Answers for 8.2}
\hypertarget {a:8.2.1}{\hyperlink {e:8.2.1}{\bfseries 1.}} \mdseries min at $x=1/2$\qquad 
\hypertarget {a:8.2.2}{\hyperlink {e:8.2.2}{\bfseries 2.}} \mdseries min at $x=-1$, max at $x=1$\qquad 
\hypertarget {a:8.2.3}{\hyperlink {e:8.2.3}{\bfseries 3.}} \mdseries max at $x=2$, min at $x=4$\qquad 
\hypertarget {a:8.2.4}{\hyperlink {e:8.2.4}{\bfseries 4.}} \mdseries min at $x=\pm 1$, max at $x=0$.\qquad 
\hypertarget {a:8.2.5}{\hyperlink {e:8.2.5}{\bfseries 5.}} \mdseries min at $x=1$\qquad 
\hypertarget {a:8.2.6}{\hyperlink {e:8.2.6}{\bfseries 6.}} \mdseries none\qquad 
\hypertarget {a:8.2.7}{\hyperlink {e:8.2.7}{\bfseries 7.}} \mdseries max at $x=0$, min at $x=\pm 11$\qquad 
\hypertarget {a:8.2.8}{\hyperlink {e:8.2.8}{\bfseries 8.}} \mdseries $f'(x) = 2ax + b$, this has only one root and hence one critical point; $a<0$ to guarantee a maximum.\qquad 
\subsection *{Answers for 8.3}
\hypertarget {a:8.3.1}{\hyperlink {e:8.3.1}{\bfseries 1.}} \mdseries concave up everywhere\qquad 
\hypertarget {a:8.3.2}{\hyperlink {e:8.3.2}{\bfseries 2.}} \mdseries concave up when $x<0$, concave down when $x>0$\qquad 
\hypertarget {a:8.3.3}{\hyperlink {e:8.3.3}{\bfseries 3.}} \mdseries concave down when $x<3$, concave up when $x>3$\qquad 
\hypertarget {a:8.3.4}{\hyperlink {e:8.3.4}{\bfseries 4.}} \mdseries concave up when $x<-1/\sqrt 3$ or $x>1/\sqrt 3$, concave down when $-1/\sqrt 3<x<1/\sqrt 3$\qquad 
\hypertarget {a:8.3.5}{\hyperlink {e:8.3.5}{\bfseries 5.}} \mdseries concave up when $x<0$ or $x>2/3$, concave down when $0<x<2/3$\qquad 
\hypertarget {a:8.3.6}{\hyperlink {e:8.3.6}{\bfseries 6.}} \mdseries concave up when $x<0$, concave down when $x>0$\qquad 
\hypertarget {a:8.3.7}{\hyperlink {e:8.3.7}{\bfseries 7.}} \mdseries concave up when $x<-1$ or $x>1$, concave down when $-1<x<0$ or $0<x<1$\qquad 
\hypertarget {a:8.3.8}{\hyperlink {e:8.3.8}{\bfseries 8.}} \mdseries concave up on $(0,\infty )$, concave down on $(-\infty ,0)$\qquad 
\hypertarget {a:8.3.9}{\hyperlink {e:8.3.9}{\bfseries 9.}} \mdseries concave up on $(0,\infty )$, concave down on $(-\infty ,0)$\qquad 
\hypertarget {a:8.3.10}{\hyperlink {e:8.3.10}{\bfseries 10.}} \mdseries concave up on $(-\infty ,-1)$ and $(0,\infty )$, concave down on $(-1,0)$\qquad 
\hypertarget {a:8.3.11}{\hyperlink {e:8.3.11}{\bfseries 11.}} \mdseries up/incr: $(3,\infty )$, up/decr: $(-\infty ,0)$, $(2,3)$, down/decr: $(0,2)$\qquad 
\subsection *{Answers for 8.4}
\hypertarget {a:8.4.1}{\hyperlink {e:8.4.1}{\bfseries 1.}} \mdseries min at $x=1/2$\qquad 
\hypertarget {a:8.4.2}{\hyperlink {e:8.4.2}{\bfseries 2.}} \mdseries min at $x=-1$, max at $x=1$\qquad 
\hypertarget {a:8.4.3}{\hyperlink {e:8.4.3}{\bfseries 3.}} \mdseries max at $x=2$, min at $x=4$\qquad 
\hypertarget {a:8.4.4}{\hyperlink {e:8.4.4}{\bfseries 4.}} \mdseries min at $x=\pm 1$, max at $x=0$.\qquad 
\hypertarget {a:8.4.5}{\hyperlink {e:8.4.5}{\bfseries 5.}} \mdseries min at $x=1$\qquad 
\hypertarget {a:8.4.6}{\hyperlink {e:8.4.6}{\bfseries 6.}} \mdseries none\qquad 
\hypertarget {a:8.4.7}{\hyperlink {e:8.4.7}{\bfseries 7.}} \mdseries none\qquad 
\hypertarget {a:8.4.8}{\hyperlink {e:8.4.8}{\bfseries 8.}} \mdseries max at $-5^{-1/4}$, min at $5^{-1/4}$\qquad 
\hypertarget {a:8.4.9}{\hyperlink {e:8.4.9}{\bfseries 9.}} \mdseries max at $-1$, min at $1$\qquad 
\hypertarget {a:8.4.10}{\hyperlink {e:8.4.10}{\bfseries 10.}} \mdseries min at $2^{-1/3}$\qquad 
\subsection *{Answers for 8.5}
\hypertarget {a:8.5.1}{\hyperlink {e:8.5.1}{\bfseries 1.}} \mdseries $y$-intercept at $(0,0)$; no vertical asymptotes; critical points: $x=\pm 1/\sqrt [4]{5}$; local max at $x=-1/\sqrt [4]{5}$, local min at $x=-1/\sqrt [4]{5}$; increasing on $(-\infty ,-1/\sqrt [4]{5})$, decreasing on $(-1/\sqrt [4]{5},1/\sqrt [4]{5})$, increasing on $(1/\sqrt [4]{5},\infty )$; concave down on $(-\infty ,0)$, concave up on $(0, \infty )$; root at $x=0$; no horizontal asymptotes; interval for sketch: $[-1.2,1.2]$ (answers may vary)\qquad 
\hypertarget {a:8.5.2}{\hyperlink {e:8.5.2}{\bfseries 2.}} \mdseries $y$-intercept at $(0,0)$; no vertical asymptotes; no critical points; no local extrema; increasing on $(-\infty ,\infty )$; concave down on $(-\infty ,0)$, concave up on $(0, \infty )$; roots at $x=0$; no horizontal asymptotes; interval for sketch: $[-3,3]$ (answers may vary)\qquad 
\hypertarget {a:8.5.3}{\hyperlink {e:8.5.3}{\bfseries 3.}} \mdseries $y$-intercept at $(0,0)$; no vertical asymptotes; critical points: $x= 1$; local max at $x=1$; increasing on $[0,1)$, decreasing on $(1,\infty )$; concave down on $[0,\infty )$; roots at $x=0$, $x=4$; no horizontal asymptotes; interval for sketch: $[0,6]$ (answers may vary)\qquad 
\hypertarget {a:8.5.4}{\hyperlink {e:8.5.4}{\bfseries 4.}} \mdseries $y$-intercept at $(0,0)$; no vertical asymptotes; critical points: $x=-3$, $x= -1$; local max at $x=-3$, local min at $x=-1$; increasing on $(-\infty ,-3)$, decreasing on $(-3,-1)$, increasing on $(-1,\infty )$; concave down on $(-\infty ,-2)$, concave up on $(-2, \infty )$; roots at $x=-3$, $x=0$; no horizontal asymptotes; interval for sketch: $[-5,3]$ (answers may vary)\qquad 
\hypertarget {a:8.5.5}{\hyperlink {e:8.5.5}{\bfseries 5.}} \mdseries $y$-intercept at $(0,5)$; no vertical asymptotes; critical points: $x=-1$, $x= 3$; local max at $x=-1$, local min at $x=3$; increasing on $(-\infty ,-1)$, decreasing on $(-1,3)$, increasing on $(3,\infty )$; concave down on $(-\infty ,1)$, concave up on $(1, \infty )$; roots are too difficult to be determined---cubic formula could be used; no horizontal asymptotes; interval for sketch: $[-2,5]$ (answers may vary)\qquad 
\hypertarget {a:8.5.6}{\hyperlink {e:8.5.6}{\bfseries 6.}} \mdseries $y$-intercept at $(0,0)$; no vertical asymptotes; critical points: $x=0$, $x=1$, $x=3$; local max at $x=1$, local min at $x=3$; increasing on $(-\infty ,0)$ and $(0,1)$, decreasing on $(1,3)$, increasing on $(3,\infty )$; concave down on $(-\infty ,0)$, concave up on $(0, (3-\sqrt {3})/2)$, concave down on $((3-\sqrt {3})/2,(3+\sqrt {3})/2)$, concave up on $((3+\sqrt {3})/2,\infty )$; roots at $x=0$, $x= \frac {5\pm \sqrt {5}}{2}$; no horizontal asymptotes; interval for sketch: $[-1,4]$ (answers may vary)\qquad 
\hypertarget {a:8.5.7}{\hyperlink {e:8.5.7}{\bfseries 7.}} \mdseries no $y$-intercept; vertical asymptote at $x=0$; critical points: $x=0$, $x=\pm 1$; local max at $x=-1$, local min at $1$; increasing on $(-\infty ,-1)$, decreasing on $(-1,0)\cup (0,1)$, increasing on $(1,\infty )$; concave down on $(-\infty ,0)$, concave up on $(0, \infty )$; no roots; no horizontal asymptotes; interval for sketch: $[-2,2]$ (answers may vary)\qquad 
\hypertarget {a:8.5.8}{\hyperlink {e:8.5.8}{\bfseries 8.}} \mdseries no $y$-intercept; vertical asymptote at $x=0$; critical points: $x=0$, $x=\frac {1}{\sqrt [3]{2}}$; local min at $x=\frac {1}{\sqrt [3]{2}}$; decreasing on $(-\infty ,0)$, decreasing on $(0,\frac {1}{\sqrt [3]{2}})$, increasing on $(\frac {1}{\sqrt [3]{2}},\infty )$; concave up on $(-\infty ,-1)$, concave down on $(-1,0)$, concave up on $(0,\infty )$; root at $x=-1$; no horizontal asymptotes; interval for sketch: $[-3,2]$ (answers may vary)\qquad 
\subsection *{Answers for 9.1}
\hypertarget {a:9.1.1}{\hyperlink {e:9.1.1}{\bfseries 1.}} \mdseries $0$\qquad 
\hypertarget {a:9.1.2}{\hyperlink {e:9.1.2}{\bfseries 2.}} \mdseries $-1$\qquad 
\hypertarget {a:9.1.3}{\hyperlink {e:9.1.3}{\bfseries 3.}} \mdseries $\frac {1}{2}$\qquad 
\hypertarget {a:9.1.4}{\hyperlink {e:9.1.4}{\bfseries 4.}} \mdseries $-3$\qquad 
\hypertarget {a:9.1.5}{\hyperlink {e:9.1.5}{\bfseries 5.}} \mdseries $-2$\qquad 
\hypertarget {a:9.1.6}{\hyperlink {e:9.1.6}{\bfseries 6.}} \mdseries $-\infty $\qquad 
\hypertarget {a:9.1.7}{\hyperlink {e:9.1.7}{\bfseries 7.}} \mdseries $\pi $\qquad 
\hypertarget {a:9.1.8}{\hyperlink {e:9.1.8}{\bfseries 8.}} \mdseries $0$\qquad 
\hypertarget {a:9.1.9}{\hyperlink {e:9.1.9}{\bfseries 9.}} \mdseries $0$\qquad 
\hypertarget {a:9.1.10}{\hyperlink {e:9.1.10}{\bfseries 10.}} \mdseries $17$\qquad 
\hypertarget {a:9.1.11}{\hyperlink {e:9.1.11}{\bfseries 11.}} \mdseries After 10 years, $\approx 174$ cats; after 50 years, $\approx 199$ cats; after 100 years, $\approx 200$ cats; after 1000 years, $\approx 200$ cats; in the sense that the population of cats cannot grow indefinitely this is somewhat realistic.\qquad 
\hypertarget {a:9.1.12}{\hyperlink {e:9.1.12}{\bfseries 12.}} \mdseries The amplitude goes to zero.\qquad 
\subsection *{Answers for 9.2}
\hypertarget {a:9.2.1}{\hyperlink {e:9.2.1}{\bfseries 1.}} \mdseries $0$\qquad 
\hypertarget {a:9.2.2}{\hyperlink {e:9.2.2}{\bfseries 2.}} \mdseries $-1$\qquad 
\hypertarget {a:9.2.3}{\hyperlink {e:9.2.3}{\bfseries 3.}} \mdseries $\frac {1}{2}$\qquad 
\hypertarget {a:9.2.4}{\hyperlink {e:9.2.4}{\bfseries 4.}} \mdseries $-3$\qquad 
\hypertarget {a:9.2.5}{\hyperlink {e:9.2.5}{\bfseries 5.}} \mdseries $-2$\qquad 
\hypertarget {a:9.2.6}{\hyperlink {e:9.2.6}{\bfseries 6.}} \mdseries $-\infty $\qquad 
\hypertarget {a:9.2.7}{\hyperlink {e:9.2.7}{\bfseries 7.}} \mdseries $\pi $\qquad 
\hypertarget {a:9.2.8}{\hyperlink {e:9.2.8}{\bfseries 8.}} \mdseries $0$\qquad 
\hypertarget {a:9.2.9}{\hyperlink {e:9.2.9}{\bfseries 9.}} \mdseries $0$\qquad 
\hypertarget {a:9.2.10}{\hyperlink {e:9.2.10}{\bfseries 10.}} \mdseries $17$\qquad 
\hypertarget {a:9.2.11}{\hyperlink {e:9.2.11}{\bfseries 11.}} \mdseries After 10 years, $\approx 174$ cats; after 50 years, $\approx 199$ cats; after 100 years, $\approx 200$ cats; after 1000 years, $\approx 200$ cats; in the sense that the population of cats cannot grow indefinitely this is somewhat realistic.\qquad 
\hypertarget {a:9.2.12}{\hyperlink {e:9.2.12}{\bfseries 12.}} \mdseries The amplitude goes to zero.\qquad 
\subsection *{Answers for 9.3}
\hypertarget {a:9.3.1}{\hyperlink {e:9.3.1}{\bfseries 1.}} \mdseries $0$\qquad 
\hypertarget {a:9.3.2}{\hyperlink {e:9.3.2}{\bfseries 2.}} \mdseries $\infty $\qquad 
\hypertarget {a:9.3.3}{\hyperlink {e:9.3.3}{\bfseries 3.}} \mdseries $1$\qquad 
\hypertarget {a:9.3.4}{\hyperlink {e:9.3.4}{\bfseries 4.}} \mdseries $0$\qquad 
\hypertarget {a:9.3.5}{\hyperlink {e:9.3.5}{\bfseries 5.}} \mdseries $0$\qquad 
\hypertarget {a:9.3.6}{\hyperlink {e:9.3.6}{\bfseries 6.}} \mdseries 1\qquad 
\hypertarget {a:9.3.7}{\hyperlink {e:9.3.7}{\bfseries 7.}} \mdseries $1/6$\qquad 
\hypertarget {a:9.3.8}{\hyperlink {e:9.3.8}{\bfseries 8.}} \mdseries $-\infty $\qquad 
\hypertarget {a:9.3.9}{\hyperlink {e:9.3.9}{\bfseries 9.}} \mdseries $1/16$\qquad 
\hypertarget {a:9.3.10}{\hyperlink {e:9.3.10}{\bfseries 10.}} \mdseries $1/3$\qquad 
\hypertarget {a:9.3.11}{\hyperlink {e:9.3.11}{\bfseries 11.}} \mdseries $0$\qquad 
\hypertarget {a:9.3.12}{\hyperlink {e:9.3.12}{\bfseries 12.}} \mdseries $3/2$\qquad 
\hypertarget {a:9.3.13}{\hyperlink {e:9.3.13}{\bfseries 13.}} \mdseries $-1/4$\qquad 
\hypertarget {a:9.3.14}{\hyperlink {e:9.3.14}{\bfseries 14.}} \mdseries $-3$\qquad 
\hypertarget {a:9.3.15}{\hyperlink {e:9.3.15}{\bfseries 15.}} \mdseries $1/2$\qquad 
\hypertarget {a:9.3.16}{\hyperlink {e:9.3.16}{\bfseries 16.}} \mdseries $0$\qquad 
\hypertarget {a:9.3.17}{\hyperlink {e:9.3.17}{\bfseries 17.}} \mdseries $0$\qquad 
\hypertarget {a:9.3.18}{\hyperlink {e:9.3.18}{\bfseries 18.}} \mdseries $-1/2$\qquad 
\hypertarget {a:9.3.19}{\hyperlink {e:9.3.19}{\bfseries 19.}} \mdseries $5$\qquad 
\hypertarget {a:9.3.20}{\hyperlink {e:9.3.20}{\bfseries 20.}} \mdseries $\infty $\qquad 
\hypertarget {a:9.3.21}{\hyperlink {e:9.3.21}{\bfseries 21.}} \mdseries $\infty $\qquad 
\hypertarget {a:9.3.22}{\hyperlink {e:9.3.22}{\bfseries 22.}} \mdseries $2/7$\qquad 
\hypertarget {a:9.3.23}{\hyperlink {e:9.3.23}{\bfseries 23.}} \mdseries $2$\qquad 
\hypertarget {a:9.3.24}{\hyperlink {e:9.3.24}{\bfseries 24.}} \mdseries $-\infty $\qquad 
\hypertarget {a:9.3.25}{\hyperlink {e:9.3.25}{\bfseries 25.}} \mdseries $1$\qquad 
\hypertarget {a:9.3.26}{\hyperlink {e:9.3.26}{\bfseries 26.}} \mdseries $1$\qquad 
\hypertarget {a:9.3.27}{\hyperlink {e:9.3.27}{\bfseries 27.}} \mdseries $2$\qquad 
\hypertarget {a:9.3.28}{\hyperlink {e:9.3.28}{\bfseries 28.}} \mdseries $1$\qquad 
\hypertarget {a:9.3.29}{\hyperlink {e:9.3.29}{\bfseries 29.}} \mdseries $0$\qquad 
\hypertarget {a:9.3.30}{\hyperlink {e:9.3.30}{\bfseries 30.}} \mdseries $1/2$\qquad 
\hypertarget {a:9.3.31}{\hyperlink {e:9.3.31}{\bfseries 31.}} \mdseries $2$\qquad 
\hypertarget {a:9.3.32}{\hyperlink {e:9.3.32}{\bfseries 32.}} \mdseries $0$\qquad 
\hypertarget {a:9.3.33}{\hyperlink {e:9.3.33}{\bfseries 33.}} \mdseries $\infty $\qquad 
\hypertarget {a:9.3.34}{\hyperlink {e:9.3.34}{\bfseries 34.}} \mdseries $1/2$\qquad 
\hypertarget {a:9.3.35}{\hyperlink {e:9.3.35}{\bfseries 35.}} \mdseries $0$\qquad 
\hypertarget {a:9.3.36}{\hyperlink {e:9.3.36}{\bfseries 36.}} \mdseries $1/2$\qquad 
\hypertarget {a:9.3.37}{\hyperlink {e:9.3.37}{\bfseries 37.}} \mdseries $5$\qquad 
\hypertarget {a:9.3.38}{\hyperlink {e:9.3.38}{\bfseries 38.}} \mdseries $2\sqrt 2$\qquad 
\hypertarget {a:9.3.39}{\hyperlink {e:9.3.39}{\bfseries 39.}} \mdseries $-1/2$\qquad 
\hypertarget {a:9.3.40}{\hyperlink {e:9.3.40}{\bfseries 40.}} \mdseries $2$\qquad 
\hypertarget {a:9.3.41}{\hyperlink {e:9.3.41}{\bfseries 41.}} \mdseries $0$\qquad 
\hypertarget {a:9.3.42}{\hyperlink {e:9.3.42}{\bfseries 42.}} \mdseries $\infty $\qquad 
\hypertarget {a:9.3.43}{\hyperlink {e:9.3.43}{\bfseries 43.}} \mdseries $0$\qquad 
\hypertarget {a:9.3.44}{\hyperlink {e:9.3.44}{\bfseries 44.}} \mdseries $3/2$\qquad 
\hypertarget {a:9.3.45}{\hyperlink {e:9.3.45}{\bfseries 45.}} \mdseries $\infty $\qquad 
\hypertarget {a:9.3.46}{\hyperlink {e:9.3.46}{\bfseries 46.}} \mdseries $5$\qquad 
\hypertarget {a:9.3.47}{\hyperlink {e:9.3.47}{\bfseries 47.}} \mdseries $-1/2$\qquad 
\hypertarget {a:9.3.48}{\hyperlink {e:9.3.48}{\bfseries 48.}} \mdseries does not exist\qquad 
\hypertarget {a:9.3.49}{\hyperlink {e:9.3.49}{\bfseries 49.}} \mdseries $\infty $\qquad 
\subsection *{Answers for 10.1}
\hypertarget {a:10.1.1}{\hyperlink {e:10.1.1}{\bfseries 1.}} \mdseries max at $(1/4,1/8)$, min at $(1,-1)$\qquad 
\hypertarget {a:10.1.2}{\hyperlink {e:10.1.2}{\bfseries 2.}} \mdseries max at $(-1,1)$, min at $(1,-1)$\qquad 
\hypertarget {a:10.1.3}{\hyperlink {e:10.1.3}{\bfseries 3.}} \mdseries max at $(3,1)$, min at $(1,-1)$\qquad 
\hypertarget {a:10.1.4}{\hyperlink {e:10.1.4}{\bfseries 4.}} \mdseries max at $(-1+1/\sqrt {3},2/(3\sqrt {3}))$, min at $(-1-1/\sqrt {3},-2/(3\sqrt {3}))$\qquad 
\hypertarget {a:10.1.5}{\hyperlink {e:10.1.5}{\bfseries 5.}} \mdseries max at $(\pi /2,1)$ and $(3\pi /2,1)$, min at $(\pi ,0)$\qquad 
\hypertarget {a:10.1.6}{\hyperlink {e:10.1.6}{\bfseries 6.}} \mdseries max at $(1,\pi /4)$, min at $(-1,-\pi /4)$\qquad 
\hypertarget {a:10.1.7}{\hyperlink {e:10.1.7}{\bfseries 7.}} \mdseries max at $(\pi /2,e)$, min at $(-\pi /2,1/e)$\qquad 
\hypertarget {a:10.1.8}{\hyperlink {e:10.1.8}{\bfseries 8.}} \mdseries max at $(0,0)$, min at $(\pi /3,-\ln (2))$\qquad 
\hypertarget {a:10.1.9}{\hyperlink {e:10.1.9}{\bfseries 9.}} \mdseries max at $(2,5)$, min at $(0,1)$\qquad 
\hypertarget {a:10.1.10}{\hyperlink {e:10.1.10}{\bfseries 10.}} \mdseries max at $(3,4)$, min at $(4,1)$\qquad 
\subsection *{Answers for 10.2}
\hypertarget {a:10.2.1}{\hyperlink {e:10.2.1}{\bfseries 1.}} \mdseries $25\times 25$\qquad 
\hypertarget {a:10.2.2}{\hyperlink {e:10.2.2}{\bfseries 2.}} \mdseries $P/4\times P/4$\qquad 
\hypertarget {a:10.2.3}{\hyperlink {e:10.2.3}{\bfseries 3.}} \mdseries $w=l=2\cdot 5^{2/3}$, $h=5^{2/3}$, $h/w=1/2$\qquad 
\hypertarget {a:10.2.4}{\hyperlink {e:10.2.4}{\bfseries 4.}} \mdseries $\root 3\of {100}\times \root 3\of {100}\times 2\root 3\of {100}$, $h/s=2$\qquad 
\hypertarget {a:10.2.5}{\hyperlink {e:10.2.5}{\bfseries 5.}} \mdseries $w=l=2^{1/3}V^{1/3}$, $h=V^{1/3}/2^{2/3}$, $h/w=1/2$\qquad 
\hypertarget {a:10.2.6}{\hyperlink {e:10.2.6}{\bfseries 6.}} \mdseries $1250$ square feet\qquad 
\hypertarget {a:10.2.7}{\hyperlink {e:10.2.7}{\bfseries 7.}} \mdseries $l^2/8$ square feet\qquad 
\hypertarget {a:10.2.8}{\hyperlink {e:10.2.8}{\bfseries 8.}} \mdseries \$5000\qquad 
\hypertarget {a:10.2.9}{\hyperlink {e:10.2.9}{\bfseries 9.}} \mdseries $100$\qquad 
\hypertarget {a:10.2.10}{\hyperlink {e:10.2.10}{\bfseries 10.}} \mdseries $r^2$\qquad 
\hypertarget {a:10.2.11}{\hyperlink {e:10.2.11}{\bfseries 11.}} \mdseries $h/r=2$\qquad 
\hypertarget {a:10.2.12}{\hyperlink {e:10.2.12}{\bfseries 12.}} \mdseries $h/r=2$\qquad 
\hypertarget {a:10.2.13}{\hyperlink {e:10.2.13}{\bfseries 13.}} \mdseries $r=5$, $h=40/\pi $, $h/r=8/\pi $\qquad 
\hypertarget {a:10.2.14}{\hyperlink {e:10.2.14}{\bfseries 14.}} \mdseries $8/\pi $\qquad 
\hypertarget {a:10.2.15}{\hyperlink {e:10.2.15}{\bfseries 15.}} \mdseries $4/27$\qquad 
\hypertarget {a:10.2.16}{\hyperlink {e:10.2.16}{\bfseries 16.}} \mdseries Go direct from $A$ to $D$.\qquad 
\hypertarget {a:10.2.17}{\hyperlink {e:10.2.17}{\bfseries 17.}} \mdseries (a) 2, (b) $7/2$\qquad 
\hypertarget {a:10.2.18}{\hyperlink {e:10.2.18}{\bfseries 18.}} \mdseries $\ds {\sqrt 3\over 6}\times {\sqrt 3\over 6}+{1\over 2}\times {1\over 4}-{\sqrt 3\over 12}$\qquad 
\hypertarget {a:10.2.19}{\hyperlink {e:10.2.19}{\bfseries 19.}} \mdseries (a) $a/6$, (b) $(a+b-\sqrt {a^2-ab+b^2})/6$\qquad 
\hypertarget {a:10.2.20}{\hyperlink {e:10.2.20}{\bfseries 20.}} \mdseries $1.5$ meters wide by $1.25$ meters tall\qquad 
\hypertarget {a:10.2.21}{\hyperlink {e:10.2.21}{\bfseries 21.}} \mdseries If $k\le 2/\pi $ the ratio is $(2-k\pi )/4$; if $k\ge 2/\pi $, the ratio is zero: the window should be semicircular with no rectangular part.\qquad 
\hypertarget {a:10.2.22}{\hyperlink {e:10.2.22}{\bfseries 22.}} \mdseries $a/b$\qquad 
\hypertarget {a:10.2.23}{\hyperlink {e:10.2.23}{\bfseries 23.}} \mdseries $w=2r/\sqrt 3$, $h=2\sqrt 2r/\sqrt 3$\qquad 
\hypertarget {a:10.2.24}{\hyperlink {e:10.2.24}{\bfseries 24.}} \mdseries $1/\sqrt 3\approx 58\%$\qquad 
\hypertarget {a:10.2.25}{\hyperlink {e:10.2.25}{\bfseries 25.}} \mdseries $18\times 18\times 36$\qquad 
\hypertarget {a:10.2.26}{\hyperlink {e:10.2.26}{\bfseries 26.}} \mdseries $r=5/(2\pi )^{1/3}\approx 2.7\hbox { cm}$,\hfill \break $h=5\cdot 2^{5/3}/\pi ^{1/3}=4r\approx 10.8 \hbox { cm}$\qquad 
\hypertarget {a:10.2.27}{\hyperlink {e:10.2.27}{\bfseries 27.}} \mdseries $h={750\over \pi }\left ({2\pi ^2\over 750^2}\right )^{1/3}$, $r=\left ({750^2\over 2\pi ^2}\right )^{1/6}$\qquad 
\hypertarget {a:10.2.28}{\hyperlink {e:10.2.28}{\bfseries 28.}} \mdseries $h/r=\sqrt 2$\qquad 
\hypertarget {a:10.2.29}{\hyperlink {e:10.2.29}{\bfseries 29.}} \mdseries The ratio of the volume of the sphere to the volume of the cone is $1033/4096+33/4096\sqrt {17}\approx 0.2854$, so the cone occupies approximately $28.54\%$ of the sphere.\qquad 
\hypertarget {a:10.2.30}{\hyperlink {e:10.2.30}{\bfseries 30.}} \mdseries $P$ should be at distance $c\root 3\of {a} / (\root 3\of {a} + \root 3\of {b})$ from charge $A$.\qquad 
\hypertarget {a:10.2.31}{\hyperlink {e:10.2.31}{\bfseries 31.}} \mdseries $1/2$\qquad 
\hypertarget {a:10.2.32}{\hyperlink {e:10.2.32}{\bfseries 32.}} \mdseries \$7000\qquad 
\subsection *{Answers for 11.1}
\hypertarget {a:11.1.1}{\hyperlink {e:11.1.1}{\bfseries 1.}} \mdseries $\sin (0.1/2)\approx 0.05$\qquad 
\hypertarget {a:11.1.2}{\hyperlink {e:11.1.2}{\bfseries 2.}} \mdseries $\sqrt [3]{10}\approx 2.17$\qquad 
\hypertarget {a:11.1.3}{\hyperlink {e:11.1.3}{\bfseries 3.}} \mdseries $\sqrt [5]{250}\approx 3.017$\qquad 
\hypertarget {a:11.1.4}{\hyperlink {e:11.1.4}{\bfseries 4.}} \mdseries $\ln (1.5)\approx 0.5$\qquad 
\hypertarget {a:11.1.5}{\hyperlink {e:11.1.5}{\bfseries 5.}} \mdseries $\ln (\sqrt {1.5})\approx 0.25$\qquad 
\hypertarget {a:11.1.6}{\hyperlink {e:11.1.6}{\bfseries 6.}} \mdseries $dy=0.22$\qquad 
\hypertarget {a:11.1.7}{\hyperlink {e:11.1.7}{\bfseries 7.}} \mdseries $dy=0.05$\qquad 
\hypertarget {a:11.1.8}{\hyperlink {e:11.1.8}{\bfseries 8.}} \mdseries $dy=0.1$\qquad 
\hypertarget {a:11.1.9}{\hyperlink {e:11.1.9}{\bfseries 9.}} \mdseries $dy=\pi /50$\qquad 
\hypertarget {a:11.1.10}{\hyperlink {e:11.1.10}{\bfseries 10.}} \mdseries $dV=8\pi /25 \text {m}^3$\qquad 
\subsection *{Answers for 12.1}
\hypertarget {a:12.1.1}{\hyperlink {e:12.1.1}{\bfseries 1.}} \mdseries $5x+C$\qquad 
\hypertarget {a:12.1.2}{\hyperlink {e:12.1.2}{\bfseries 2.}} \mdseries $-7x^5/5 +8x + C$\qquad 
\hypertarget {a:12.1.3}{\hyperlink {e:12.1.3}{\bfseries 3.}} \mdseries $2e^x -4x + C$\qquad 
\hypertarget {a:12.1.4}{\hyperlink {e:12.1.4}{\bfseries 4.}} \mdseries $7^x/\ln (7) - x^8/8 +C$\qquad 
\hypertarget {a:12.1.5}{\hyperlink {e:12.1.5}{\bfseries 5.}} \mdseries $15\ln |x| + x^{16}/16 + C$\qquad 
\hypertarget {a:12.1.6}{\hyperlink {e:12.1.6}{\bfseries 6.}} \mdseries $3\cos (x) -\ln |\cos (x)|+C$\qquad 
\hypertarget {a:12.1.7}{\hyperlink {e:12.1.7}{\bfseries 7.}} \mdseries $\tan (x) + \cot (x) + C$\qquad 
\hypertarget {a:12.1.8}{\hyperlink {e:12.1.8}{\bfseries 8.}} \mdseries $\ln |x| - x^{-1} + 2\sqrt {x} +C$\qquad 
\hypertarget {a:12.1.9}{\hyperlink {e:12.1.9}{\bfseries 9.}} \mdseries $17\arctan (x) + 13\ln |x| +C$\qquad 
\hypertarget {a:12.1.10}{\hyperlink {e:12.1.10}{\bfseries 10.}} \mdseries $-\csc (x)/4 - 4\arcsin (x) + C$\qquad 
\hypertarget {a:12.1.11}{\hyperlink {e:12.1.11}{\bfseries 11.}} \mdseries $(x^2+4)^6/6 + C$\qquad 
\hypertarget {a:12.1.12}{\hyperlink {e:12.1.12}{\bfseries 12.}} \mdseries $(\ln (x))^5/5 +C$\qquad 
\hypertarget {a:12.1.13}{\hyperlink {e:12.1.13}{\bfseries 13.}} \mdseries $\sqrt {2x+1} + C$\qquad 
\hypertarget {a:12.1.14}{\hyperlink {e:12.1.14}{\bfseries 14.}} \mdseries $\sqrt {x^2+1} + C$\qquad 
\hypertarget {a:12.1.15}{\hyperlink {e:12.1.15}{\bfseries 15.}} \mdseries $-(4-x^2)^{3/2}/3 +C$\qquad 
\hypertarget {a:12.1.16}{\hyperlink {e:12.1.16}{\bfseries 16.}} \mdseries $2(\ln (x))^{3/2}/3 +C$\qquad 
\hypertarget {a:12.1.17}{\hyperlink {e:12.1.17}{\bfseries 17.}} \mdseries $e^{x^3-1} + C$\qquad 
\hypertarget {a:12.1.18}{\hyperlink {e:12.1.18}{\bfseries 18.}} \mdseries $e^{3(x^2)}/6+C$\qquad 
\hypertarget {a:12.1.19}{\hyperlink {e:12.1.19}{\bfseries 19.}} \mdseries $-e^{-(x^2)} + C$\qquad 
\hypertarget {a:12.1.20}{\hyperlink {e:12.1.20}{\bfseries 20.}} \mdseries $-4e^{-(x^2)} +C$\qquad 
\hypertarget {a:12.1.21}{\hyperlink {e:12.1.21}{\bfseries 21.}} \mdseries $xe^{5x}/5 - e^{5x}/25 + C$\qquad 
\hypertarget {a:12.1.22}{\hyperlink {e:12.1.22}{\bfseries 22.}} \mdseries $-4e^{-x/2} - 2xe^{-x/2} +C$\qquad 
\hypertarget {a:12.1.23}{\hyperlink {e:12.1.23}{\bfseries 23.}} \mdseries $\ln (2x)/2 + C$\qquad 
\hypertarget {a:12.1.24}{\hyperlink {e:12.1.24}{\bfseries 24.}} \mdseries $\ln (x^5+1)/5 +C$\qquad 
\hypertarget {a:12.1.25}{\hyperlink {e:12.1.25}{\bfseries 25.}} \mdseries $-\ln (3-x^3)/3+C$\qquad 
\hypertarget {a:12.1.26}{\hyperlink {e:12.1.26}{\bfseries 26.}} \mdseries $\ln (\ln (x))+C$\qquad 
\hypertarget {a:12.1.27}{\hyperlink {e:12.1.27}{\bfseries 27.}} \mdseries $\ln (e^{2x}+e^{-2x})/2 +C$\qquad 
\hypertarget {a:12.1.28}{\hyperlink {e:12.1.28}{\bfseries 28.}} \mdseries $\ln (\ln (x^2))/2 + C$\qquad 
\hypertarget {a:12.1.29}{\hyperlink {e:12.1.29}{\bfseries 29.}} \mdseries $-\cos (x^5+3) +C$\qquad 
\hypertarget {a:12.1.30}{\hyperlink {e:12.1.30}{\bfseries 30.}} \mdseries $-\sin (-2x^2)/4 +C$\qquad 
\hypertarget {a:12.1.31}{\hyperlink {e:12.1.31}{\bfseries 31.}} \mdseries $-\cos (5x^2)/10 +C$\qquad 
\hypertarget {a:12.1.32}{\hyperlink {e:12.1.32}{\bfseries 32.}} \mdseries $4\sin (x^2)+C$\qquad 
\hypertarget {a:12.1.33}{\hyperlink {e:12.1.33}{\bfseries 33.}} \mdseries $-2\cos (e^{3x})+C$\qquad 
\hypertarget {a:12.1.34}{\hyperlink {e:12.1.34}{\bfseries 34.}} \mdseries $\sin (\ln (x)) + C$\qquad 
\subsection *{Answers for 12.2}
\hypertarget {a:12.2.1}{\hyperlink {e:12.2.1}{\bfseries 1.}} \mdseries $c=1/2$\qquad 
\hypertarget {a:12.2.2}{\hyperlink {e:12.2.2}{\bfseries 2.}} \mdseries $c=\sqrt {18}-2$\qquad 
\hypertarget {a:12.2.3}{\hyperlink {e:12.2.3}{\bfseries 3.}} \mdseries $c=\sqrt {65}-7$\qquad 
\hypertarget {a:12.2.4}{\hyperlink {e:12.2.4}{\bfseries 4.}} \mdseries $f(x)$ is not continuous on $[\pi ,2\pi ]$\qquad 
\hypertarget {a:12.2.5}{\hyperlink {e:12.2.5}{\bfseries 5.}} \mdseries $f(x)$ is not continuous on $[1,4]$\qquad 
\hypertarget {a:12.2.6}{\hyperlink {e:12.2.6}{\bfseries 6.}} \mdseries $x^3/3+47x^2/2-5x+k$\qquad 
\hypertarget {a:12.2.7}{\hyperlink {e:12.2.7}{\bfseries 7.}} \mdseries $\arctan (x) + k$\qquad 
\hypertarget {a:12.2.8}{\hyperlink {e:12.2.8}{\bfseries 8.}} \mdseries $x^4/4 -\ln (x) +k$\qquad 
\hypertarget {a:12.2.9}{\hyperlink {e:12.2.9}{\bfseries 9.}} \mdseries $-\cos (2x)/2 +k$\qquad 
\hypertarget {a:12.2.10}{\hyperlink {e:12.2.10}{\bfseries 10.}} \mdseries Seeking a contradiction, suppose that we have 3 real roots, call them $a$, $b$, and $c$. By Rolle's Theorem, $24x^3-7$ must have a root on both $(a,b)$ and $(b,c)$, but this is impossible as $24x^3-7$ has only one real root.\qquad 
\hypertarget {a:12.2.11}{\hyperlink {e:12.2.11}{\bfseries 11.}} \mdseries Seeking a contradiction, suppose that we have 2 real roots, call them $a$, $b$. By Rolle's Theorem, $f'(x)$ must have a root on $(a,b)$, but this is impossible.\qquad 
\subsection *{Answers for 12.3}
\hypertarget {a:12.3.1}{\hyperlink {e:12.3.1}{\bfseries 1.}} \mdseries $-2.25$ m/s\qquad 
\hypertarget {a:12.3.2}{\hyperlink {e:12.3.2}{\bfseries 2.}} \mdseries $\approx 2.57$ s\qquad 
\hypertarget {a:12.3.3}{\hyperlink {e:12.3.3}{\bfseries 3.}} \mdseries $531$ minutes\qquad 
\hypertarget {a:12.3.4}{\hyperlink {e:12.3.4}{\bfseries 4.}} \mdseries it takes $104.4$ minutes for the population to double each time\qquad 
\hypertarget {a:12.3.5}{\hyperlink {e:12.3.5}{\bfseries 5.}} \mdseries $\approx 7.07$ g\qquad 
\hypertarget {a:12.3.6}{\hyperlink {e:12.3.6}{\bfseries 6.}} \mdseries $\approx 400$ days\qquad 
\hypertarget {a:12.3.7}{\hyperlink {e:12.3.7}{\bfseries 7.}} \mdseries $1.96$\qquad 
\hypertarget {a:12.3.8}{\hyperlink {e:12.3.8}{\bfseries 8.}} \mdseries $7.14$\qquad 
\hypertarget {a:12.3.9}{\hyperlink {e:12.3.9}{\bfseries 9.}} \mdseries $3$\qquad 
\hypertarget {a:12.3.10}{\hyperlink {e:12.3.10}{\bfseries 10.}} \mdseries $3$\qquad 
\subsection *{Answers for 13.1}
\hypertarget {a:13.1.1}{\hyperlink {e:13.1.1}{\bfseries 1.}} \mdseries positive\qquad 
\hypertarget {a:13.1.2}{\hyperlink {e:13.1.2}{\bfseries 2.}} \mdseries negative\qquad 
\hypertarget {a:13.1.3}{\hyperlink {e:13.1.3}{\bfseries 3.}} \mdseries zero\qquad 
\hypertarget {a:13.1.4}{\hyperlink {e:13.1.4}{\bfseries 4.}} \mdseries positive\qquad 
\hypertarget {a:13.1.5}{\hyperlink {e:13.1.5}{\bfseries 5.}} \mdseries $1$\qquad 
\hypertarget {a:13.1.6}{\hyperlink {e:13.1.6}{\bfseries 6.}} \mdseries $2$\qquad 
\hypertarget {a:13.1.7}{\hyperlink {e:13.1.7}{\bfseries 7.}} \mdseries $1$\qquad 
\hypertarget {a:13.1.8}{\hyperlink {e:13.1.8}{\bfseries 8.}} \mdseries $1/2$\qquad 
\hypertarget {a:13.1.9}{\hyperlink {e:13.1.9}{\bfseries 9.}} \mdseries $4e - \frac {4}{e} - 2$\qquad 
\hypertarget {a:13.1.10}{\hyperlink {e:13.1.10}{\bfseries 10.}} \mdseries $14 -2 \ln (4)$\qquad 
\hypertarget {a:13.1.11}{\hyperlink {e:13.1.11}{\bfseries 11.}} \mdseries $(-\pi ,\pi )$\qquad 
\hypertarget {a:13.1.12}{\hyperlink {e:13.1.12}{\bfseries 12.}} \mdseries $(-2\pi ,-\pi )\cup (\pi ,2\pi )$\qquad 
\subsection *{Answers for 13.2}
\hypertarget {a:13.2.1}{\hyperlink {e:13.2.1}{\bfseries 1.}} \mdseries $10.2$\qquad 
\hypertarget {a:13.2.2}{\hyperlink {e:13.2.2}{\bfseries 2.}} \mdseries $19.66$\qquad 
\hypertarget {a:13.2.3}{\hyperlink {e:13.2.3}{\bfseries 3.}} \mdseries $0.24$\qquad 
\hypertarget {a:13.2.4}{\hyperlink {e:13.2.4}{\bfseries 4.}} \mdseries $0.08$\qquad 
\hypertarget {a:13.2.5}{\hyperlink {e:13.2.5}{\bfseries 5.}} \mdseries $\sum _{i=0}^{n-1} \left ( 4 - (1+2i/n)^2\right )\cdot \frac {2}{n}$\qquad 
\hypertarget {a:13.2.6}{\hyperlink {e:13.2.6}{\bfseries 6.}} \mdseries $\sum _{i=1}^{n} \left (\frac {\sin \left (-\pi + \frac {2\pi i}{n}\right )}{-\pi + \frac {2\pi i}{n}}\right )\cdot \frac {2\pi }{n}$\qquad 
\hypertarget {a:13.2.7}{\hyperlink {e:13.2.7}{\bfseries 7.}} \mdseries $\sum _{i=0}^{n-1} e^{((1+2i)/2n)^2} \cdot \frac {1}{n}$\qquad 
\hypertarget {a:13.2.8}{\hyperlink {e:13.2.8}{\bfseries 8.}} \mdseries $3/2$\qquad 
\hypertarget {a:13.2.9}{\hyperlink {e:13.2.9}{\bfseries 9.}} \mdseries $12$\qquad 
\hypertarget {a:13.2.10}{\hyperlink {e:13.2.10}{\bfseries 10.}} \mdseries $56$\qquad 
\subsection *{Answers for 14.1}
\hypertarget {a:14.1.1}{\hyperlink {e:14.1.1}{\bfseries 1.}} \mdseries $87/2$\qquad 
\hypertarget {a:14.1.2}{\hyperlink {e:14.1.2}{\bfseries 2.}} \mdseries $2$\qquad 
\hypertarget {a:14.1.3}{\hyperlink {e:14.1.3}{\bfseries 3.}} \mdseries $\ln (10)$\qquad 
\hypertarget {a:14.1.4}{\hyperlink {e:14.1.4}{\bfseries 4.}} \mdseries $e^5-1$\qquad 
\hypertarget {a:14.1.5}{\hyperlink {e:14.1.5}{\bfseries 5.}} \mdseries $3^4/4$\qquad 
\hypertarget {a:14.1.6}{\hyperlink {e:14.1.6}{\bfseries 6.}} \mdseries $2^6/6 -1/6$\qquad 
\hypertarget {a:14.1.7}{\hyperlink {e:14.1.7}{\bfseries 7.}} \mdseries $416/3$\qquad 
\hypertarget {a:14.1.8}{\hyperlink {e:14.1.8}{\bfseries 8.}} \mdseries $8$\qquad 
\hypertarget {a:14.1.9}{\hyperlink {e:14.1.9}{\bfseries 9.}} \mdseries $-7\ln (2)$\qquad 
\hypertarget {a:14.1.10}{\hyperlink {e:14.1.10}{\bfseries 10.}} \mdseries $965/3$\qquad 
\hypertarget {a:14.1.11}{\hyperlink {e:14.1.11}{\bfseries 11.}} \mdseries $2189/3$\qquad 
\hypertarget {a:14.1.12}{\hyperlink {e:14.1.12}{\bfseries 12.}} \mdseries $4356 \sqrt {3}/5$\qquad 
\hypertarget {a:14.1.13}{\hyperlink {e:14.1.13}{\bfseries 13.}} \mdseries $2/3$\qquad 
\hypertarget {a:14.1.14}{\hyperlink {e:14.1.14}{\bfseries 14.}} \mdseries $35$\qquad 
\hypertarget {a:14.1.15}{\hyperlink {e:14.1.15}{\bfseries 15.}} \mdseries $x^2-3x$\qquad 
\hypertarget {a:14.1.16}{\hyperlink {e:14.1.16}{\bfseries 16.}} \mdseries $2x(x^4-3x^2)$\qquad 
\hypertarget {a:14.1.17}{\hyperlink {e:14.1.17}{\bfseries 17.}} \mdseries $e^{\left (x^2\right )}$\qquad 
\hypertarget {a:14.1.18}{\hyperlink {e:14.1.18}{\bfseries 18.}} \mdseries $2xe^{\left (x^4\right )}$\qquad 
\hypertarget {a:14.1.19}{\hyperlink {e:14.1.19}{\bfseries 19.}} \mdseries $\tan (x^2)$\qquad 
\hypertarget {a:14.1.20}{\hyperlink {e:14.1.20}{\bfseries 20.}} \mdseries $2x\tan (x^4)$\qquad 
\subsection *{Answers for 14.2}
\hypertarget {a:14.2.1}{\hyperlink {e:14.2.1}{\bfseries 1.}} \mdseries $8\sqrt 2/15$\qquad 
\hypertarget {a:14.2.2}{\hyperlink {e:14.2.2}{\bfseries 2.}} \mdseries $1/12$\qquad 
\hypertarget {a:14.2.3}{\hyperlink {e:14.2.3}{\bfseries 3.}} \mdseries $9/2$\qquad 
\hypertarget {a:14.2.4}{\hyperlink {e:14.2.4}{\bfseries 4.}} \mdseries $4/3$\qquad 
\hypertarget {a:14.2.5}{\hyperlink {e:14.2.5}{\bfseries 5.}} \mdseries $2/3-2/\pi $\qquad 
\hypertarget {a:14.2.6}{\hyperlink {e:14.2.6}{\bfseries 6.}} \mdseries $3/\pi - 3\sqrt 3/(2\pi )-1/8$\qquad 
\hypertarget {a:14.2.7}{\hyperlink {e:14.2.7}{\bfseries 7.}} \mdseries $1/3$\qquad 
\hypertarget {a:14.2.8}{\hyperlink {e:14.2.8}{\bfseries 8.}} \mdseries $10\sqrt {5}/3-6$\qquad 
\hypertarget {a:14.2.9}{\hyperlink {e:14.2.9}{\bfseries 9.}} \mdseries $500/3$\qquad 
\hypertarget {a:14.2.10}{\hyperlink {e:14.2.10}{\bfseries 10.}} \mdseries $2$\qquad 
\hypertarget {a:14.2.11}{\hyperlink {e:14.2.11}{\bfseries 11.}} \mdseries $1/5$\qquad 
\hypertarget {a:14.2.12}{\hyperlink {e:14.2.12}{\bfseries 12.}} \mdseries $1/6$\qquad 
\subsection *{Answers for 15.1}
\hypertarget {a:15.1.1}{\hyperlink {e:15.1.1}{\bfseries 1.}} \mdseries $-(1-t)^{10}/10+C$\qquad 
\hypertarget {a:15.1.2}{\hyperlink {e:15.1.2}{\bfseries 2.}} \mdseries $x^5/5+2x^3/3+x+C$\qquad 
\hypertarget {a:15.1.3}{\hyperlink {e:15.1.3}{\bfseries 3.}} \mdseries $(x^2+1)^{101}/202+C$\qquad 
\hypertarget {a:15.1.4}{\hyperlink {e:15.1.4}{\bfseries 4.}} \mdseries $-3(1-5t)^{2/3}/10+C$\qquad 
\hypertarget {a:15.1.5}{\hyperlink {e:15.1.5}{\bfseries 5.}} \mdseries $(\sin ^4x)/4+C$\qquad 
\hypertarget {a:15.1.6}{\hyperlink {e:15.1.6}{\bfseries 6.}} \mdseries $-(100-x^2)^{3/2}/3+C$\qquad 
\hypertarget {a:15.1.7}{\hyperlink {e:15.1.7}{\bfseries 7.}} \mdseries $-2\sqrt {1-x^3}/3+C$\qquad 
\hypertarget {a:15.1.8}{\hyperlink {e:15.1.8}{\bfseries 8.}} \mdseries $\sin (\sin \pi t)/\pi +C$\qquad 
\hypertarget {a:15.1.9}{\hyperlink {e:15.1.9}{\bfseries 9.}} \mdseries $1/(2\cos ^2 x)=(1/2)\sec ^2x+C$\qquad 
\hypertarget {a:15.1.10}{\hyperlink {e:15.1.10}{\bfseries 10.}} \mdseries $-\ln |\cos x|+C$\qquad 
\hypertarget {a:15.1.11}{\hyperlink {e:15.1.11}{\bfseries 11.}} \mdseries $0$\qquad 
\hypertarget {a:15.1.12}{\hyperlink {e:15.1.12}{\bfseries 12.}} \mdseries $\tan ^2(x)/2+C$\qquad 
\hypertarget {a:15.1.13}{\hyperlink {e:15.1.13}{\bfseries 13.}} \mdseries $1/4$\qquad 
\hypertarget {a:15.1.14}{\hyperlink {e:15.1.14}{\bfseries 14.}} \mdseries $-\cos (\tan x)+C$\qquad 
\hypertarget {a:15.1.15}{\hyperlink {e:15.1.15}{\bfseries 15.}} \mdseries $1/10$\qquad 
\hypertarget {a:15.1.16}{\hyperlink {e:15.1.16}{\bfseries 16.}} \mdseries $\sqrt 3/4$\qquad 
\hypertarget {a:15.1.17}{\hyperlink {e:15.1.17}{\bfseries 17.}} \mdseries $(27/8)(x^2-7)^{8/9}$\qquad 
\hypertarget {a:15.1.18}{\hyperlink {e:15.1.18}{\bfseries 18.}} \mdseries $-(3^7+1)/14$\qquad 
\hypertarget {a:15.1.19}{\hyperlink {e:15.1.19}{\bfseries 19.}} \mdseries $0$\qquad 
\hypertarget {a:15.1.20}{\hyperlink {e:15.1.20}{\bfseries 20.}} \mdseries $f(x)^2/2$\qquad 
\subsection *{Answers for 15.2}
\hypertarget {a:15.2.1}{\hyperlink {e:15.2.1}{\bfseries 1.}} \mdseries $x/2-\sin (2x)/4+C$\qquad 
\hypertarget {a:15.2.2}{\hyperlink {e:15.2.2}{\bfseries 2.}} \mdseries $-\cos x+(\cos ^3x)/3+C$\qquad 
\hypertarget {a:15.2.3}{\hyperlink {e:15.2.3}{\bfseries 3.}} \mdseries $3x/8-(\sin 2x)/4+(\sin 4x)/32+C$\qquad 
\hypertarget {a:15.2.4}{\hyperlink {e:15.2.4}{\bfseries 4.}} \mdseries $(\cos ^5 x)/5-(\cos ^3x)/3+C$\qquad 
\hypertarget {a:15.2.5}{\hyperlink {e:15.2.5}{\bfseries 5.}} \mdseries $\sin x-(\sin ^3x)/3+C$\qquad 
\hypertarget {a:15.2.6}{\hyperlink {e:15.2.6}{\bfseries 6.}} \mdseries $x/8-(\sin 4x)/32+C$\qquad 
\hypertarget {a:15.2.7}{\hyperlink {e:15.2.7}{\bfseries 7.}} \mdseries $(\sin ^3x)/3-(\sin ^5x)/5+C$\qquad 
\hypertarget {a:15.2.8}{\hyperlink {e:15.2.8}{\bfseries 8.}} \mdseries $-2(\cos x)^{5/2}/5+C$\qquad 
\hypertarget {a:15.2.9}{\hyperlink {e:15.2.9}{\bfseries 9.}} \mdseries $\tan x-\cot x+C$\qquad 
\hypertarget {a:15.2.10}{\hyperlink {e:15.2.10}{\bfseries 10.}} \mdseries $(\sec ^3x)/3-\sec x+C$\qquad 
\subsection *{Answers for 15.3}
\hypertarget {a:15.3.1}{\hyperlink {e:15.3.1}{\bfseries 1.}} \mdseries $\cos x+x\sin x+C$\qquad 
\hypertarget {a:15.3.2}{\hyperlink {e:15.3.2}{\bfseries 2.}} \mdseries $x^2\sin x-2 \sin x+2x\cos x +C$\qquad 
\hypertarget {a:15.3.3}{\hyperlink {e:15.3.3}{\bfseries 3.}} \mdseries $(x-1)e^x +C$\qquad 
\hypertarget {a:15.3.4}{\hyperlink {e:15.3.4}{\bfseries 4.}} \mdseries $(1/2)e^{x^2} +C$\qquad 
\hypertarget {a:15.3.5}{\hyperlink {e:15.3.5}{\bfseries 5.}} \mdseries $(x/2)-\sin (2x)/4 +C$\qquad 
\hypertarget {a:15.3.6}{\hyperlink {e:15.3.6}{\bfseries 6.}} \mdseries $x\ln x-x +C$\qquad 
\hypertarget {a:15.3.7}{\hyperlink {e:15.3.7}{\bfseries 7.}} \mdseries $(x^2\arctan x +\arctan x -x)/2+C$\qquad 
\hypertarget {a:15.3.8}{\hyperlink {e:15.3.8}{\bfseries 8.}} \mdseries $-x^3\cos x+3x^2\sin x+6x\cos x-6\sin x+C$\qquad 
\hypertarget {a:15.3.9}{\hyperlink {e:15.3.9}{\bfseries 9.}} \mdseries $x^3\sin x+3x^2\cos x-6x\sin x-6\cos x+C$\qquad 
\hypertarget {a:15.3.10}{\hyperlink {e:15.3.10}{\bfseries 10.}} \mdseries $x^2/4-(\cos ^2 x)/4-(x\sin x\cos x)/2+C$\qquad 
\hypertarget {a:15.3.11}{\hyperlink {e:15.3.11}{\bfseries 11.}} \mdseries $x/4-(x\cos ^2 x)/2+(\cos x\sin x)/4+C$\qquad 
\hypertarget {a:15.3.12}{\hyperlink {e:15.3.12}{\bfseries 12.}} \mdseries $x\arctan (\sqrt x)+\arctan (\sqrt x)-\sqrt {x}+C$\qquad 
\hypertarget {a:15.3.13}{\hyperlink {e:15.3.13}{\bfseries 13.}} \mdseries $2\sin (\sqrt x)-2\sqrt x\cos (\sqrt x)+C$\qquad 
\hypertarget {a:15.3.14}{\hyperlink {e:15.3.14}{\bfseries 14.}} \mdseries $\sec x\csc x-2\cot x+C$\qquad 
\subsection *{Answers for 16.1}
\hypertarget {a:16.1.1}{\hyperlink {e:16.1.1}{\bfseries 1.}} \mdseries $8\pi /3$\qquad 
\hypertarget {a:16.1.2}{\hyperlink {e:16.1.2}{\bfseries 2.}} \mdseries $\pi /30$\qquad 
\hypertarget {a:16.1.3}{\hyperlink {e:16.1.3}{\bfseries 3.}} \mdseries $\pi (\pi /2-1)$\qquad 
\hypertarget {a:16.1.4}{\hyperlink {e:16.1.4}{\bfseries 4.}} \mdseries (a) $114\pi /5$ (b) $74\pi /5$ (c) $20\pi $\hfill \break (d) $4\pi $\qquad 
\hypertarget {a:16.1.5}{\hyperlink {e:16.1.5}{\bfseries 5.}} \mdseries $16\pi $, $24\pi $\qquad 
\hypertarget {a:16.1.6}{\hyperlink {e:16.1.6}{\bfseries 6.}} \mdseries $4\pi r^3/3$\qquad 
\hypertarget {a:16.1.7}{\hyperlink {e:16.1.7}{\bfseries 7.}} \mdseries $\pi h^2(3r-h)/3$\qquad 
\hypertarget {a:16.1.8}{\hyperlink {e:16.1.8}{\bfseries 8.}} \mdseries $(1/3)(\hbox {area of base})(\hbox {height})$\qquad 
\hypertarget {a:16.1.9}{\hyperlink {e:16.1.9}{\bfseries 9.}} \mdseries $2\pi $\qquad 
\subsection *{Answers for 16.2}
\hypertarget {a:16.2.1}{\hyperlink {e:16.2.1}{\bfseries 1.}} \mdseries $(22\sqrt {22}-8)/27$\qquad 
\hypertarget {a:16.2.2}{\hyperlink {e:16.2.2}{\bfseries 2.}} \mdseries $\ln (2)+3/8$\qquad 
\hypertarget {a:16.2.3}{\hyperlink {e:16.2.3}{\bfseries 3.}} \mdseries $a+a^3/3$\qquad 
\hypertarget {a:16.2.4}{\hyperlink {e:16.2.4}{\bfseries 4.}} \mdseries $\ln ((\sqrt 2+1)/\sqrt 3)$\qquad 
\hypertarget {a:16.2.5}{\hyperlink {e:16.2.5}{\bfseries 5.}} \mdseries $\approx 3.82$\qquad 
\hypertarget {a:16.2.6}{\hyperlink {e:16.2.6}{\bfseries 6.}} \mdseries $\approx 1.01$\qquad 
\hypertarget {a:16.2.7}{\hyperlink {e:16.2.7}{\bfseries 7.}} \mdseries $\sqrt {1+e^2}-\sqrt 2+ {1\over 2}\ln \left ({\sqrt {1+e^2}-1\over \sqrt {1+e^2}+1}\right )+ {1\over 2}\ln (3+2\sqrt 2)$\qquad 
}
\normalsize
\backmatter
%\addcontentsline{toc}{chapter}{Index}
\printindex


\end{document}


