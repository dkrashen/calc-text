\chapter{Differential Equations}

Many physical phenomena can be modeled using the
language of calculus. For example, observational evidence suggests
that the temperature of a cup of tea (or some other liquid) in a
room of constant temperature will cool over time at a rate
proportional to the difference between the room temperature  and the
temperature of the tea.

In symbols, if $t$ is the time, $M$ is the room temperature,
and $f(t)$ is the temperature of the tea at time $t$ then $f'(t) =
k(M-f(t))$ where $k>0$ is a constant which will depend on the kind of tea
(or more generally the kind of liquid) but not on the room temperature
or the temperature of the tea. This is {\dfont Newton's law of
  cooling\index{Newton's law of cooling}\/} and the equation that we
just wrote down is an example of a {\dfont differential
  equation\index{differential equation}}.  Ideally we would like to
solve this equation, namely, find the function $f(t)$ that describes
the temperature over time, though this often turns out to be
impossible, in which case various approximation techniques must be
used.  The use and solution of differential equations is an important
field of mathematics; here we see how to solve some simple but useful
types of differential equation.

Informally, a differential equation is an equation in which one or
more of the derivatives of some function appear. Typically, a
scientific theory will produce a differential equation (or a system of
differential equations) that describes or governs some physical
process, but the theory will not produce the desired function or
functions directly. 

Recall from section~\xrefn{sec:related rates} that when the variable
is time the derivative of a function $y(t)$ is sometimes written as 
$\dot y$ instead of $y'$; this is quite common in the study of
differential equations.

\input section17.01
\input section17.02
\input section17.03
\input section17.04
\input section17.05
\input section17.06
\input section17.07
%\input section17.08
%\input section17.09

