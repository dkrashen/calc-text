\section{Sequences}{}{}
\nobreak
While the idea of a sequence of numbers, $a_1,a_2,a_3,\ldots$ is
straightforward, it is useful to think of a sequence as a function. We
have up until now dealt with functions whose domains are the real
numbers, or a subset of the real numbers, like $f(x)=\sin x$. A
sequence is a function with domain the natural numbers
$\N=\{1,2,3,\ldots\}$ or the non-negative integers,
$\ds \Z^{\ge0}=\{0,1,2,3,\ldots\}$. The range of the function is still
allowed to be the real numbers; in symbols, we say that a sequence is
a function $f\colon \N\to\R$. Sequences are written in a few different
ways, all equivalent; these all mean the same thing:
$$
  \displaylines{a_1,a_2,a_3,\ldots \\
  \left\{a_n\right\}_{n=1}^\infty \\
  \left\{f(n)\right\}_{n=1}^\infty \\}
%%  \left\{a_n\mid n\in\N\right\} \\
%%  \left\{f(n)\mid n\in\N\right\} \\
$$


As with functions on the real numbers,
we will most often encounter sequences that can be expressed by a
formula. We have already seen the sequence $\ds a_i=f(i)=1-1/2^i$, and others
are easy to come by:
$$\eqalign{
  f(i)&={i\over i+1} \\
  f(n)&={1\over2^n} \\
  f(n)&=\sin(n\pi/6) \\
  f(i)&={(i-1)(i+2)\over2^i} \\
}$$
Frequently these formulas will make sense if thought of either as
functions with domain $\R$ or $\N$, though occasionally one will make
sense only for integer values. 

Faced with a sequence we are interested in the limit
$$\lim_{i\to \infty} f(i) = \lim_{i\to\infty} a_i.$$
We already understand
$$\lim_{x\to\infty} f(x)$$
when $x$ is a real valued variable; now we simply want to restrict the
``input'' values to be integers. No real difference is required in the
definition of limit, except that we specify, perhaps implicitly, that
the variable is an integer.  Compare this definition to
definition~\xrefn{defn:limit at infinity}.

\begin{definition} \relax\index{limit of a sequence}
Suppose that $\ds\left\{a_n\right\}_{n=1}^\infty$ is a sequence.
We say that $\ds \lim_{n\to \infty}a_n=L$ if for every $\epsilon>0$
there is an $N > 0$ so that whenever $n>N$, $|a_n-L|<\epsilon$. If
$\ds \lim_{n\to\infty}a_n=L$ we say that the sequence {\dfont
converges\index{convergent sequence}\index{sequence!convergent}\/},
otherwise it {\dfont diverges\index{divergent
sequence}\index{sequence!divergent}\/}.  
\end{definition} 

If $f(i)$ defines a sequence, and $f(x)$ makes sense, and 
$\ds \lim_{x\to\infty}f(x)=L$, then it is clear that
$\ds \lim_{i\to\infty}f(i)=L$ as well, but it is important to note that
the converse of this statement is not true. For example, since
$\ds \lim_{x\to\infty}(1/x)=0$, it is clear that also
$\ds \lim_{i\to\infty}(1/i)=0$, that is, the numbers
$${1\over1},{1\over2},{1\over3},{1\over4},{1\over5},{1\over6},\ldots$$
get closer and closer to 0. Consider this, however: Let 
$f(n)=\sin(n\pi)$. This is the sequence
$$
  \sin(0\pi), \sin(1\pi),\sin(2\pi),\sin(3\pi),\ldots=0,0,0,0,\ldots
$$
since $\sin(n\pi)=0$ when $n$ is an integer. Thus
$\ds \lim_{n\to\infty}f(n)=0$. But $\ds \lim_{x\to\infty}f(x)$, when $x$ is
real, does not exist: as $x$ gets bigger and bigger, the values
$\sin(x\pi)$ do not get closer and closer to a single value, but take
on all values between $-1$ and $1$ over and over. In general, whenever
you want to know $\ds \lim_{n\to\infty}f(n)$ you should first attempt to
compute $\ds \lim_{x\to\infty}f(x)$, since if the latter exists it is also
equal to the first limit. But if for some reason
$\ds \lim_{x\to\infty}f(x)$ does not exist, it may still be true that 
$\ds \lim_{n\to\infty}f(n)$ exists, but you'll have to figure out
another way to compute it.

It is occasionally useful to think of the graph of a sequence. Since
the function is defined only for integer values, the graph is just a
sequence of dots. In figure~\xrefn{fig:graphs of sequences} we see the
graphs of two sequences and the graphs of the corresponding real
functions.

\figure
\vbox{\beginpicture
\normalgraphs
\ninepoint
\setcoordinatesystem units <0.5truecm,0.5truecm> point at -14 0
\setplotarea x from 0 to 10, y from 0 to 5
\axis left ticks numbered from 0 to 5 by 1 /
\axis bottom  ticks numbered from 0 to 10 by 5 /
\setquadratic
\plot 0.200 5.000 0.445 2.247 0.690 1.449 0.935 1.070 1.180 0.847 
1.425 0.702 1.670 0.599 1.915 0.522 2.160 0.463 2.405 0.416 
2.650 0.377 2.895 0.345 3.140 0.318 3.385 0.295 3.630 0.275 
3.875 0.258 4.120 0.243 4.365 0.229 4.610 0.217 4.855 0.206 
5.100 0.196 5.345 0.187 5.590 0.179 5.835 0.171 6.080 0.164 
6.325 0.158 6.570 0.152 6.815 0.147 7.060 0.142 7.305 0.137 
7.550 0.132 7.795 0.128 8.040 0.124 8.285 0.121 8.530 0.117 
8.775 0.114 9.020 0.111 9.265 0.108 9.510 0.105 9.755 0.103 
10.000 0.100 /
\put {$f(x)=1/x$} at 5 2.5
\setcoordinatesystem units <0.5truecm,0.5truecm> point at 0 0
\setplotarea x from 0 to 10, y from 0 to 5
\axis left ticks numbered from 0 to 5 by 1 /
\axis bottom  ticks numbered from 0 to 10 by 5 /
\multiput {\fivepoint$\bullet$} at 1 1 2 0.5 3 0.3333 4 0.25 5 0.2
6 0.1667 7 0.1429 8 0.125 9 0.1111 10 0.1 /
\put {$f(n)=1/n$} at 5 2.5
\setcoordinatesystem units <0.63truecm,1truecm> point at -11.1 3
\setplotarea x from 0 to 8, y from -1 to 1
\axis left ticks numbered from -1 to 1 by 1 /
\axis bottom shiftedto y=0 /
\setquadratic
\plot 0.000 0.000 0.080 0.249 0.160 0.482 0.240 0.685 0.320 0.844 
0.400 0.951 0.480 0.998 0.560 0.982 0.640 0.905 0.720 0.771 
0.800 0.588 0.880 0.368 0.960 0.125 1.040 -0.125 1.120 -0.368 
1.200 -0.588 1.280 -0.771 1.360 -0.905 1.440 -0.982 1.520 -0.998 
1.600 -0.951 1.680 -0.844 1.760 -0.685 1.840 -0.482 1.920 -0.249 
2.000 0.000 2.080 0.249 2.160 0.482 2.240 0.685 2.320 0.844 
2.400 0.951 2.480 0.998 2.560 0.982 2.640 0.905 2.720 0.771 
2.800 0.588 2.880 0.368 2.960 0.125 3.040 -0.125 3.120 -0.368 
3.200 -0.588 3.280 -0.771 3.360 -0.905 3.440 -0.982 3.520 -0.998 
3.600 -0.951 3.680 -0.844 3.760 -0.685 3.840 -0.482 3.920 -0.249 
4.000 0.000 4.080 0.249 4.160 0.482 4.240 0.685 4.320 0.844 
4.400 0.951 4.480 0.998 4.560 0.982 4.640 0.905 4.720 0.771 
4.800 0.588 4.880 0.368 4.960 0.125 5.040 -0.125 5.120 -0.368 
5.200 -0.588 5.280 -0.771 5.360 -0.905 5.440 -0.982 5.520 -0.998 
5.600 -0.951 5.680 -0.844 5.760 -0.685 5.840 -0.482 5.920 -0.249 
6.000 0.000 6.080 0.249 6.160 0.482 6.240 0.685 6.320 0.844 
6.400 0.951 6.480 0.998 6.560 0.982 6.640 0.905 6.720 0.771 
6.800 0.588 6.880 0.368 6.960 0.125 7.040 -0.125 7.120 -0.368 
7.200 -0.588 7.280 -0.771 7.360 -0.905 7.440 -0.982 7.520 -0.998 
7.600 -0.951 7.680 -0.844 7.760 -0.685 7.840 -0.482 7.920 -0.249 
8.000 0.000 /
\put {$f(x)=\sin(x\pi)$} at 4 1.3
\setcoordinatesystem units <0.63truecm,1truecm> point at 0 3
\setplotarea x from 0 to 8, y from -1 to 1
\axis left ticks numbered from -1 to 1 by 1 /
\axis bottom shiftedto y=0 ticks numbered from 1 to 8 by 1 /
\multiput {\fivepoint$\bullet$} at 0 0 1 0 2 0 3 0 4 0 5 0 6 0 7 0 8 0 /
\put {$f(n)=\sin(n\pi)$} at 4 0.5
\endpicture}
\figrdef{fig:graphs of sequences}
\endfigure{Graphs of sequences and their corresponding real functions.}

Not surprisingly, the properties of limits of real functions translate
into properties of sequences quite easily. 
Theorem~\xrefn{thm:properties of limits} about limits becomes

%\vbox{
\begin{theorem} \relax\label{thm:properties of sequences}
Suppose that $\ds\lim_{n\to\infty}a_n=L$ and 
$\ds\lim_{n\to\infty}b_n=M$ and
$k$ is some constant. Then
$$\eqalign{
&\lim_{n\to\infty} ka_n = k\lim_{n\to\infty}a_n=kL \\
&\lim_{n\to\infty} (a_n+b_n) = \lim_{n\to\infty}a_n+\lim_{n\to\infty}b_n=L+M \\
&\lim_{n\to\infty} (a_n-b_n) = \lim_{n\to\infty}a_n-\lim_{n\to\infty}b_n=L-M \\
&\lim_{n\to\infty} (a_nb_n) = \lim_{n\to\infty}a_n\cdot\lim_{n\to\infty}b_n=LM \\
&\lim_{n\to\infty} {a_n\over b_n} = {\lim_{n\to\infty}a_n\over
  \lim_{n\to\infty}b_n}={L\over M},\hbox{ if $M$ is not 0} \\
}$$
\end{proof}
%}

Likewise the Squeeze Theorem (\xrefn{thm:squeeze theorem}) becomes

\begin{theorem}\relax\label{thm:squeeze theorem for sequences}
Suppose that $\ds a_n \le b_n \le c_n$ for all $n>N$, for some $N$.
If $\ds\lim_{n\to\infty}a_n=\ds\lim_{n\to\infty}c_n=L$, 
then $\ds\lim_{n\to\infty}b_n=L$.
\end{proof}

And a final useful fact:

\begin{theorem} \relax\label{thm:absolute value sequence}
$\ds\lim_{n\to\infty}|a_n|=0$ if and only if
$\ds\lim_{n\to\infty}a_n=0$.
\end{proof}

This says simply that the size of $\ds a_n$ gets close to zero if and
only if $\ds a_n$ gets close to zero.

\begin{example}
Determine whether $\ds\left\{{n\over n+1}\right\}_{n=0}^\infty$ converges or
diverges. If it converges, compute the limit. Since this makes sense
for real numbers we consider
$$
\lim_{x\to\infty}{x\over x+1}=\lim_{x\to\infty}1-{1\over x+1}=1-0=1.
$$
Thus the sequence converges to 1.
\end{example}

\begin{example}
Determine whether $\ds\bigg\{{\ln n\over n}\bigg\}_{n=1}^\infty$ converges or
diverges. If it converges, compute the limit. We compute
$$\lim_{x\to\infty}{\ln x\over x}=\lim_{x\to\infty}{1/x\over 1}=
0,$$
using L'H\^opital's Rule. 
Thus the sequence converges to 0.
\end{example}

\begin{example}
Determine whether $\ds\{(-1)^n\}_{n=0}^\infty$ converges or
diverges. If it converges, compute the limit. This does not make sense
for all real exponents, but the sequence is easy to understand: it is
$$1,-1,1,-1,1\ldots$$
and clearly diverges.
\end{example}
\label{example:alternating ones}

\begin{example}
Determine whether $\ds\{(-1/2)^n\}_{n=0}^\infty$ converges or
diverges. If it converges, compute the limit. We consider the sequence 
$\ds\{|(-1/2)^n|\}_{n=0}^\infty=\{(1/2)^n\}_{n=0}^\infty$.
Then
$$
  \lim_{x\to\infty}\left({1\over2}\right)^x=\lim_{x\to\infty}{1\over2^x}=0,
$$
so by theorem~\xrefn{thm:absolute value sequence} the sequence converges to 0.
\end{example}

\begin{example}
Determine whether $\ds\{(\sin n)/\sqrt{n}\}_{n=1}^\infty$ converges or
diverges. If it converges, compute the limit. 
Since $|\sin n|\le 1$, $\ds 0\le|\sin n/\sqrt{n}|\le
1/\sqrt{n}$ and we can use theorem~\xrefn{thm:squeeze theorem for
sequences} with $\ds a_n=0$ and $\ds c_n=1/\sqrt{n}$. Since
$\ds\lim_{n\to\infty} a_n=\ds\lim_{n\to\infty} c_n=0$, 
$\ds\lim_{n\to\infty}\sin n/\sqrt{n}=0$ and the sequence converges to 0.
\end{example}

\begin{example}
A particularly common and useful sequence is $\ds \{r^n\}_{n=0}^\infty$,
for various values of $r$. Some are quite easy to understand: If $r=1$
the sequence converges to 1 since every term is 1, and likewise if
$r=0$ the sequence converges to 0. If $r=-1$ this is
the sequence of example~\xrefn{example:alternating ones} and
diverges. If $r>1$ or $r<-1$ the terms $\ds r^n$ get large without limit,
so the sequence diverges. If $0<r<1$ then the sequence converges to
0. If $-1<r<0$ then $\ds |r^n|=|r|^n$ and $0<|r|<1$, so the sequence
$\ds \{|r|^n\}_{n=0}^\infty$ converges to 0, so also 
$\ds\{r^n\}_{n=0}^\infty$ converges to 0.
converges. In summary, $\ds \{r^n\}$ converges precisely when
$-1<r\le1$ in which case
$$
  \lim_{n\to\infty} r^n=\cases{
  0& if $-1<r<1$ \\
  1& if $r=1$ \\}
$$
\vskip-10pt\end{example}

Sometimes we will not be able to determine the limit of a sequence,
but we still would like to know whether it converges. In some cases we
can determine this even without being able to compute the limit.

A sequence is called {\dfont increasing\index{sequence!increasing}\/}
or sometimes {\dfont strictly increasing\/} if $\ds a_i<a_{i+1}$ for all
$i$. It is called {\dfont
non-decreasing\index{sequence!non-decreasing}\/} or sometimes
(unfortunately) {\dfont increasing\/} if $\ds a_i\le a_{i+1}$ for all
$i$. Similarly a sequence is {\dfont
decreasing\index{sequence!decreasing}\/} if $\ds a_i>a_{i+1}$ for all $i$
and {\dfont non-increasing\index{sequence!non-increasing}\/} if
$\ds a_i\ge a_{i+1}$ for all $i$. If a sequence has any of these
properties it is called {\dfont monotonic\index{sequence!monotonic}\/}.

\begin{example}
The sequence
$$
  \left\{{2^i-1\over2^i}\right\}_{i=1}^\infty=
  {1\over2},{3\over4},{7\over8},{15\over16},\ldots,
$$
is increasing, and
$$ 
  \left\{{n+1\over n}\right\}_{i=1}^\infty=
  {2\over1},{3\over2},{4\over3},{5\over4},\ldots
$$
is decreasing.
\end{example}

A sequence is {\dfont bounded above\index{sequence!bounded above}\/}
if there is some number $N$ such that $\ds a_n\le N$ for every $n$,
and {\dfont bounded below\index{sequence!bounded below}\/} if there is
some number $N$ such that $\ds a_n\ge N$ for every $n$. If a sequence
is bounded above and bounded below it is {\dfont
bounded\index{sequence!bounded}\/}. If a sequence $\ds
\{a_n\}_{n=0}^\infty$ is increasing or non-decreasing it is bounded
below (by $\ds a_0$), and if it is decreasing or non-increasing it is
bounded above (by $\ds a_0$).  Finally, with all this new terminology
we can state an important theorem.

\begin{theorem} If a sequence is bounded and monotonic then it converges.
\end{proof}

We will not prove this; the proof appears in many calculus books. It
is not hard to believe: suppose that a sequence is increasing and
bounded, so each term is larger than the one before, yet never larger
than some fixed value $N$. The terms must then get closer and closer
to some value between $\ds a_0$ and $N$. It need not be $N$, since $N$ may
be a ``too-generous'' upper bound; the limit will be the
smallest number that is above all of the terms $\ds a_i$.

\begin{example}
All of the terms $\ds (2^i-1)/2^i$ are less than 2, and the sequence is
increasing. As we have seen, the limit of the sequence is 1---1 is the
smallest number that is bigger than all the terms in the sequence.
Similarly, all of the terms $(n+1)/n$ are bigger than $1/2$, and the
limit is 1---1 is the largest number that is smaller than the terms of
the sequence.
\end{example}

We don't actually need to know that a sequence is monotonic to apply
this theorem---it is enough to know that the sequence is
``eventually'' monotonic, that is, that at some point it becomes
increasing or decreasing. For example, the sequence $10$, $9$, $8$,
$15$, $3$, $21$, $4$, $3/4$, $7/8$, $15/16$, $31/32,\ldots$ is not
increasing, because among the first few terms it is not. But starting
with the term $3/4$ it is increasing, so the theorem tells us that the
sequence $3/4, 7/8, 15/16, 31/32,\ldots$ converges.  Since convergence
depends only on what happens as $n$ gets large, adding a few
terms at the beginning can't turn a convergent sequence into a
divergent one.

\begin{example}
Show that $\ds\{n^{1/n}\}$ converges. 
\par\nobreak\ssk\noindent
We first show that 
this sequence is decreasing, that is, that $\ds n^{1/n}>
(n+1)^{1/(n+1)}$. Consider the real function $\ds f(x)=x^{1/x}$ when
$x\ge1$. We can compute the derivative, $\ds f'(x)=x^{1/x}(1-\ln x)/x^2$,
and note that when $x\ge 3$ this is negative. Since the function has
negative slope, $\ds n^{1/n}>
(n+1)^{1/(n+1)}$ when $n\ge 3$. Since all terms of the sequence are
positive, the sequence is decreasing and bounded when $n\ge3$, and so
the sequence converges. (As it happens, we can compute the limit in
this case, but we know it converges even without knowing the limit; see
exercise~\xrefn{exercise:exponential limit}.)
\end{example}

\begin{example}
Show that $\ds\{n!/n^n\}$ converges.
\par\nobreak\ssk\noindent
Again we show that the sequence is decreasing, and since each term is
positive the sequence converges. We can't take the derivative this
time, as $x!$ doesn't make sense for $x$ real. But we note that if 
$\ds a_{n+1}/a_n < 1$ then $\ds a_{n+1}< a_n$, which is what we want to
know. So we look at $\ds a_{n+1}/a_n$:
$$ 
  {a_{n+1}\over a_n} = {(n+1)!\over (n+1)^{n+1}}{n^n\over n!}=
  {(n+1)!\over n!}{n^n\over (n+1)^{n+1}}=
  {n+1\over n+1}\left({n\over n+1}\right)^n=
  \left({n\over n+1}\right)^n < 1.
$$
(Again it is possible to compute the limit; see
exercise~\xrefn{exercise:factorial limit}.)
\end{example}

\begin{exercises}

\begin{exercise} \label{exercise:exponential limit}
Compute $\ds\lim_{x\to\infty} x^{1/x}$.
\begin{answer} $1$
\end{answer}\end{exercise}

\begin{exercise} Use the squeeze theorem to show that 
$\ds\lim_{n\to\infty} {n!\over n^n}=0$.
\label{exercise:factorial limit}

\begin{exercise} Determine whether $\ds\{\sqrt{n+47}-\sqrt{n}\}_{n=0}^\infty$ 
converges or diverges. If it converges, compute the limit.
\begin{answer} $0$
\end{answer}\end{exercise}

\begin{exercise} Determine whether 
$\ds\left\{{n^2+1\over (n+1)^2}\right\}_{n=0}^\infty$ 
converges or diverges. If it converges, compute the limit.
\begin{answer} $1$
\end{answer}\end{exercise}

\begin{exercise} Determine whether 
$\ds\left\{{n+47\over\sqrt{n^2+3n}}\right\}_{n=1}^\infty$ 
converges or diverges. If it converges, compute the limit.
\begin{answer} $1$
\end{answer}\end{exercise}

\begin{exercise} Determine whether 
$\ds\left\{{2^n\over n!}\right\}_{n=0}^\infty$ 
converges or diverges. 
\begin{answer} $0$
\end{answer}\end{exercise}

\end{exercises}

