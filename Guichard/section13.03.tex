\section{Arc length and curvature}{}{}
\label{sec:arc length 3D}

\index{arc length}
Sometimes it is useful to compute the length of a curve in space; for
example, if the curve represents the path of a moving object, the
length of the curve between two points may be the distance traveled by
the object between two times.

Recall that if the curve is given by the vector function $\bf r$ then
the vector $\Delta {\bf r}=
{\bf r}(t+\Delta t)-{\bf r}(t)$ points from one position
on the curve to another, as depicted in figure~\xrefn{fig:vector
  derivative}. If the points are close together, the length of 
$\Delta {\bf r}$ is close to the length of the
curve between the two points. If we add up the lengths of many such
tiny vectors, placed head to tail along a segment of the curve, we get
an approximation to the length of the curve over that segment. In the
limit, as usual, this sum turns into an integral that computes precisely
the length of the curve. 
First, note that 
$$|\Delta {\bf r}|={|\Delta {\bf r}|\over \Delta t}\,\Delta t\approx
|{\bf r}'(t)|\,\Delta t,$$
when $\Delta t$ is small.
Then the length of the curve between
${\bf r}(a)$ and ${\bf r}(b)$ is 
$$\lim_{n\to\infty} \sum_{i=0}^{n-1}|\Delta {\bf r}|
=\lim_{n\to\infty} \sum_{i=0}^{n-1} {|\Delta {\bf r}|\over \Delta t}\,\Delta t
=\lim_{n\to\infty} \sum_{i=0}^{n-1} |{\bf r}'(t)|\,\Delta t=
\int_a^b|{\bf r}'(t)|\,dt.$$
(Well, sometimes. This works if between $a$ and $b$ the segment of curve
is traced out exactly once.)

\begin{example} Let's find the length of one turn of the helix
${\bf r}=\langle \cos t, \sin t, t\rangle$ (see figure~\xrefn{fig:helixes}).
We compute ${\bf r}'=\langle -\sin t, \cos t, 1\rangle$ and
$|{\bf r}'|=\sqrt{\sin^2 t+\cos^2 t+1}=\sqrt2$, so the length is
$$\int_0^{2\pi} \sqrt2\,dt = 2\sqrt2\pi.$$
\vskip-10pt
\end{example}

\begin{example} Suppose $y=\ln x$; what is the length of this curve between
$x=1$ and $x=\sqrt3$?

Although this problem does not appear to involve vectors or three
dimensions, we can interpret it in those terms: let 
${\bf r}(t)=\langle t,\ln t,0\rangle$. This vector function traces
out precisely $y=\ln x$ in the $x$-$y$ plane. Then 
${\bf r}'(t)=\langle 1,1/t,0\rangle$ and
$\ds |{\bf r}'(t)|=\sqrt{1+1/t^2}$ and the desired length is
$$\int_1^{\sqrt3} \sqrt{1+{1\over
    t^2}}\,dt=2-\sqrt2+\ln(\sqrt2+1)-{1\over2}\ln3.$$
(This integral is a bit tricky, but requires only methods we have
learned.) 
\end{example}

Notice that there is nothing special about $y=\ln x$, except that the
resulting integral can be computed. In general, given any $y=f(x)$, we
can think of this as the vector function
${\bf r}(t)=\langle t,f(t),0\rangle$. Then 
${\bf r}'(t)=\langle 1,f'(t),0\rangle$ and
$\ds |{\bf r}'(t)|=\sqrt{1+(f')^2}$. The length of the curve $y=f(x)$
between $a$ and $b$ is thus
$$\int_a^b \sqrt{1+(f'(x))^2}\,dx.$$
Unfortunately, such integrals are often impossible to do exactly and
must be approximated.

One useful application of arc length is the {\dfont arc length
  parameterization\index{arc length parameterization}}. A vector
function ${\bf r}(t)$ gives the position of a point in terms of the
parameter $t$, which is often time, but need not be. Suppose $s$ is
the distance along the curve from some fixed starting point; if we use
$s$ for the variable, we get ${\bf r}(s)$, the position in space in
terms of distance along the curve. We might still imagine that the
curve represents the position of a moving object; now we get the
position of the object as a function of how far the object has
traveled.

\begin{example} Suppose ${\bf r}(t)=\langle \cos t,\sin t,0\rangle$. We know
that this curve is a circle of radius 1. While $t$ might represent
time, it can also in this case represent the usual angle between the
positive $x$-axis and ${\bf r}(t)$. The distance along the circle from
$(1,0,0)$ to $(\cos t,\sin t,0)$ is also $t$---this is the
definition of radian measure. Thus, in this case $s=t$ and
${\bf r}(s)=\langle \cos s,\sin s,0\rangle$.
\end{example}

\begin{example} Suppose ${\bf r}(t)=\langle \cos t,\sin t,t\rangle$. We know
that this curve is a helix. The distance along the helix from $(1,0,0)$
to $(\cos t,\sin t,t)$ is 
$$s=\int_0^t |{\bf r}'(u)|\,du=\int_0^t \sqrt{\cos^2u+\sin^2u+1}\,du=
\int_0^t \sqrt{2}\,du=\sqrt2t.$$
Thus, the value of $t$ that gets us distance $s$ along the helix is
$t=s/\sqrt2$, and so the same curve is given by $\hat{\bf r}(s)=
\langle \cos(s/\sqrt2),\sin(s/\sqrt2),s/\sqrt2\rangle$.
\end{example}

In general, if we have a vector function ${\bf r}(t)$, to convert it
to a vector function in terms of arc length we compute
$$s=\int_a^t |{\bf r}'(u)|\,du=f(t),$$
solve $s=f(t)$ for $t$, getting $t=g(s)$, and substitute this back
into ${\bf r}(t)$ to get $\hat{\bf r}(s)={\bf r}(g(s))$.

Suppose that $t$ is time.
By the Fundamental Theorem of Calculus, if we start with
arc length
$$s(t)=\int_a^t |{\bf r}'(u)|\,du$$
and take the derivative, we get
$$s'(t)=|{\bf r}'(t)|.$$
Here $s'(t)$ is the rate at which the arc length is changing, and
we have seen that $|{\bf r}'(t)|$ is the speed of a moving object;
these are of course the same.

Suppose that ${\bf r}(s)$ is given in terms of arc length; what is
$|{\bf r}'(s)|$? It is the rate at which arc length is changing {\it
  relative to arc length}; it must be 1! In the case of the helix, for
example, the arc length parameterization is $\langle
\cos(s/\sqrt2),\sin(s/\sqrt2),s/\sqrt2\rangle$, the derivative is
$\langle -\sin(s/\sqrt2)/\sqrt2,\cos(s/\sqrt2)/\sqrt2,1/\sqrt2\rangle$,
and the length of this is
$$\sqrt{{\sin^2(s/\sqrt2)\over2}+{\cos^2(s/\sqrt2)\over2}+{1\over2}}=
\sqrt{{1\over2}+{1\over2}}=1.$$
So in general, ${\bf r}'$ is a unit tangent vector.

Given a curve ${\bf r}(t)$, we would like to be able to measure, at
various points, how sharply curved it is. Clearly this is related to
how ``fast'' a tangent vector is changing direction, so a first guess
might be that we can measure curvature with $|{\bf r}''(t)|$. A little
thought shows that this is flawed; if we think of $t$ as time, for
example, we could be tracing out the curve more or less quickly
as time passes. The second derivative $|{\bf r}''(t)|$ incorporates
this notion of time, so it depends not simply on the geometric
properties of the curve but on how quickly we move along the curve.

\begin{example} Consider ${\bf r}(t)=\langle \cos t,\sin t,0\rangle$ and 
${\bf s}(t)=\langle \cos 2t,\sin 2t,0\rangle$. Both of these vector
functions represent the unit circle in the $x$-$y$ plane, but if $t$
is interpreted as time, the second describes an object moving twice as
fast as the first. Computing the second derivatives, we find
$|{\bf r}''(t)|=1$, $|{\bf s}''(t)|=2$.
\end{example}

To remove the dependence on time, we use the arc length
parameterization. If a curve is given by ${\bf r}(s)$, then the first
derivative ${\bf r}'(s)$ is a unit vector, that is, 
${\bf r}'(s)={\bf T}(s)$. We now compute the second derivative
${\bf r}''(s)={\bf T}'(s)$ and use  $|{\bf T}'(s)|$ as the
``official'' measure of 
{\dfont curvature\index{curvature}}, usually denoted $\kappa$.

\begin{example} We have seen that the arc length parameterization of a 
particular helix is ${\bf r}(s)=
\langle \cos(s/\sqrt2),\sin(s/\sqrt2),s/\sqrt2\rangle$.
Computing the second derivative gives
${\bf r}''(s)=
\langle -\cos(s/\sqrt2)/2,-\sin(s/\sqrt2)/2,0\rangle$ with length $1/2$.
\end{example} 

What if we are given a curve as a vector function ${\bf r}(t)$, where
$t$ is not arc length? We have seen that arc length can be difficult
to compute; fortunately, we do not need to convert to the arc length
parameterization to compute curvature. Instead, let us imagine that we have
done this, so we have found $t=g(s)$ and then formed
$\hat{\bf r}(s)={\bf r}(g(s))$. The first derivative $\hat{\bf r}'(s)$
is a unit tangent vector, so it is the same as the unit tangent vector
${\bf T}(t)={\bf T}(g(s))$. Taking the derivative of this we get
$${d\over ds}{\bf T}(g(s))= {\bf T}'(g(s)) g'(s)={\bf T}'(t){dt\over
  ds}.$$
The curvature\index{curvature formula} is the length of this vector:
$$\kappa = |{\bf T}'(t)||{dt\over ds}|={|{\bf T}'(t)|\over|ds/dt|}=
{|{\bf T}'(t)|\over|{\bf r}'(t)|}.$$
(Recall that we have seen that $ds/dt=|{\bf r}'(t)|$.) Thus we can
compute the curvature by computing only derivatives with respect to
$t$; we do not need to do the conversion to arc length.

\begin{example} 
Returning to the helix, suppose we start with the parameterization
${\bf r}(t)=\langle \cos t,\sin t,t\rangle$. Then 
${\bf r}'(t)=\langle -\sin t,\cos t,1\rangle$, 
$|{\bf r}'(t)|=\sqrt2$, and ${\bf T}(t)=\langle -\sin t,\cos
t,1\rangle/\sqrt2$. Then
${\bf T}'(t)=\langle -\cos t,-\sin t,0\rangle/\sqrt2$ and 
$|{\bf T}'(t)|=1/\sqrt2$. Finally, $\kappa=1/\sqrt2/\sqrt2=1/2$,
as before.
\end{example} 

\begin{example} 
Consider this circle of radius $a$:
${\bf r}(t)=\langle a\cos t,a\sin t,1\rangle$. Then 
${\bf r}'(t)=\langle -a\sin t,a\cos t,0\rangle$, 
$|{\bf r}'(t)|=a$, and ${\bf T}(t)=\langle -a\sin t,a\cos
t,0\rangle/a$. Now
${\bf T}'(t)=\langle -a\cos t,-a\sin t,0\rangle/a$ and 
$|{\bf T}'(t)|=1$. Finally, $\kappa=1/a$:
the curvature of a circle is everywhere the inverse of the radius. It
is sometimes useful to think of curvature as describing what circle a
curve most resembles at a point. The curvature of the helix in the
previous example is $1/2$; this means that a small piece of the helix
looks very much like a circle of radius $2$, as shown in
figure~\xrefn{fig:osculating circle}.
\end{example} 

\figure
\vbox{\beginpicture
\normalgraphs
\ninepoint
\setcoordinatesystem units <3truecm,3truecm>
\setplotarea x from 0 to 1.1, y from 0 to 1.1
\put {\hbox{\epsfxsize5cm\epsfbox{osculating_circle.eps}}} at 0 0
\endpicture}
\figrdef{fig:osculating circle}
\endfigure{A circle with the same curvature as the helix.
(\expandafter\url\expandafter{\liveurl jmol_circle_helix}%
AP\endurl)}
%(\expandafter\url\expandafter{\sageurl 2220}%
%AP\endurl)}

\begin{example} Consider ${\bf r}(t)=
\langle \cos t,\sin
t,\cos 2t\rangle$, as shown in figure~\xrefn{fig:roller coaster}.
${\bf r}'(t)=
\langle -\sin t,\cos
t,-2\sin (2t)\rangle$ and $|{\bf r}'(t)|=
\sqrt{1+4\sin^2(2t)}$, so 
$${\bf T}(t)=\left\langle {-\sin t\over \sqrt{1+4\sin^2(2t)}},
{\cos t\over \sqrt{1+4\sin^2(2t)}},
{-2\sin 2t\over \sqrt{1+4\sin^2(2t)}}\right\rangle.$$
Computing the derivative of this and then the length of the resulting
vector is possible but unpleasant.
\end{example}

Fortunately, there is an alternate formula for the
curvature\index{curvature formula} that is
often simpler than the one we have:
$$\kappa = {|{\bf r}'(t)\times{\bf r}''(t)|\over|{\bf r}'(t)|^3}.$$

\begin{example} Returning to the previous example, we compute the second derivative
${\bf r}''(t)=
\langle -\cos t,-\sin t,-4\cos(2t)\rangle$. Then the cross product 
${\bf r}'(t)\times{\bf r}''(t)$ is 
$$\langle -4\cos t\cos 2t-2\sin t\sin 2t,
2\cos t\sin 2t-4\sin t \cos2t,1\rangle.$$ Computing the length of this
vector and dividing by $|{\bf r}'(t)|^3$ is still a bit tedious.
With the aid of a computer we get
$$\kappa = {\sqrt{48\cos^4 t - 48\cos^2 t + 17}\over
(-16\cos^4 t +16\cos^2t+1)^{3/2}}.$$
Graphing this we get
$$\vbox{\beginpicture
\normalgraphs
\ninepoint
\setcoordinatesystem units <10truemm,10truemm>
\setplotarea x from 0 to 6.28, y from 0 to 4.2
\axis left ticks length <2pt> withvalues {$2$} {$4$} / at 2 4 / /
\axis bottom ticks length <2pt> withvalues {$\pi\over2$} {$\pi$} 
{$3\pi\over2$} {$2\pi$} / at 1.57 3.14 4.71 6.28 / /
\plot 
0.000 4.123 0.031 4.022 0.063 3.742 0.094 3.343 0.126 2.894 
0.157 2.451 0.188 2.048 0.220 1.699 0.251 1.408 0.283 1.169 
0.314 0.975 0.346 0.819 0.377 0.692 0.408 0.590 0.440 0.507 
0.471 0.439 0.503 0.384 0.534 0.340 0.565 0.303 0.597 0.273 
0.628 0.250 0.660 0.231 0.691 0.217 0.723 0.208 0.754 0.202 
0.785 0.200 0.817 0.202 0.848 0.208 0.880 0.217 0.911 0.231 
0.942 0.250 0.974 0.273 1.005 0.303 1.037 0.340 1.068 0.384 
1.100 0.439 1.131 0.507 1.162 0.590 1.194 0.692 1.225 0.819 
1.257 0.975 1.288 1.169 1.319 1.408 1.351 1.699 1.382 2.048 
1.414 2.451 1.445 2.894 1.477 3.343 1.508 3.742 1.539 4.022 
1.571 4.123 1.602 4.022 1.634 3.742 1.665 3.343 1.696 2.894 
1.728 2.451 1.759 2.048 1.791 1.699 1.822 1.408 1.854 1.169 
1.885 0.975 1.916 0.819 1.948 0.692 1.979 0.590 2.011 0.507 
2.042 0.439 2.073 0.384 2.105 0.340 2.136 0.303 2.168 0.273 
2.199 0.250 2.231 0.231 2.262 0.217 2.293 0.208 2.325 0.202 
2.356 0.200 2.388 0.202 2.419 0.208 2.450 0.217 2.482 0.231 
2.513 0.250 2.545 0.273 2.576 0.303 2.608 0.340 2.639 0.384 
2.670 0.439 2.702 0.507 2.733 0.590 2.765 0.692 2.796 0.819 
2.827 0.975 2.859 1.169 2.890 1.408 2.922 1.699 2.953 2.048 
2.985 2.451 3.016 2.894 3.047 3.343 3.079 3.742 3.110 4.022 
3.142 4.123 3.173 4.022 3.204 3.742 3.236 3.343 3.267 2.894 
3.299 2.451 3.330 2.048 3.362 1.699 3.393 1.408 3.424 1.169 
3.456 0.975 3.487 0.819 3.519 0.692 3.550 0.590 3.581 0.507 
3.613 0.439 3.644 0.384 3.676 0.340 3.707 0.303 3.738 0.273 
3.770 0.250 3.801 0.231 3.833 0.217 3.864 0.208 3.896 0.202 
3.927 0.200 3.958 0.202 3.990 0.208 4.021 0.217 4.053 0.231 
4.084 0.250 4.115 0.273 4.147 0.303 4.178 0.340 4.210 0.384 
4.241 0.439 4.273 0.507 4.304 0.590 4.335 0.692 4.367 0.819 
4.398 0.975 4.430 1.169 4.461 1.408 4.492 1.699 4.524 2.048 
4.555 2.451 4.587 2.894 4.618 3.343 4.650 3.742 4.681 4.022 
4.712 4.123 4.744 4.022 4.775 3.742 4.807 3.343 4.838 2.894 
4.869 2.451 4.901 2.048 4.932 1.699 4.964 1.408 4.995 1.169 
5.027 0.975 5.058 0.819 5.089 0.692 5.121 0.590 5.152 0.507 
5.184 0.439 5.215 0.384 5.246 0.340 5.278 0.303 5.309 0.273 
5.341 0.250 5.372 0.231 5.404 0.217 5.435 0.208 5.466 0.202 
5.498 0.200 5.529 0.202 5.561 0.208 5.592 0.217 5.623 0.231 
5.655 0.250 5.686 0.273 5.718 0.303 5.749 0.340 5.781 0.384 
5.812 0.439 5.843 0.507 5.875 0.590 5.906 0.692 5.938 0.819 
5.969 0.975 6.000 1.169 6.032 1.408 6.063 1.699 6.095 2.048 
6.126 2.451 6.158 2.894 6.189 3.343 6.220 3.742 6.252 4.022 
6.283 4.123 /
\endpicture
}$$
Compare this to figure~\xrefn{fig:roller coaster}---you may
want to load the Java applet there so that you can see it from
different angles. The highest
curvature occurs where the curve has its highest and lowest points,
and indeed in the picture these appear to be the most sharply curved
portions of the curve, while the curve is almost a straight line
midway between those points. 
\end{example}

Let's see why this alternate formula is correct. Starting with the
definition of ${\bf T}$,
${\bf r}'=|{\bf r}'|{\bf T}$ so by the product rule
${\bf r}''=|{\bf r}'|'{\bf T}+|{\bf r}'|{\bf T}'$. Then by
Theorem~\xrefn{thm:cross product properties} the cross product is
$$\eqalign{
{\bf r}'\times{\bf r}''&=|{\bf r}'|{\bf T}\times|{\bf r}'|'{\bf T}+
|{\bf r}'|{\bf T}\times|{\bf r}'|{\bf T}' \\
&=|{\bf r}'||{\bf r}'|'({\bf T}\times{\bf T})+|{\bf r}'|^2
({\bf T}\times{\bf T}') \\
&=|{\bf r}'|^2({\bf T}\times{\bf T}') \\
}$$
because ${\bf T}\times{\bf T}={\bf 0}$, since ${\bf T}$ is
parallel to itself. Then
$$\eqalign{
|{\bf r}'\times{\bf r}''|&=|{\bf r}'|^2|{\bf T}\times{\bf T}'| \\
&=|{\bf r}'|^2|{\bf T}||{\bf T}'|\sin\theta \\
&=|{\bf r}'|^2|{\bf T}'| \\
}$$
using 
exercise~\xrefn{exercise:derivative is perpendicular} in 
section~\xrefn{section:calculus with vector functions}
to see that $\theta=\pi/2$. Dividing both sides by 
$|{\bf r}'|^3$ then gives the desired formula.

We used the fact here that ${\bf T}'$ is perpendicular to ${\bf T}$;
the vector ${\bf N}={\bf T}'/|{\bf T}'|$ is thus a unit vector
perpendicular to ${\bf T}$, called the {\dfont unit normal\index{unit
    normal}\index{normal}\/} 
to the curve. Occasionally of use is the {\dfont unit
  binormal\index{unit binormal}\index{binormal}\/} ${\bf B}={\bf T}\times{\bf N}$, a
unit vector perpendicular to both ${\bf T}$ and ${\bf N}$.

\begin{exercises}

\begin{exercise} Find the length of $\langle 3\cos t,2t,3\sin t\rangle$, 
$t\in[0,2\pi]$.
\begin{answer} $2\pi\sqrt{13}$
\end{answer}\end{exercise}

\begin{exercise} Find the length of $\langle t^2,2,t^3\rangle$, $t\in[0,1]$.
\begin{answer} $(-8+13\sqrt{13})/27$
\end{answer}\end{exercise}

\begin{exercise} Find the length of $\langle t^2,\sin t,\cos t\rangle$, $t\in[0,1]$.
\begin{answer} $\sqrt5/2+\ln(\sqrt5+2)/4$
\end{answer}\end{exercise}

\begin{exercise} Find the length of the curve $y=x^{3/2}$, $x\in[1,9]$.
\begin{answer} $(85\sqrt{85}-13\sqrt{13})/27$
\end{answer}\end{exercise}

\begin{exercise} Set up an integral to compute the length of
$\langle \cos t, \sin t, e^t\rangle$, $t\in[0,5]$. (It is tedious but
not too difficult to compute this integral.)
\begin{answer} $\int_0^5 \sqrt{1+e^{2t}}\,dt$
\end{answer}\end{exercise}

\begin{exercise} Find the curvature of $\langle t,t^2,t\rangle$.
\begin{answer} $2\sqrt2/(2+4t^2)^{3/2}$
\end{answer}\end{exercise}

\begin{exercise} Find the curvature of $\langle t,t^2,t^2\rangle$.
\begin{answer} $2\sqrt2/(1+8t^2)^{3/2}$
\end{answer}\end{exercise}

\begin{exercise} Find the curvature of $\langle t,t^2,t^3\rangle$.
\begin{answer} $2\sqrt{1+9t^2+9t^4}/(1+4t^2+9t^4)^{3/2}$
\end{answer}\end{exercise}

\begin{exercise} Find the curvature of $y=x^4$ at $(1,1)$.
\begin{answer} $12\sqrt{17}/289$
\end{answer}\end{exercise}

\end{exercises}

