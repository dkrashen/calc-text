\section{Functions of Several Variables}{}{}

In single-variable calculus we were concerned with functions that map
the real numbers $\R$ to $\R$, sometimes called ``real functions of
one variable'', meaning the ``input'' is a single real number and the
``output'' is likewise a single real number. In the last chapter we
considered functions taking a real number to a vector, which may also
be viewed as functions $f\colon\R\to\R^3$, that is, for each input
value we get a position in space. Now we turn to functions of several
variables, meaning several input variables, functions
$f\colon\R^n\to\R$. We will deal primarily with $n=2$ and to a lesser
extent $n=3$; in fact many of the techniques we discuss can be applied
to larger values of $n$ as well.

A function $f\colon\R^2\to\R$ maps a pair of values $(x,y)$ to a
single real number. The three-dimensional coordinate system we have
already used is a convenient way to visualize such functions: above
each point $(x,y)$ in the $x$-$y$ plane we graph the point $(x,y,z)$,
where of course $z=f(x,y)$. 

\begin{example} Consider $f(x,y)=3x+4y-5$. Writing this as 
$z=3x+4y-5$ and then $3x+4y-z=5$ we recognize the equation of a
plane. In the form $f(x,y)=3x+4y-5$ the emphasis has shifted: we now
think of $x$ and $y$ as independent variables and $z$ as a variable
dependent on them, but the geometry is unchanged.
\end{example}

\begin{example} We have seen that $x^2+y^2+z^2=4$ represents a sphere of radius
2. We cannot write this in the form $f(x,y)$, since for
each $x$ and $y$ in the disk $x^2+y^2<4$ there are two corresponding
points on the sphere. As with the equation of a circle, we can resolve
this equation into two functions, $\ds f(x,y)=\sqrt{4-x^2-y^2}$ and 
$\ds f(x,y)=-\sqrt{4-x^2-y^2}$, representing the upper and lower
hemispheres. Each of these is an example of a function with a
restricted domain: only certain values of $x$ and $y$ make sense
(namely, those for which $x^2+y^2\le 4$) and the graphs of these
functions are limited to a small region of the plane.
\end{example}

\begin{example} Consider $f=\sqrt x+\sqrt y$. This function is defined only when
both $x$ and $y$ are non-negative. When $y=0$ we get $f(x,y)=\sqrt x$,
the familiar square root function in the $x$-$z$ plane, and when $x=0$
we get the same curve in the $y$-$z$ plane. Generally speaking, we see
that starting from $f(0,0)=0$ this function gets larger in every direction
in roughly the same way that the square root function gets
larger. For example, if we restrict attention to the line $x=y$, we
get $f(x,y)=2\sqrt x$ and along the line $y=2x$ we have $f(x,y)=\sqrt
x+\sqrt{2x}=(1+\sqrt2)\sqrt x$.
\end{example}

\figure
\vbox{\beginpicture
\normalgraphs
\ninepoint
\setcoordinatesystem units <3truecm,3truecm>
\setplotarea x from 0 to 1.1, y from 0 to 1.1
\put {\hbox{\epsfxsize5cm\epsfbox{square_root.eps}}} at 0 0
\endpicture}
\figrdef{fig:double square root}
\endfigure{$f(x,y)=\sqrt x+\sqrt y$
(\expandafter\url\expandafter{\liveurl square_root.html}%
AP\endurl)}

A computer program that plots such surfaces can be very useful, as it
is often difficult to get a good idea of what they look like. Still,
it is valuable to be able to visualize relatively simple surfaces
without such aids. As in the previous example, it is often a good idea
to examine the function on restricted subsets of the plane, especially
lines. It can also be useful to identify those points $(x,y)$ that
share a common $z$-value.

\begin{example} Consider $f(x,y)=x^2+y^2$. When $x=0$ this becomes
$f=y^2$, a parabola in the $y$-$z$ plane; when $y=0$ we get the
``same'' parabola $f=x^2$ in the $x$-$z$ plane. 
Now consider the line $y=kx$. If we simply replace $y$ by $kx$ we get
$f(x,y)=(1+k^2)x^2$ which is a parabola, but it does not really
``represent'' the cross-section along $y=kx$, because the cross-section
has the line $y=kx$ where the horizontal axis should be.
In order to pretend that this line is the horizontal axis, we need to
write the function in terms of the distance from the origin, which is
$\ds \sqrt{x^2+y^2}=\sqrt{x^2+k^2x^2}$. Now
$\ds f(x,y)=x^2+k^2x^2=(\sqrt{x^2+k^2x^2})^2$. So the cross-section is the
``same'' parabola as in the $x$-$z$ and $y$-$z$ planes, namely, the
height is always the distance from the origin squared. This means that 
$f(x,y)=x^2+y^2$ can be formed by starting with $z=x^2$ and rotating
this curve around the $z$ axis.

Finally, picking a value $z=k$, at what points does
$f(x,y)=k$? This means $x^2+y^2=k$, which we recognize as the equation
of a circle of radius $\sqrt k$. So the graph of $f(x,y)$ has
parabolic cross-sections, and the same height everywhere on concentric
circles with center at the origin. This fits with what we have already
discovered. 
\end{example}

\figure
\vbox{\beginpicture
\normalgraphs
\ninepoint
\setcoordinatesystem units <3truecm,3truecm>
\setplotarea x from 0 to 1.1, y from 0 to 1.1
\put {\hbox{\epsfxsize5cm\epsfbox{parabolic_bowl.eps}}} at 0 0
\put {\hbox{\epsfxsize4cm\epsfbox{parabolic_level_curves.eps}}} at 2 0
\endpicture}
\figrdef{fig:parabolic bowl}
\endfigure{$f(x,y)=x^2 + y^2$
(\expandafter\url\expandafter{\liveurl parabolic_bowl.html}%
AP\endurl)}

As in this example, the points $(x,y)$ such that $f(x,y)=k$ usually
form a curve, called a {\dfont level curve\index{level curve}} of the
function. A graph of some level curves can give a good idea of the
shape of the surface; it looks much like a topographic map of the
surface. In figure~\xrefn{fig:parabolic bowl} both the surface and
its associated level curves are shown. Note that, as with a
topographic map, the heights corresponding to the level curves are
evenly spaced, so that where curves are closer together the surface is
steeper.

Functions $f\colon \R^n\to\R$ behave much like functions of two
variables; we will on occasion discuss functions of three variables.
The principal difficulty with such functions is visualizing them, as
they do not ``fit'' in the three dimensions we are familiar with. For
three variables there are various ways to interpret functions that
make them easier to understand. For example, $f(x,y,z)$ could
represent the temperature at the point $(x,y,z)$, or the pressure, or
the strength of a magnetic field.  It remains useful to consider those
points at which $f(x,y,z)=k$, where $k$ is some constant value. If
$f(x,y,z)$ is temperature, the set of points $(x,y,z)$ such that
$f(x,y,z)=k$ is the collection of points in space with temperature $k$;
in general this is called a {\dfont level set\index{level set}}; for
three variables, a level set is typically a surface, called a {\dfont
  level surface\index{level surface}}.

\begin{example} Suppose the temperature at $(x,y,z)$ is 
$T(x,y,z)=e^{-(x^2+y^2+z^2)}$. This function has a maximum value of 1
at the origin, and tends to 0 in all directions. If $k$ is positive
and at most 1,
the set of points for which $T(x,y,z)=k$ is those points satisfying
$x^2+y^2+z^2=-\ln k$, a sphere centered at the origin. The level
surfaces are the concentric spheres centered at the origin.
\end{example}

\begin{exercises}

\begin{exercise} Let $f(x,y)=(x-y)^2$. 
Determine the equations and shapes of the cross-sections when
$x=0$, $y=0$, $x=y$, and describe the level curves.
Use a three-dimensional graphing tool to graph the surface.
\begin{answer} $z=y^2$, $z=x^2$, $z=0$, lines of slope 1
\end{answer}\end{exercise}

\begin{exercise} Let $f(x,y)=|x|+|y|$. 
Determine the equations and shapes of the cross-sections when
$x=0$, $y=0$, $x=y$, and describe the level curves.
Use a three-dimensional graphing tool to graph the surface.
\begin{answer} $z=|y|$, $z=|x|$, $z=2|x|$, diamonds
\end{answer}\end{exercise}

\begin{exercise} Let $f(x,y)=e^{-(x^2+y^2)}\sin(x^2+y^2)$. 
Determine the equations and shapes of the cross-sections when
$x=0$, $y=0$, $x=y$, and describe the level curves.
Use a three-dimensional graphing tool to graph the surface.
\begin{answer} $z=e^{-y^2}\sin(y^2)$, $z=e^{-x^2}\sin(x^2)$, 
$z=e^{-2x^2}\sin(2x^2)$, circles
\end{answer}\end{exercise}

\begin{exercise} Let $f(x,y)=\sin(x-y)$. 
Determine the equations and shapes of the cross-sections when
$x=0$, $y=0$, $x=y$, and describe the level curves.
Use a three-dimensional graphing tool to graph the surface.
\begin{answer} $z=-\sin(y)$, $z=\sin(x)$, 
$z=0$, lines of slope 1
\end{answer}\end{exercise}

\begin{exercise} Let $f(x,y)=(x^2-y^2)^2$. 
Determine the equations and shapes of the cross-sections when
$x=0$, $y=0$, $x=y$, and describe the level curves.
Use a three-dimensional graphing tool to graph the surface.
\begin{answer} $z=y^4$, $z=x^4$, 
$z=0$, hyperbolas
\end{answer}\end{exercise}

%% Balof
%% \begin{exercise}
%%  Classify the following surfaces, completing the square where necessary.
%% 
%% \begin{itemize} % BADBAD
%% \item{a.} $z^2=4x^2+9y^2+144$
%% \item{b.} $x^2-y^2+z^2-2x+2y+4z+2=0$
%% \item{c.} $-4x^2+y^2-4z^2=4$
%% \end{itemize}

\begin{exercise} Find the domain of each of the following functions of two variables:

\begin{itemize} % BADBAD
\item{a.}  $\ds\sqrt{9-x^2}+\sqrt{y^2-4}$
\item{b.}  $\arcsin(x^2+y^2-2)$
\item{c.}  $\ds\sqrt{16-x^2-4y^2}$
\end{itemize}
\begin{answer} (a) $\{(x,y)\mid |x|\le3\ \hbox{and}\ |y|\ge2\}$\hfill\break
(b) $\{(x,y)\mid 1\le x^2+y^2\le3\}$\hfill\break
(c) $\{(x,y)\mid x^2+4y^2\le16\}$
\end{answer}\end{exercise}

\begin{exercise} Below are two sets of level curves.  One is for a cone, one
is for a paraboloid.  Which is which? Explain.

\nobreak
\hbox to \hsize{\hfill\epsfxsize4cm\epsfbox{parabolic_level_curves_2.eps}
\hfill\epsfxsize4cm\epsfbox{conical_level_curves.eps}\hfill}

\end{exercises}

