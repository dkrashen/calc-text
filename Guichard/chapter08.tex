\iflatetranscendentals
\chapter{Applications of Integration}

\input section09.01
\input section09.02
\input section09.03
\input section09.04
\input section09.05
\else
\chapter{Techniques of Integration}

Over the next few sections we examine some techniques that are
frequently successful when seeking antiderivatives of
functions. Sometimes this is a simple problem, since it will be
apparent that the function you wish to integrate is a derivative in
some straightforward way. For example, faced with
$$\int x^{10}\,dx$$
we realize immediately that the derivative of $\ds x^{11}$ will supply an
$\ds x^{10}$: $\ds (x^{11})'=11x^{10}$. We don't want the ``11'', but
constants are easy to alter, because differentiation
``ignores'' them in certain circumstances, so
$${d\over dx}{1\over 11}{x^{11}}={1\over 11}11{x^{10}}=x^{10}.$$

From our knowledge of derivatives, we can immediately write down a
number of antiderivatives. Here is a list of those most often used:

$$\displaylines{
\int x^n\,dx={x^{n+1}\over n+1}+C, \quad\hbox{if $n\not=-1$} \\
\int x^{-1}\,dx = \ln |x|+C \\
\int e^x\,dx = e^x+C \\
\int \sin x\,dx = -\cos x+C \\
\int \cos x\,dx = \sin x+C \\
\int \sec^2 x\,dx = \tan x+C \\
\int \sec x\tan x\,dx = \sec x+C \\
\int {1\over1+x^2}\,dx = \arctan x+C \\
\int {1\over \sqrt{1-x^2}}\,dx = \arcsin x+C \\
}$$

\input section08.01
\input section08.02
\input section08.03
\input section08.04
\input section08.05
\input section08.06
\input section08.07
\fi
