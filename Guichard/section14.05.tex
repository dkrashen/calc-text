\section{Directional Derivatives}{}{}

We still have not answered one of our first questions about the
steepness of a surface: starting at a
point on a surface given by $f(x,y)$, and walking in a particular
direction, how steep is the surface? We are now ready to answer the
question.

We already know roughly what has to be done: as shown in
figure~\xrefn{fig:parabolic bowl cross-section}, we extend a line in
the $x$-$y$ plane to a vertical plane, and we then compute the slope
of the curve that is the cross-section of the surface in that
plane. The major stumbling block is that what appears in this plane to
be the horizontal axis, namely the line in the $x$-$y$ plane, is not
an actual axis---we know nothing about the ``units'' along the
axis. Our goal is to make this line into a $t$ axis; then we need
formulas to write $x$ and $y$ in terms of this new variable $t$; then
we can write $z$ in terms of $t$ since we know $z$ in terms of $x$ and
$y$; and finally we can simply take the derivative.

So we need to somehow ``mark off'' units on the line, and we need a
convenient way to refer to the line in calculations. It turns out that
we can accomplish both by using the vector form of a line. Suppose
that ${\bf u}$ is a unit vector $\langle u_1,u_2\rangle$ in the
direction of interest. A vector equation for the line through
$(x_0,y_0)$ in this direction is ${\bf v}(t)=\langle
u_1t+x_0,u_2t+y_0\rangle$. The height of the surface above the point 
$(u_1t+x_0,u_2t+y_0)$ is $g(t)=f(u_1t+x_0,u_2t+y_0)$. Because $\bf u$
is a unit vector, the value of $t$ is precisely the distance along the
line from $(x_0,y_0)$ to $(u_1t+x_0,u_2t+y_0)$; this means that the
line is effectively a $t$ axis, with origin at the point $(x_0,y_0)$,
so the slope we seek is 
$$\eqalign{
g'(0)&=\langle f_x(x_0,y_0),f_y(x_0,y_0)\rangle\cdot
\langle u_1,u_2\rangle \\
&=\langle f_x,f_y\rangle\cdot{\bf u} \\
&=\nabla f\cdot {\bf u} \\
}$$
Here we have used the chain rule and the derivatives
${d\over dt}(u_1t+x_0)=u_1$ and ${d\over dt}(u_2t+y_0)=u_2$.
The vector $\langle f_x,f_y\rangle$ is very useful, so it has its own
symbol, $\nabla f$, pronounced ``del f''\index{del ($\nabla$)}; 
it is also called the
{\dfont gradient\index{gradient}\/} of $f$.

\begin{example} Find the slope of $z=x^2+y^2$ at $(1,2)$ in the direction of the
vector $\langle 3,4\rangle$. 

We first compute the gradient at $(1,2)$:
$\nabla f=\langle 2x,2y\rangle$, which is $\langle 2,4\rangle$ at
$(1,2)$. A unit vector in the desired direction is $\langle
3/5,4/5\rangle$, and the desired slope is then
$\langle 2,4\rangle\cdot\langle 3/5,4/5\rangle=6/5+16/5=22/5$.
\end{example}

\begin{example} Find a tangent vector to $z=x^2+y^2$ at $(1,2)$ in the direction of the
vector $\langle 3,4\rangle$ and show that it is parallel to the
tangent plane at that point.

Since $\langle 3/5,4/5\rangle$ is a unit vector in the desired
direction, we can easily expand it to a tangent vector simply by
adding the third coordinate: $\langle 3/5,4/5,22/5\rangle$.
To see that this vector is parallel to the tangent plane, we 
can compute its dot product with a normal to the plane. We know that a
normal to the tangent plane is
$$\langle f_x(1,2),f_y(1,2),-1\rangle = \langle 2,4,-1\rangle,$$
and the dot product is $\langle 2,4,-1\rangle\cdot\langle
3/5,4/5,22/5\rangle=6/5+16/5-22/5=0$, so the two vectors are
perpendicular. (Note that the vector normal to the surface, namely 
$\langle f_x,f_y,-1\rangle$, is simply the gradient with a $-1$ tacked
on as the third component.)
\end{example}

The slope of a surface given by $z=f(x,y)$ in the direction of a
(two-dimensional) vector $\bf u$ is called the {\dfont directional
  derivative\index{directional derivative}\/} of $f$, written $D_{\bf u}f$.
The directional derivative immediately provides us with some
additional information. We know that 
$$D_{\bf u}f=\nabla f\cdot {\bf u}=|\nabla f||{\bf u}|\cos\theta=
|\nabla f|\cos\theta$$
if $\bf u$ is a unit vector; $\theta$ is the angle between $\nabla f$
and $\bf u$. This tells us immediately that the largest value of
$D_{\bf u}f$ occurs when $\cos\theta=1$, namely, when $\theta=0$, so 
$\nabla f$ is parallel to $\bf u$. In other words, the gradient
$\nabla f$ points in the direction of steepest ascent of the surface, and 
$|\nabla f|$ is the slope in that direction. Likewise, the smallest value of
$D_{\bf u}f$ occurs when $\cos\theta=-1$, namely, when $\theta=\pi$, so 
$\nabla f$ is anti-parallel to $\bf u$. In other words, 
$-\nabla f$ points in the direction of steepest descent of the surface, and 
$-|\nabla f|$ is the slope in that direction.

\begin{example} Investigate the direction of steepest ascent and descent for
$z=x^2+y^2$. 

The gradient is $\langle 2x,2y\rangle=2\langle x,y\rangle$; this is a
vector parallel to the vector $\langle x,y\rangle$, so the direction
of steepest ascent is directly away from the origin, starting at the
point $(x,y)$. The direction of steepest descent is thus directly
toward the origin from $(x,y)$. Note that at $(0,0)$ the gradient
vector is $\langle 0,0\rangle$, which has no direction, and it is
clear from the plot of this surface that there is a minimum point at
the origin, and tangent vectors in all directions are parallel to the
$x$-$y$ plane.
\end{example}

If $\nabla f$ is perpendicular to $\bf u$, 
$D_{\bf u}f=|\nabla f|\cos(\pi/2)=0$, since $\cos(\pi/2)=0$.
This means that in either of the two directions perpendicular to
$\nabla f$, the slope of the surface is 0; this implies that a vector in
either of these directions is tangent to the level curve at that
point. Starting with $\nabla f=\langle f_x,f_y\rangle$, it is easy to
find a vector perpendicular to it: either $\langle f_y,-f_x\rangle$ or 
$\langle -f_y,f_x\rangle$ will work.

If $f(x,y,z)$ is a function of three variables, all the calculations
proceed in essentially the same way. The rate at which $f$ changes in
a particular direction is $\nabla f\cdot{\bf u}$, where now
$\nabla f=\langle f_x,f_y,f_z\rangle$ and ${\bf u}=\langle
u_1,u_2,u_3\rangle$ is a unit vector. Again $\nabla f$ points in the
direction of maximum rate of increase, $-\nabla f$ points in the
direction of maximum rate of decrease, and any vector perpendicular to
$\nabla f$ is tangent to the level surface $f(x,y,z)=k$ at the point
in question. Of course there are no longer just two such vectors; the
vectors perpendicular to $\nabla f$ describe the tangent plane to the
level surface, or in other words $\nabla f$ is a normal to the tangent
plane. 

\begin{example} Suppose the temperature at a point in space is given by
$T(x,y,z)=T_0/(1+x^2+y^2+z^2)$; at the origin the temperature in
Kelvin is
$T_0>0$, and it decreases in every direction from there. It might be,
for example, that there is a source of heat at the origin, and as we
get farther from the source, the temperature decreases.
The gradient is
$$\eqalign{
\nabla T&=\langle {-2T_0x\over (1+x^2+y^2+z^2)^2}+
{-2T_0x\over (1+x^2+y^2+z^2)^2}+{-2T_0x\over (1+x^2+y^2+z^2)^2}\rangle \\
&={-2T_0\over (1+x^2+y^2+z^2)^2}\langle x,y,z\rangle. \\
}$$
The gradient points directly at the origin from the point
$(x,y,z)$---by moving directly toward the heat source, we increase the
temperature as quickly as possible.
\end{example}

\begin{example} Find the points on the surface defined by $x^2+2y^2+3z^2=1$
where the tangent plane  is parallel to the plane defined by
$3x-y+3z=1$. 

Two planes are parallel if their normals are parallel or
anti-parallel, so we want to find the points on the surface with
normal parallel or anti-parallel to $\langle 3,-1,3\rangle$.
Let $f=x^2+2y^2+3z^2$; the gradient of $f$ is normal to the level
surface at every point, so we are looking for a gradient parallel 
or anti-parallel to
$\langle 3,-1,3\rangle$. The gradient is $\langle 2x,4y,6z\rangle$; if
it is parallel or anti-parallel to $\langle 3,-1,3\rangle$, then
$$\langle 2x,4y,6z\rangle=k\langle 3,-1,3\rangle$$
for some $k$. This means we need a solution to the equations
$$2x=3k\qquad 4y=-k\qquad 6z=3k$$
but this is three equations in four unknowns---we need another
equation. What we haven't used so far is that the points we seek are
on the surface $x^2+2y^2+3z^2=1$; this is the fourth equation. If we
solve the first three equations for $x$, $y$, and $z$ and substitute
into the fourth equation we get
$$\eqalign{
1&=\left({3k\over2}\right)^2+2\left({-k\over4}\right)^2+3\left({3k\over6}\right)^2 \\
&=\left({9\over4}+{2\over16}+{3\over4}\right)k^2 \\
&={25\over8}k^2 \\
}$$
so $\ds k=\pm{2\sqrt2\over 5}$. The desired points are
$\ds\left({3\sqrt2\over5},-{\sqrt2\over10},{\sqrt2\over 5}\right)$ 
and $\ds\left(-{3\sqrt2\over5},{\sqrt2\over10},-{\sqrt2\over 5}\right)$.
You can see the surface and all three planes in the Java 
\expandafter\url\expandafter{\liveurl three_planes.html}%
applet.\endurl
\end{example}

\begin{exercises}

\begin{exercise} Find $D_{\bf u} f$ for 
$\ds f=x^2+xy+y^2$ in the direction of ${\bf u}=\langle 2,1\rangle$ 
at the point $(1,1)$.
\begin{answer} $9\sqrt5/5$
\end{answer}\end{exercise}

\begin{exercise} Find $D_{\bf u} f$ for 
$\ds f=\sin(xy)$ in the direction of ${\bf u}=\langle -1,1\rangle$ 
at the point $(3,1)$.
\begin{answer} $\sqrt2\cos3$
\end{answer}\end{exercise}

\begin{exercise} Find $D_{\bf u} f$ for 
$\ds f=e^x\cos(y)$ in the direction 30 degrees from the positive $x$ axis
at the point $(1,\pi/4)$.
\begin{answer} $e\sqrt2(\sqrt3-1)/4$
\end{answer}\end{exercise}

\begin{exercise} The temperature of a thin plate in the $x$-$y$ plane is 
$\ds T=x^2+y^2$. How fast does temperature change at the point $(1,5)$
moving in a direction 30 degrees from the positive $x$ axis?
\begin{answer} $\sqrt3+5$
\end{answer}\end{exercise}

\begin{exercise} Suppose the density of a thin plate at $(x,y)$ is
$\ds 1/\sqrt{x^2+y^2+1}$. Find the rate of change of the density at
$(2,1)$ in a direction $\pi/3$ radians from the positive $x$ axis.
\begin{answer} $-\sqrt6(2+\sqrt3)/72$
\end{answer}\end{exercise}

\begin{exercise} Suppose the electric potential at $(x,y)$ is
$\ds\ln\sqrt{x^2+y^2}$. Find the rate of change of the potential at
$(3,4)$ toward the origin and also in a direction at a right angle to
the direction toward the origin.
\begin{answer} $-1/5$, $0$
\end{answer}\end{exercise}

\begin{exercise} A plane perpendicular to the $x$-$y$ plane contains the
point $(2,1,8)$ on the paraboloid $z=x^2+4y^2$. The cross-section of
the paraboloid created by this plane has slope 0 at this point. Find
an equation of the plane.
\begin{answer} $4(x-2)+8(y-1)=0$
\end{answer}\end{exercise}

\begin{exercise} A plane perpendicular to the $x$-$y$ plane contains the
point $(3,2,2)$ on the paraboloid $36z=4x^2+9y^2$. The cross-section
of the paraboloid created by this plane has slope 0 at this point.
Find an equation of the plane.
\begin{answer} $2(x-3)+3(y-2)=0$
\end{answer}\end{exercise}

\begin{exercise} Suppose the temperature at $(x,y,z)$ is given by 
$\ds T=xy+\sin(yz)$. In what direction should you go from the point 
$(1,1,1)$ to decrease the temperature as quickly as possible? What is
the rate of change of temperature in this direction?
\begin{answer} $\langle -1,-1-\cos1,-\cos1\rangle$, $-\sqrt{2+2\cos1+2\cos^21}$
\end{answer}\end{exercise}

\begin{exercise} Suppose the temperature at $(x,y,z)$ is given by 
$\ds T=xyz$. In what direction can you go from the point 
$(1,1,1)$ to maintain the same temperature?
\begin{answer} Any direction perpendicular to $\nabla T=\langle
1,1,1\rangle$,
for example, $\langle -1,1,0\rangle$
\end{answer}\end{exercise}

\begin{exercise} Find an equation for the plane tangent to 
$\ds x^2-3y^2+z^2=7$ at $(1,1,3)$.
\begin{answer} $2(x-1)-6(y-1)+6(z-3)=0$
\end{answer}\end{exercise}

\begin{exercise} Find an equation for the plane tangent to 
$\ds xyz=6$ at $(1,2,3)$.
\begin{answer} $6(x-1)+3(y-2)+2(z-3)=0$
\end{answer}\end{exercise}

\begin{exercise} Find an equation for the line normal to 
$\ds x^2+2y^2+4z^2=26 $ at $(2,-3,-1)$.
\begin{answer} $\langle 2+4t,-3-12t,-1-8t\rangle$
\end{answer}\end{exercise}

\begin{exercise} Find an equation for the line normal to 
$\ds x^2+y^2+9z^2=56$ at $(4,2,-2)$.
\begin{answer} $\langle 4+8t,2+4t,-2-36t\rangle$
\end{answer}\end{exercise}

\begin{exercise} Find an equation for the line normal to 
$\ds x^2+5y^2-z^2=0$ at $(4,2,6)$.
\begin{answer} $\langle 4+8t,2+20t,6-12t\rangle$
\end{answer}\end{exercise}

\begin{exercise} Find the directions in which the directional derivative of
$f(x,y)=x^2+\sin(xy)$ at the point $(1,0)$ has the value 1.
\begin{answer} $\langle 0,1\rangle$, $\langle 4/5,-3/5\rangle$
\end{answer}\end{exercise}

\begin{exercise} Show that the curve 
${\bf r}(t) = \langle\ln(t),t\ln(t),t\rangle$
is tangent to the surface $xz^2-yz+\cos(xy) = 1$ at the point
$(0,0,1)$.

\begin{exercise} A bug is crawling on the surface of a hot plate, the
temperature of which at the point $x$ units to the right of the lower
left corner and $y$ units up from the lower left corner is given by
$T(x,y)=100-x^2-3y^3$.

\begin{itemize} % BADBAD

\item{a.} If the bug is at the point $(2,1)$, in what direction should it
  move to cool off the fastest?  How fast will the temperature drop in
  this direction?

\item{b.} If the bug is at the point $(1,3)$, in what direction should it
  move in order to maintain its temperature?

\end{itemize}
\begin{answer} (a) $\langle 4,9\rangle$ (b) $\langle -81,2\rangle$ or
$\langle 81,-2\rangle$
\end{answer}\end{exercise}

% Albert 
\begin{exercise} The elevation on a portion of a hill is given by $f(x,y) =
100 -4x^2 - 2y$.  From the location above $(2,1)$, in which direction will
water run?
\begin{answer} in the direction of $\langle 8,1\rangle$
\end{answer}\end{exercise}
%/Albert

\begin{exercise} Suppose that $g(x,y)=y-x^2$.  Find the gradient at the point
$(-1, 3)$.  Sketch the level curve to the graph of $g$ when
$g(x,y)=2$, and plot both the tangent line and the gradient vector at
the point $(-1,3)$. (Make your sketch large).  What do you notice,
geometrically?
\begin{answer} $\ds \nabla g(-1,3)=\langle 2,1\rangle$
\end{answer}\end{exercise}

\begin{exercise} The gradient $\nabla f$ is a vector
valued function of two variables.  Prove the following gradient rules.
Assume $f(x,y)$ and $g(x,y)$ are differentiable functions.

\begin{itemize} % BADBAD
\item{a.} $\nabla(fg)=f\nabla(g)+g\nabla(f)$
\item{b.} $\nabla(f/g)=(g\nabla f - f \nabla g)/g^2$
\item{c.} $\nabla((f(x,y))^n)=nf(x,y)^{n-1}\nabla f$
\end{itemize}

\end{exercises}
