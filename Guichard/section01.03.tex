\section{Functions}{}{}
\label{sec:functions}

A {\dfont function}\index{function} $y=f(x)$ is a rule for determining
$y$ when we're given a value of $x$.  For example, the rule
$y=f(x)=2x+1$ is a function.  Any line $y=mx+b$ is called a {\dfont
linear} function\index{function!linear}.  The graph of a function
looks like a curve above (or below) the $x$-axis, where for any value
of $x$ the rule $y=f(x)$ tells us how far to go above (or below) the
$x$-axis to reach the curve.

Functions can be defined in various ways: by an algebraic formula or several
algebraic formulas, by
a graph, or by an experimentally determined table of values.  (In the latter
case, the table gives a bunch of points in the plane, which we might then
interpolate with a smooth curve, if that makes sense.)

Given a value of $x$, a function must give
at most one value of $y$.  Thus, vertical lines are not functions.  For
example, the line $x=1$ has infinitely many values of $y$ if $x=1$. It
is also true that 
if $x$ is any number not 1 there is no $y$ which corresponds to $x$,
but that is not a problem---only multiple $y$ values is a problem.

In addition to lines, another familiar example of a  
function is the parabola $y=f(x)=x^2$.  We can draw the graph of this
function by taking various values of $x$ (say, at regular intervals) and
plotting the points $(x,f(x))=(x,x^2)$.  Then connect the points with a
smooth curve.  (See figure~\xrefn{fig:some graphs}.)

The two examples $y=f(x)=2x+1$ and $y=f(x)=x^2$ are both functions which
can be evaluated at {\it any} value of $x$ from negative infinity to
positive infinity.  For many functions, however, it only makes sense to
take $x$ in some interval or outside of some ``forbidden'' region.  The
interval of $x$-values at which we're allowed to evaluate the function is
called the {\dfont domain\index{domain}} of the function.

% BADBAD
% \figure
% \hbox to \hsize{\vbox{\beginpicture
% \normalgraphs
% \ninepoint
% \setcoordinatesystem units <0.7truecm,0.7truecm>
% \setplotarea x from -3 to 3, y from -3 to 3
% \axis left shiftedto x=0 /
% \axis bottom shiftedto y=0 /
% \setquadratic
% \plot -2.5 3 0 0 2.5 3 /
% \put {$y=f(x)=x^2$} [t] <0pt,-5pt> at 0 -3
% \endpicture}\hfill
% \vbox{\beginpicture
% \normalgraphs
% \ninepoint
% \setcoordinatesystem units <0.7truecm,0.7truecm>
% \setplotarea x from -3 to 3, y from -3 to 3
% \axis left shiftedto x=0 /
% \axis bottom shiftedto y=0 /
% \setquadratic
% \plot 0 0 1 1 3 1.732 /
% \put {$y=f(x)=\sqrt{x}$} [t] <0pt,-5pt> at 0 -3
% \endpicture}\hfill
% \vbox{\beginpicture
% \normalgraphs
% \ninepoint
% \setcoordinatesystem units <0.7truecm,0.7truecm>
% \setplotarea x from -3 to 3, y from -3 to 3
% \axis left shiftedto x=0 /
% \axis bottom shiftedto y=0 /
% \setquadratic
% \plot -3.000 -0.333 -2.867 -0.349 -2.733 -0.366 -2.600 -0.385 -2.467 -0.405 
% -2.333 -0.429 -2.200 -0.455 -2.067 -0.484 -1.933 -0.517 -1.800 -0.556 
% -1.667 -0.600 -1.533 -0.652 -1.400 -0.714 -1.267 -0.789 -1.133 -0.882 
% -1.000 -1.000 -0.867 -1.154 -0.733 -1.364 -0.600 -1.667 -0.467 -2.143 
% -0.333 -3.000 /
% \plot 0.333 3.000 0.467 2.143 0.600 1.667 0.733 1.364 0.867 1.154 
% 1.000 1.000 1.133 0.882 1.267 0.789 1.400 0.714 1.533 0.652 
% 1.667 0.600 1.800 0.556 1.933 0.517 2.067 0.484 2.200 0.455 
% 2.333 0.429 2.467 0.405 2.600 0.385 2.733 0.366 2.867 0.349 
% 3.000 0.333 /
% \put {$y=f(x)=1/x$} [t] <0pt,-5pt> at 0 -3
% \endpicture}}
% \figrdef{fig:some graphs}
% \endfigure{Some graphs.}

For example, the square-root function $y=f(x)=\sqrt{x}$ is the rule
which says, given an $x$-value, take the nonnegative number whose
square is $x$.  This rule only makes sense if $x$ is positive or zero.
We say that the domain of this function is $x\ge 0$, or more formally
$\{x\in\R\mid x\ge 0\}$.  Alternately, we
can use interval notation, and write that the domain is $[0,\infty)$.
(In interval notation, square brackets mean that the endpoint is
included, and a parenthesis means that the endpoint is not included.)
The fact that the domain of $y=\sqrt{x}$ is $[0,\infty)$ means that in the
graph of this function ((see figure~\xrefn{fig:some graphs})
we have points $(x,y)$ only above $x$-values on the right side of the
$x$-axis.


Another example of a function whose domain is not the entire $x$-axis
is: $y=f(x)=1/x$, the reciprocal function.  We cannot substitute $x=0$
in this formula.  The function makes sense, however, for any nonzero
$x$, so we take the domain to be: $\{x\in\R\mid x\ne 0\}$.  The graph
of this function does not have any point $(x,y)$ with $x=0$.  As $x$
gets close to 0 from either side, the graph goes off toward infinity.
We call the vertical line $x=0$ an {\dfont
  asymptote\index{asymptote}}.

To summarize, two reasons why certain $x$-values are excluded from the
domain of a function are that (i) we cannot divide by zero, and (ii)
we cannot take the square root of a negative number. We will
encounter some other ways in which functions might be undefined later.

Another reason why the domain of a function might be restricted is
that in a given situation the $x$-values outside of some range might
have no practical meaning.  For example, if $y$ is the area of a
square of side $x$, then we can write $y=f(x)=x^2$.  In a purely
mathematical context the domain of the function $y=x^2$ is all of
$\R$.  But in the story-problem context of finding areas of squares,
we restrict the domain to positive values of $x$, because a square
with negative or zero side makes no sense.

In a problem in pure mathematics, we usually take the domain to be all
values of $x$ at which the formulas can be evaluated.  But in
a story problem there might be further restrictions on the domain
because only certain values of $x$ are of interest or make practical
sense.

In a story problem, often letters different from $x$ and $y$ are used.
For example, the volume $V$ of a sphere is a function of the radius
$r$, given by the formula $V=f(r)=\frac4/3\pi r^3$.
Also, letters different from $f$ may be used.  For example, if $y$ is
the velocity of something at time $t$, we may write $y=v(t)$ with
the letter $v$ (instead of $f$) standing for the velocity function (and
$t$ playing the role of $x$).

The letter playing the role of $x$ is called the {\dfont independent
variable}, and the letter playing the role of $y$ is called the
{\dfont dependent variable\index{dependent variable}} (because its
value ``depends on'' the value of the independent
variable\index{independent variable}).  In story problems, when one
has to translate from English into mathematics, a crucial step is to
determine what letters stand for variables.  If only words and no
letters are given, then we have to decide which letters to use.  Some
letters are traditional.  For example, almost always, $t$ stands for
time.

\begin{example} An open-top box is made from an $a\times b$ rectangular piece of
 cardboard by cutting out a square of side $x$ from each of the four
 corners, and then folding the sides up and sealing them with duct
 tape.  Find a formula for the volume $V$ of the box as a function of
 $x$, and find the domain of this function.


The box we get will have height $x$ and rectangular base of
dimensions $a-2x$ by $b-2x$.  Thus, 
$$
     V=f(x)=x(a-2x)(b-2x).
$$
Here $a$ and $b$ are constants, and $V$ is the variable that depends
on $x$, i.e., $V$ is playing the role of $y$.  

This formula makes mathematical sense for any $x$, but in the story
problem the domain is much less.  In the first place, $x$ must be
positive.  In the second place, it must be less than half the length
of either of the sides of the cardboard.  Thus, the domain is
$$
 \{x\in\R\mid 0<x<{1\over2}(\hbox{minimum~of~$a$~and~$b$})\}.
$$
In interval notation we write: the domain is the interval
$(0,\min(a,b)/2)$. (You might think about whether we could allow 0 or 
(minimum~of~$a$~and~$b$) to be in the domain. They make a certain
physical sense, though we normally would not call the result a box. If we
were to allow these values, what would the corresponding volumes be?
Does that make sense?)
\end{example}

\begin{example} (Circle of radius $r$ centered at the origin) The equation for
this circle is usually given in the form $x^2+y^2=r^2$.  To write the
equation in the form $y=f(x)$ we solve for $y$, obtaining
$y=\pm\sqrt{r^2-x^2}$.  But {\it this is not a function}, because when
we substitute a value in $(-r,r)$ for $x$ there are two corresponding
values of $y$.
To get a function, we must choose one of the two signs in front of the
square root.  If we choose the positive sign, for example, we get the
upper semicircle $y=f(x)=\sqrt{r^2- x^2}$ (see figure~\xrefn{fig:upper
semicircle}).  The domain of this function is the interval $[-r,r]$,
i.e., $x$ must be between $-r$ and $r$ (including the endpoints).  If
$x$ is outside of that interval, then $r^2-x^2$ is negative, and we
cannot take the square root.  In terms of the graph, this just means
that there are no points on the curve whose $x$-coordinate is greater
than $r$ or less than $-r$.
\end{example}

% BADBAD
% \figure
% \vbox{\beginpicture
% \normalgraphs
% \ninepoint
% \setcoordinatesystem units <0.5truein,0.5truein>
% \setplotarea x from -3.1 to 3.1, y from 0 to 3.1
% \axis bottom ticks withvalues {$-r$} {$r$} / at -3 3 / /
% \axis left shiftedto x=0 /
% \circulararc 180 degrees from 3 0 center at 0 0
% \endpicture}
% \figrdef{fig:upper semicircle}
% \endfigure{Upper semicircle $y=\sqrt{r^2- x^2}$}

\begin{example}
Find the domain of 
$$
 y=f(x)={1\over\sqrt{4x-x^2}}.
$$
To answer this question, we must rule out the $x$-values that make
$4x-x^2$ negative (because we cannot take the square root of a
negative number)
and also the $x$-values that make $4x-x^2$ zero (because if $4x-x^2=0$, then
when we take the square root we get 0, and we cannot divide by 0).
In other words, the domain consists of all $x$ for which $4x-x^2$ is
strictly positive.  We give two different methods to find out when  $
4x-x^2>0$.

{\it First method.} Factor $4x-x^2$ as $x(4-x)$.  The product of two numbers
is positive when either both are positive or both are negative, i.e., if
either $x>0$ and $4-x>0$, or else $x<0$ and $4-x<0$.  The latter alternative
is impossible, since if $x$ is negative, then $4-x$ is greater than 4, and
so cannot be negative.  As for the first alternative, the condition $4-x>0$
can be rewritten (adding $x$ to both sides) as $4>x$, so we need:
$x>0$ and $4>x$ (this is sometimes combined in the form $4>x>0$, or,
equivalently, $0<x<4$).  In interval notation, this says that the domain
is the interval $(0,4)$.

{\it Second method.}  Write $4x-x^2$ as $-(x^2-4x)$, and then complete
the square, obtaining $-\Bigl((x-2)^2-4\Bigr)=4-(x-2)^2$.  For this
to be positive we need $(x-2)^2<4$, which means that $x-2$ must be less
than 2 and greater than $-2$:  $-2<x-2<2$.  Adding 2 to everything gives
$0<x<4$.  Both of these methods are equally correct; you may use either
in a problem of this type.
\end{example}

A function does not always have to be given by a single formula, as we
have already seen (in the income tax problem, for example).  
Suppose that $y=v(t)$ is the velocity function for a car
which starts out from rest (zero velocity) at time $t=0$; then
increases its speed steadily to 20 m/sec, taking 10 seconds to do
this; then travels at constant speed 20 m/sec for 15 seconds; and
finally applies the brakes to decrease speed steadily to 0, taking 5
seconds to do this.  The formula for $y=v(t)$ is different in each of
the three time intervals: first $y=2x$, then $y=20$, then $y=-4x+120$.
The graph of this function is shown in figure~\xrefn{fig:piecewise
velocity}.

% BADBAD
% \figure
% \vbox{\beginpicture
% \normalgraphs
% \ninepoint
% \setcoordinatesystem units <4truemm,2truemm>
% \setplotarea x from 0 to 31, y from 0 to 22
% \axis bottom ticks withvalues {$10$} {$25$} {$30$} / at 10 25 30 / /
% \axis left ticks numbered from 0 to 20 by 10 /
% \plot 0 0 10 20 25 20 30 0 /
% \put {$t$} [l] <3pt,0pt> at 31 0
% \put {$v$} [b] <0pt,3pt> at 0 22
% \endpicture}
% \figrdef{fig:piecewise velocity}
% \endfigure{A velocity function.}

Not all functions are given by formulas at all.  A function can be
given by an experimentally determined table of values, or by a
description other than a formula.  For example, the population $y$ of
the U.S. is a function of the time $t$: we can write $y=f(t)$.  This
is a perfectly good function---we could graph it (up to the
present) if we had data for various $t$---but we can't find an
algebraic formula for it.

\begin{exercises}

Find the domain of each of the following functions:

\begin{exercise} $\ds y=f(x)=\sqrt{2x-3}$
\begin{answer} $\ds \{x\mid x\ge 3/2\}$
\end{answer}\end{exercise}
\begin{exercise} $\ds y=f(x)=1/(x+1)$
\begin{answer} $\ds \{x\mid x\not=-1\}$
\end{answer}\end{exercise}
\begin{exercise} $\ds y=f(x)=1/(x^2-1)$
\begin{answer} $\ds \{x\mid x\not=1 \hbox{ and } x\not=-1\}$
\end{answer}\end{exercise}
\begin{exercise} $\ds y=f(x)=\sqrt{-1/x}$
\begin{answer} $\ds \{x\mid x<0\}$
\end{answer}\end{exercise}
\begin{exercise} $\ds y=f(x)={\root 3 \of x}$
\begin{answer} $\ds \{x\mid x\in \R\}$, i.e., all $x$
\end{answer}\end{exercise}
\begin{exercise} $\ds y=f(x)={\root 4 \of x}$
\begin{answer} $\ds \{x\mid x\ge0\}$
\end{answer}\end{exercise}
\begin{exercise} $\ds y=f(x)=\sqrt{r^2-(x-h)^2\ }$, where
$r$ and $h$ are positive constants.
\begin{answer} $\ds \{x\mid h-r\le x\le h+r\}$
\end{answer}\end{exercise}
\begin{exercise} $\ds y=f(x)=\sqrt{1-(1/x)}$
\begin{answer} $\ds \{x\mid x\ge 1\}$
\end{answer}\end{exercise}
\begin{exercise} $\ds y=f(x)=1/\sqrt{1-(3x)^2}$
\begin{answer} $\ds \{x\mid -1/3< x< 1/3\}$
\end{answer}\end{exercise}
\begin{exercise} $\ds y=f(x)=\sqrt{x}+1/(x-1)$
\begin{answer} $\ds \{x\mid x\ge0  \hbox{ and } x\not=1\}$
\end{answer}\end{exercise}
\begin{exercise} $\ds y=f(x)=1/(\sqrt{x}-1)$
\begin{answer} $\ds \{x\mid x\ge0  \hbox{ and } x\not=1\}$
\end{answer}\end{exercise}

\begin{exercise} Find the domain of $\ds h(x) = \begin{cases}
(x^2-9)/(x-3)& x\neq 3 \\
 6 \mbox{ if $x=3$.} \end{cases}$
\begin{answer} $\R$
\end{answer}\end{exercise}

\begin{exercise} Suppose $f(x) = 3x-9$ and $\ds g(x) = \sqrt{x}$.  What is the
domain of the composition $(g\circ f)(x)$?  (Recall that 
{\dfont composition\index{function composition}%
\index{composition of functions}\/}
is defined as $(g\circ f)(x) = g(f(x))$.)  What is the
domain of $(f\circ g)(x)$?
\begin{answer} $\ds \{x\mid x\ge3\}$, $\{x\mid x\ge0\}$
\end{answer}\end{exercise}

\begin{exercise} A farmer wants to build a fence along a river.  He has
500 feet of fencing and wants to enclose a rectangular pen on three
sides (with the river providing the fourth side).  If $x$ is the
length of the side perpendicular to the river, determine the area of
the pen as a function of $x$.  What is the domain of this function?
\begin{answer} $A=x(500-2x)$, $\ds \{x\mid 0\le x\le 250\}$
\end{answer}\end{exercise}

\begin{exercise} A can in the shape of a cylinder is to be made with a total
of 100 square centimeters of material in the side, top, and bottom;
the manufacturer wants the can to hold the maximum possible
volume. Write the volume as a function of the radius $r$ of the can;
find the domain of the function.
\begin{answer} $\ds V=r(50-\pi r^2)$, $\ds \{r\mid 0< r\le \sqrt{50/\pi}\}$
\end{answer}\end{exercise}

\begin{exercise} A can in the shape of a cylinder is to be made to hold a
volume of one liter (1000 cubic centimeters). The manufacturer wants
to use the least possible material for the can. Write the surface area
of the can (total of the top, bottom, and side) as a function of the
radius $r$ of the can; find the domain of the function.
\begin{answer} $\ds A=2\pi r^2+2000/r$, $\ds \{r\mid 0<r<\infty\}$
\end{answer}\end{exercise}

\end{exercises}

