\section{Exponential and Logarithmic functions}{}{}
\nobreak
An exponential function\index{exponential function} has the form
$a^x$, where $a$ is a constant; examples are $\ds 2^x$, $\ds 10^x$, $\ds e^x$. The
logarithmic functions\index{logarithmic function}\index{logarithm} are
the {\dfont inverses\index{inverse function}\/} of the exponential
functions, that is, functions that ``undo'' the exponential functions,
just as, for example, the cube root function ``undoes'' the cube
function: $\ds \root3\of{2^3}=2$. Note that the original function also
undoes the inverse function: $\ds (\root3\of{8})^3=8$.

Let $\ds f(x)=2^x$. The inverse of this function is called the
logarithm base 2, denoted $\ds \log_2(x)$ or (especially in computer
science circles) $\lg(x)$. What does this really mean? The logarithm
must undo the action of the exponential function, so for example it
must be that $\ds \lg(2^3)=3$---starting with 3, the exponential function
produces $\ds 2^3=8$, and the logarithm of 8 must get us back to 3. A
little thought shows that it is not a coincidence that $\ds \lg(2^3)$
simply gives the exponent---the exponent {\it is\/} the original value
that we must get back to. In other words, {\it the logarithm is the
  exponent.} Remember this catchphrase, and what it means, and you
won't go wrong. (You {\it do\/} have to remember what it means. Like
any good mnemonic, ``the logarithm is the exponent'' leaves out a lot
of detail, like ``Which exponent?'' and ``Exponent of what?'')

\begin{example}
What is the value of $\ds \log_{10}(1000)$? The ``10'' tells us the
appropriate number to use for the base of the exponential
function. The logarithm is the exponent, so the question is, what
exponent $E$ makes $\ds 10^E=1000$? If we can find such an $E$, then
$\log_{10}(1000)=\log_{10}(10^E)=E$; finding the appropriate exponent
is the same as finding the logarithm. In this case, of course, it is
easy: $E=3$ so $\ds \log_{10}(1000)=3$.
\end{example}

Let's review some laws of exponents and logarithms; let $a$ be a
positive number. Since 
$a^5=a\cdot a\cdot a\cdot a\cdot a$ and 
$a^3=a\cdot a\cdot a$, it's clear that $\ds a^5\cdot a^3 =
a\cdot a\cdot a\cdot a\cdot a\cdot a\cdot a\cdot a = a^8 = a^{5+3}$,
and in general that $\ds a^ma^n=a^{m+n}$. Since ``the logarithm is the
exponent,'' it's no surprise that this translates directly into a fact
about the logarithm function. Here are three facts from the example:
$\log_a(a^5)=5$, $\ds \log_a(a^3)=3$, $\ds \log_a(a^8)=8$. So
$\log_a(a^5a^3)=\log_a(a^8)=8 = 5+3=\log_a(a^5)+\log_a(a^3)$. Now
let's make this a bit more general. Suppose $A$ and $B$ are two
numbers, $\ds A=a^x$, and $\ds B=a^y$. Then
$\log_a(AB)=\log_a(a^xa^y)=\log_a(a^{x+y})=x+y=\log_a(A)+\log_a(B)$.

Now consider $\ds (a^5)^3=a^5\cdot a^5\cdot a^5=a^{5+5+5}=a^{5\cdot 3}=a^{15}$.
Again it's clear that more generally $\ds (a^m)^n=a^{mn}$, and again this
gives us a fact about logarithms. If $\ds A=a^x$ then
$A^y=(a^x)^y=a^{xy}$, so $\ds \log_a(A^y)=xy=y\log_a(A)$---the exponent
can be ``pulled out in front.''

We have cheated a bit in the previous two paragraphs. It is obvious
that $\ds a^5=a\cdot a\cdot a\cdot a\cdot a$ and $\ds a^3=a\cdot a\cdot a$ and
that the rest of the example follows; likewise for the second
example. But when we consider an exponential function $\ds a^x$ we can't
be limited to substituting integers for $x$. What does $\ds a^{2.5}$ or
$a^{-1.3}$ or $\ds a^\pi$ mean? And is it really true that
$a^{2.5}a^{-1.3}=a^{2.5-1.3}$? The answer to the first question is
actually quite difficult, so we will evade it; the answer to the
second question is ``yes.''

We'll evade the full answer to the hard question, but we have to know
something about exponential functions. You need first to understand
that since it's not ``obvious'' what $\ds 2^x$ should mean, we are really
free to make it mean whatever we want, so long as we keep the behavior
that {\it is\/} obvious, namely, when $x$ is a positive integer.
What else do we want to be true about $\ds 2^x$? We
want the properties of the previous two paragraphs to be true for all
exponents: $\ds 2^x2^y=2^{x+y}$ and $\ds (2^x)^y=2^{xy}$.

After the positive integers, the next easiest
number to understand is 0: $\ds 2^0=1$. You have presumably learned this
fact in the past; why is it true?  It
is true precisely because we want $\ds 2^a2^b=2^{a+b}$ to be true about
the function $\ds 2^x$. We need it to be true that $\ds 2^02^x=2^{0+x}=2^x$,
and this only works if $\ds 2^0=1$. The same argument implies that $\ds a^0=1$
for any $a$.

The next
easiest set of numbers to
understand is the negative integers: for example, $\ds 2^{-3}=1/2^3$. 
We know that whatever $\ds 2^{-3}$ means it must be
that $\ds 2^{-3}2^{3}=2^{-3+3}=2^0=1$, which means that $\ds 2^{-3}$ must be
$1/2^3$. In fact, by the same argument, once we know what $\ds 2^x$ means
for some value of $x$, $\ds 2^{-x}$ must be $\ds 1/2^{x}$ and more generally
$a^{-x}=1/a^x$.

Next, consider an exponent $1/q$, where $q$ is a positive integer. We
want it to be true that $\ds (2^x)^y=2^{xy}$, so $\ds
(2^{1/q})^q=2$. This means that $\ds 2^{1/q}$ is a $q$-th root of 2,
$\ds 2^{1/q}=\root q\of{2\ }$. This is all we need to understand that
$2^{p/q}=(2^{1/q})^p=(\root q\of{2\ })^p$ and
$a^{p/q}=(a^{1/q})^p=(\root q\of{a\ })^p$.

What's left is the hard part: what does $\ds 2^x$ mean when $x$ cannot be
written as a fraction, like $\ds x=\sqrt{2\ }$ or $\ds x=\pi$? What we know so
far is how to assign meaning to $\ds 2^x$ whenever $x=p/q$; if we were to
graph this we'd see something like this:
% BADBAD
% $$\vbox{
% \beginpicture
% \normalgraphs
% \sevenpoint
% \setcoordinatesystem units <1truecm,0.2truecm> point at 0 0
% \setplotarea x from -4 to 4, y from 0 to 16
% \axis left shiftedto x=0 /
% \axis bottom shiftedto y=0 /
% \setquadratic
% \plot -4.000 0.062 
% -3.867 0.069 -3.733 0.075 -3.600 0.082 -3.467 0.090 
% -3.333 0.099 -3.200 0.109 -3.067 0.119 -2.933 0.131 -2.800 0.144 
% -2.667 0.157 -2.533 0.173 -2.400 0.189 -2.267 0.208 -2.133 0.228 
% -2.000 0.250 -1.867 0.274 -1.733 0.301 -1.600 0.330 -1.467 0.362 
% -1.333 0.397 -1.200 0.435 -1.067 0.477 -0.933 0.524 -0.800 0.574 
% -0.667 0.630 -0.533 0.691 -0.400 0.758 -0.267 0.831 -0.133 0.912 
% 0.000 1.000 0.133 1.097 0.267 1.203 0.400 1.320 0.533 1.447 
% 0.667 1.587 0.800 1.741 0.933 1.910 1.067 2.095 1.200 2.297 
% 1.333 2.520 1.467 2.764 1.600 3.031 1.733 3.325 1.867 3.647 
% 2.000 4.000 2.133 4.387 2.267 4.812 2.400 5.278 2.533 5.789 
% 2.667 6.350 2.800 6.964 2.933 7.639 3.067 8.378 3.200 9.190 
% 3.333 10.079 3.467 11.055 3.600 12.126 3.733 13.300 3.867 14.588 
% 4.000 16.000 /
% \endpicture}$$
But this is a poor picture, because you can't see that the ``curve''
is really a whole lot of individual points, above the rational numbers
on the $x$-axis. There are really a lot of ``holes'' in the curve,
above $x=\pi$, for example. But (this is the hard part) it is possible
to prove that the holes can be ``filled in'', and that the resulting
function, called $\ds 2^x$, really does have the properties we want,
namely that $\ds 2^x2^y=2^{x+y}$ and $\ds (2^x)^y=2^{xy}$.

\begin{exercises}

\begin{exercise} Expand $\ds\log_{10} ((x+45)^7 (x-2))$.
\end{exercise}

\begin{exercise} Expand $\ds\log_2 {x^3\over 3x-5 +(7/x)}$.
\end{exercise}

\begin{exercise} Write $\ds \log_2 3x + 17 \log_2 (x-2) -
2\log_2 (x^2 + 4x + 1)$ as a single logarithm.
\end{exercise}

\begin{exercise} Solve $\ds \log_2 (1+ \sqrt{x} ) = 6$ for  $x$.
\end{exercise}

\begin{exercise} Solve $\ds 2^{x^2} = 8$ for $x$.
\end{exercise}

\begin{exercise} Solve $\ds \log_2 (\log_3 (x) ) = 1$ for $x$.
\end{exercise}

\end{exercises}
