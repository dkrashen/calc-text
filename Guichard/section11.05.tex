\section{Comparison Tests}{}{}
\nobreak
As we begin to compile a list of convergent and divergent series, new
ones can sometimes be analyzed by comparing them to ones that we
already understand.

\begin{example} Does $\ds\sum_{n=2}^\infty {1\over n^2\ln n}$ converge?

\ssk
The obvious first approach, based on what we know, is the integral test.
Unfortunately, we can't compute the required antiderivative. But
looking at the series, it would appear that it must converge, because
the terms we are adding are smaller than the terms of a $p$-series,
that is,
$${1\over n^2\ln n}<{1\over n^2},$$
when $n\ge3$. Since adding up the terms $\ds 1/n^2$ doesn't get ``too
big'', the new series ``should'' also converge. Let's make this more
precise.

The series $\ds\sum_{n=2}^\infty {1\over n^2\ln n}$ converges if and
only if $\ds\sum_{n=3}^\infty {1\over n^2\ln n}$ converges---all we've
done is dropped the initial term. We know that 
$\ds\sum_{n=3}^\infty {1\over n^2}$ converges. Looking at two typical
partial sums:
$$
  s_n={1\over 3^2\ln 3}+{1\over 4^2\ln 4}+{1\over 5^2\ln 5}+\cdots+
  {1\over n^2\ln n} < {1\over 3^2}+{1\over 4^2}+
  {1\over 5^2}+\cdots+{1\over n^2}=t_n.
$$
Since the $p$-series converges, say to $L$, and since the terms are positive,
$\ds t_n<L$. Since the terms of the new series are positive, the $\ds s_n$
form an increasing sequence and $\ds s_n<t_n<L$ for all $n$. Hence the
sequence $\ds \{s_n\}$ is bounded and so converges.
\end{example}

Sometimes, even when the integral test applies, comparison to a known
series is easier, so it's generally a good idea to think about doing a
comparison before doing the integral test.

\begin{example} Does $\ds\sum_{n=2}^\infty {|\sin n|\over n^2}$ converge?

\ssk
We can't apply the integral test here, because the terms of this
series are not decreasing. Just as in the previous example, however,
$$ {|\sin n|\over n^2}\le {1\over n^2},$$
because $|\sin n|\le 1$. Once again the partial sums are
non-decreasing and bounded above by $\ds \sum 1/n^2=L$, so the new series
converges. 
\end{example}
\label{example:absolute sine over n squared}

Like the integral test, the comparison test can be used to show both
convergence and divergence. In the case of the integral test, a single
calculation will confirm whichever is the case. To use the comparison
test we must first have a good idea as to convergence or divergence
and pick the sequence for comparison accordingly.

\begin{example} Does $\ds\sum_{n=2}^\infty {1\over\sqrt{n^2-3}}$ converge?
\ssk
We observe that the $-3$ should have little effect compared to the
$\ds n^2$ inside the square root, and therefore guess that the terms are
enough like $\ds 1/\sqrt{n^2}=1/n$ that the series should diverge. We
attempt to show this by comparison to the harmonic series. We note
that 
$${1\over\sqrt{n^2-3}} > {1\over\sqrt{n^2}} = {1\over n},$$
so that
$$
  s_n={1\over\sqrt{2^2-3}}+{1\over\sqrt{3^2-3}}+\cdots+
  {1\over\sqrt{n^2-3}} > {1\over 2} + {1\over3}+\cdots+{1\over n}=t_n,
$$
where $\ds t_n$ is 1 less than the corresponding partial sum of the
harmonic series (because we start at $n=2$ instead of $n=1$). Since
$\ds\lim_{n\to\infty}t_n=\infty$, $\ds\lim_{n\to\infty}s_n=\infty$ as
well.
\end{example}

So the general approach is this: If you believe that a new series is
convergent, attempt to find a convergent series whose terms are
larger than the terms of the new series; if you believe that a new
series is divergent, attempt to find a divergent series whose terms
are smaller than the terms of the new series.

\begin{example} Does $\ds\sum_{n=1}^\infty {1\over\sqrt{n^2+3}}$ converge?
\ssk
Just as in the last example, we guess that this is very much like the
harmonic series and so diverges. Unfortunately,
$${1\over\sqrt{n^2+3}} < {1\over n},$$
so we can't compare the series directly to the harmonic series.
A little thought leads us to
$${1\over\sqrt{n^2+3}} > {1\over\sqrt{n^2+3n^2}} = {1\over2n},$$ so if
$\sum 1/(2n)$ diverges then the given series diverges. But since $\sum
1/(2n)=(1/2)\sum 1/n$, theorem~\xrefn{thm:series are linear} implies
that it does indeed diverge.
\end{example}

For reference we summarize the comparison test in a theorem.

\begin{theorem} Suppose that $\ds a_n$ and $\ds b_n$ are non-negative for all $n$ and
that $\ds a_n\le b_n$ when $n\ge N$, for some $N$.

\begin{itemize} % BADBAD
\item{} If $\ds\sum_{n=0}^\infty b_n$ converges, so does 
$\ds\sum_{n=0}^\infty a_n$.
\item{} If $\ds\sum_{n=0}^\infty a_n$ diverges, so does 
$\ds\sum_{n=0}^\infty b_n$.

\end{itemize}
\end{proof}

\begin{exercises}

Determine whether the series converge or diverge.

\twocol

\begin{exercise} $\ds\sum_{n=1}^\infty {1\over 2n^2+3n+5} $
\begin{answer} converges
\end{answer}\end{exercise}

\begin{exercise} $\ds\sum_{n=2}^\infty {1\over 2n^2+3n-5} $
\begin{answer}  converges
\end{answer}\end{exercise}

\begin{exercise} $\ds\sum_{n=1}^\infty {1\over 2n^2-3n-5} $
\begin{answer}  converges
\end{answer}\end{exercise}

\begin{exercise} $\ds\sum_{n=1}^\infty {3n+4\over 2n^2+3n+5} $
\begin{answer} diverges
\end{answer}\end{exercise}

\begin{exercise} $\ds\sum_{n=1}^\infty {3n^2+4\over 2n^2+3n+5} $
\begin{answer} diverges
\end{answer}\end{exercise}

\begin{exercise} $\ds\sum_{n=1}^\infty {\ln n\over n}$
\begin{answer} diverges
\end{answer}\end{exercise}

\begin{exercise} $\ds\sum_{n=1}^\infty {\ln n\over n^3}$
\begin{answer} converges
\end{answer}\end{exercise}

\begin{exercise} $\ds\sum_{n=2}^\infty {1\over \ln n}$
\begin{answer} diverges
\end{answer}\end{exercise}

\begin{exercise} $\ds\sum_{n=1}^\infty {3^n\over 2^n+5^n}$
\begin{answer} converges
\end{answer}\end{exercise}

\begin{exercise} $\ds\sum_{n=1}^\infty {3^n\over 2^n+3^n}$
\begin{answer} diverges
\end{answer}\end{exercise}

\endtwocol

\end{exercises}

