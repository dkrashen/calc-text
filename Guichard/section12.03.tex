%% Better component exercises like the example: into & down roof eg.
\section{The Dot Product}{}{}
\nobreak
Here's a question whose answer turns out to be very useful: Given two
vectors, what is the angle between them?

It may not be immediately clear that the question makes sense, but
it's not hard to turn it into a question that does. Since vectors have
no position, we are as usual free to place vectors wherever we
like. If the two vectors are placed tail-to-tail, there is now a
reasonable interpretation of the question: we seek the measure of the
smallest angle between the two vectors, in the plane in which they lie.
Figure~\xrefn{fig:angle between vectors} illustrates the situation.

\figure
\vbox{\beginpicture
\normalgraphs
\ninepoint
\setcoordinatesystem units <6truemm,6truemm>
\setplotarea x from 0 to 7, y from 0 to 4
\arrow <4pt> [0.35, 1] from 0 0 to 3 4
\arrow <4pt> [0.35, 1] from 0 0 to 7 3
\setdashes
\plot 3 4 7 3 /
\put {\bf A} [b] <0pt,3pt> at 3 4
\put {\bf B} [l] <3pt,0pt> at 7 3
\put {$\theta$} [bl] <11pt,8pt> at 0 0
\endpicture}
\figrdef{fig:angle between vectors}
\endfigure{The angle between vectors {\bf A} and {\bf B}.}

Since the angle $\theta$ lies in a triangle, we can compute it using a
bit of trigonometry, namely, the law of cosines\index{law of
cosines}\index{cosines!law of}. The lengths of the sides of the
triangle in figure~\xrefn{fig:angle between vectors} are $|{\bf A}|$,
$|{\bf B}|$, and $|{\bf A}-{\bf B}|$. Let $\ds {\bf A}=\langle
a_1,a_2,a_3\rangle$ and $\ds {\bf B}=\langle b_1,b_2,b_3\rangle$; then
$$\eqalign{
  |{\bf A}-{\bf B}|^2&=|{\bf A}|^2+|{\bf B}|^2-2|{\bf A}||{\bf B}|\cos\theta \\
  2|{\bf A}||{\bf B}|\cos\theta&=|{\bf A}|^2+|{\bf B}|^2-|{\bf A}-{\bf B}|^2 \\
  &=a_1^2+a_2^2+a_3^2+b_1^2+b_2^2+b_3^2-(a_1-b_1)^2-(a_2-b_2)^2-(a_3-b_3)^2 \\
  &=a_1^2+a_2^2+a_3^2+b_1^2+b_2^2+b_3^2 \\
  &\qquad-(a_1^2-2a_1b_1+b_1^2)
  -(a_2^2-2a_2b_2+b_2^2)-(a_3^2-2a_3b_3+b_3^2) \\
  &=2a_1b_1+2a_2b_2+2a_3b_3 \\
  |{\bf A}||{\bf B}|\cos\theta&=a_1b_1+a_2b_2+a_3b_3 \\
  \cos\theta&=(a_1b_1+a_2b_2+a_3b_3)/(|{\bf A}||{\bf B}|) \\
}$$
So a bit of simple arithmetic with the coordinates of {\bf A} and {\bf
B} allows us to compute the cosine of the angle between them. If
necessary we can use the arccosine to get $\theta$, but in many
problems $\cos\theta$ turns out to be all we really need.

The numerator of the fraction that gives us $\cos\theta$ turns up a
lot, so we give it a name and more compact notation: we call it
the {\dfont dot  product\index{dot product}\/}, and write it as
$${\bf A}\cdot{\bf B} = a_1b_1+a_2b_2+a_3b_3.$$
This is the same symbol we use for ordinary multiplication, but there
should never be any confusion; you can tell from context whether we
are ``multiplying'' vectors or numbers. (We might also use the dot for
scalar multiplication: $a\cdot{\bf V}=a{\bf V}$; again, it is clear
what is meant from context.)

\begin{example} Find the angle between the vectors ${\bf A}=\langle 1,2,1\rangle$ and
${\bf B}=\langle 3,1,-5\rangle$. We know that
$\cos\theta={\bf A}\cdot{\bf B}/(|{\bf A}||{\bf B}|)=
(1\cdot3 + 2\cdot1 + 1\cdot(-5))/(|{\bf A}||{\bf B}|)=0$, so
$\theta=\pi/2$, that is, the vectors are perpendicular.
\end{example}

\begin{example} Find the angle between the vectors ${\bf A}=\langle 3,3,0\rangle$ and
${\bf B}=\langle 1,0,0\rangle$. We compute
$$\eqalign{
\cos\theta &= (3\cdot1 + 3\cdot0 + 0\cdot0)/(\sqrt{9+9+0}\sqrt{1+0+0}) \\
&= 3/\sqrt{18} = 1/\sqrt2 \\}$$
so $\theta=\pi/4$.
\end{example}

\begin{example} Some special cases are worth looking at: Find the angles
between ${\bf A}$ and ${\bf A}$; ${\bf A}$ and ${\bf -A}$;  ${\bf A}$
and ${\bf 0}=\langle 0,0,0\rangle$.

$\ds \cos\theta= {\bf A}\cdot{\bf A}/(|{\bf A}||{\bf A}|)=(a_1^2+a_2^2+a_3^2)/
(\sqrt{a_1^2+a_2^2+a_3^2}\sqrt{a_1^2+a_2^2+a_3^2})=1$, so the angle
between ${\bf A}$ and itself is zero, which of course is correct.

$\ds\cos\theta={\bf A}\cdot{\bf -A}/(|{\bf A}||{\bf -A}|)=(-a_1^2-a_2^2-a_3^2)/
(\sqrt{a_1^2+a_2^2+a_3^2}\sqrt{a_1^2+a_2^2+a_3^2})=-1$, so the angle
is $\pi$, that is, the vectors point in opposite directions, as of
course we already knew.

$\ds \cos\theta= {\bf A}\cdot{\bf 0}/(|{\bf A}||{\bf 0}|)=(0+0+0)/
(\sqrt{a_1^2+a_2^2+a_3^2}\sqrt{0^2+0^2+0^2})$, which is undefined.
On the other hand, note that since ${\bf A}\cdot{\bf 0}=0$ it looks
at first as if $\cos\theta$ will be zero, which as we have seen means
that vectors are perpendicular; only when we notice that the
denominator is also zero do we run into trouble. One way to ``fix''
this is to adopt the convention that the zero vector ${\bf 0}$ is
perpendicular to all vectors; then we can say in general that if
${\bf A}\cdot{\bf B}=0$, {\bf A} and {\bf B} are perpendicular.
\end{example}

Generalizing the examples, note the following useful facts:
\begin{itemize} % BADBAD
\item{1.} If {\bf A} is parallel or anti-parallel to {\bf B} then
${\bf A}\cdot{\bf B}/(|{\bf A}||{\bf B}|)=\pm1$, and conversely, if
${\bf A}\cdot{\bf B}/(|{\bf A}||{\bf B}|)=1$, {\bf A} and {\bf B}
are parallel, while if ${\bf A}\cdot{\bf B}/(|{\bf A}||{\bf
B}|)=-1$, {\bf A} and {\bf B} are anti-parallel. (Vectors are
parallel\index{vector!parallel}\index{parallel vectors} 
if they point in the same direction,
anti-parallel\index{vector!anti-parallel}\index{anti-parallel vectors} 
if they point in opposite directions.)
\item{2.} If {\bf A} is perpendicular to {\bf B} then 
${\bf A}\cdot{\bf B}/(|{\bf A}||{\bf B}|)=0$, and conversely if 
${\bf A}\cdot{\bf B}/(|{\bf A}||{\bf B}|)=0$ then 
{\bf A} and {\bf B} are perpendicular.
\end{itemize}

Given two vectors, it is often useful to find the {\dfont
projection\index{vector!projection}\index{projection of a vector}\/} 
of one vector onto the other, because this turns out to have important
meaning in many circumstances. More precisely, given ${\bf A}$ and
${\bf B}$, we seek a vector parallel to $\bf B$ but with length
determined by $\bf A$ in a natural way, as shown in
figure~\xrefn{fig:vector projection}. $\bf V$ is chosen so that the
triangle formed by $\bf A$, $\bf V$, and ${\bf A}-{\bf V}$
is a right triangle.

\figure
\vbox{\beginpicture
\normalgraphs
\ninepoint
\setcoordinatesystem units <6truemm,6truemm>
\setplotarea x from 0 to 7, y from 0 to 4
\arrow <4pt> [0.35, 1] from 0 0 to 3 4
\arrow <4pt> [0.35, 1] from 0 0 to 3.98 1.71
\setdashes
\arrow <4pt> [0.35, 1] from 0 0 to 7 3
\plot 3 4 3.98 1.71 /
\put {\bf A} [b] <0pt,3pt> at 3 4
\put {\bf B} [l] <3pt,0pt> at 7 3
\put {\bf V} [tl] <3pt,-3pt> at 3.98 1.71
\put {$\theta$} [bl] <11pt,8pt> at 0 0
\endpicture}
\figrdef{fig:vector projection}
\endfigure{$\bf V$ is the projection of {\bf A} onto {\bf B}.}

Using a little trigonometry, we see that 
$$
  |{\bf V}|=|{\bf A}|\cos\theta= 
  |{\bf A}|{{\bf A}\cdot{\bf B}\over|{\bf A}||{\bf B}|}=
  {{\bf A}\cdot{\bf B}\over|{\bf B}|};
$$
this is sometimes called the 
{\dfont scalar projection of $\bf A$ onto $\bf B$}. To get $\bf V$
itself, we multiply this length by a vector of length one parallel to
$\bf B$: 
$$
  {\bf V}= {{\bf A}\cdot{\bf B}\over|{\bf B}|}{{\bf B}\over|{\bf B}|}=
  {{\bf A}\cdot{\bf B}\over|{\bf B}|^2}{\bf B}.
$$
Be sure that you understand why ${\bf B}/|{\bf B}|$ is a vector of
length one (also called a 
{\dfont unit vector\index{unit vector}\index{vector!unit}\/}) 
parallel to $\bf B$.

The discussion so far implicitly assumed that $0\le\theta\le\pi/2$.
If $\pi/2<\theta\le\pi$, the picture is like 
figure~\xrefn{fig:obtuse vector projection}.
In this case ${\bf A}\cdot {\bf B}$ is negative, so the vector
$${{\bf A}\cdot{\bf B}\over|{\bf B}|^2}{\bf B}$$
is anti-parallel to $\bf B$, and its length is 
$$\left|{{\bf A}\cdot{\bf B}\over|{\bf B}|}\right|.$$
So in general, the scalar projection of $\bf A$ onto $\bf B$
may be positive or negative. If
it is negative, it means that the projection vector is anti-parallel
to $\bf B$ and that the length of the projection vector is the
absolute value of the scalar projection. Of course, you can also
compute the length of the projection vector as usual, by applying the
distance formula to the vector.

\figure
\vbox{\beginpicture
\normalgraphs
\ninepoint
\setcoordinatesystem units <6truemm,6truemm>
\setplotarea x from -4 to 7, y from -2 to 4
\arrow <4pt> [0.35, 1] from 0 0 to -4.966 0.586
\arrow <4pt> [0.35, 1] from 0 0 to -3.98 -1.71
\setdashes
\arrow <4pt> [0.35, 1] from 0 0 to 7 3
\plot -4.966 0.586 -3.98 -1.71 /
\put {\bf A} [b] <0pt,3pt> at -4.966 0.586
\put {\bf B} [l] <3pt,0pt> at 7 3
\put {\bf V} [tl] <3pt,-3pt> at -3.98 -1.71
\put {$\theta$} [b] <0pt,3pt> at 0 0
\endpicture}
\figrdef{fig:obtuse vector projection}
\endfigure{$\bf V$ is the projection of {\bf A} onto {\bf B}.}

Note that the phrase ``projection onto {\bf B}'' is a bit misleading
if taken literally; all that $\bf B$ provides is a direction; the
length of $\bf B$ has no impact on the final vector. In
figure~\xrefn{fig:short projection}, for example, $\bf B$ is shorter than
the projection vector, but this is perfectly acceptable.

\figure
\vbox{\beginpicture
\normalgraphs
\ninepoint
\setcoordinatesystem units <6truemm,6truemm>
\setplotarea x from 0 to 7, y from 0 to 4
\arrow <4pt> [0.35, 1] from 0 0 to 3 4
\setdashes
\arrow <4pt> [0.35, 1] from 0 0 to 2.3333 1
\put {\bf A} [b] <0pt,3pt> at 3 4
\put {\bf B} [t] <0pt,-5pt> at 2.3333 1
\put {$\theta$} [bl] <11pt,8pt> at 0 0
\setcoordinatesystem units <6truemm,6truemm> point at -9 0
\setplotarea x from 0 to 7, y from 0 to 4
\setsolid
\arrow <4pt> [0.35, 1] from 0 0 to 3 4
\arrow <4pt> [0.35, 1] from 0 0 to 3.98 1.71
\setdashes
\arrow <4pt> [0.35, 1] from 0 0 to 2.3333 1
\plot 3 4 3.98 1.71 /
\put {\bf A} [b] <0pt,3pt> at 3 4
\put {\bf B} [t] <0pt,-5pt> at 2.3333 1
\put {\bf V} [tl] <3pt,-3pt> at 3.98 1.71
\put {$\theta$} [bl] <11pt,8pt> at 0 0
\endpicture}
\figrdef{fig:short projection}
\endfigure{$\bf V$ is the projection of {\bf A} onto {\bf B}.}

\begin{example}\relax
\label{example:components of force vector}
Physical force is a vector quantity. It is often necessary to compute
the ``component'' of a force acting in a different direction than the
force is being applied. For example, suppose a ten pound weight is
resting on an inclined plane---a pitched roof, for example. Gravity
exerts a force of ten pounds on the object, directed straight down. It
is useful to think of the component of this force directed down and
parallel to the roof, and the component down and directly into the
roof. These forces are the projections of the force vector onto
vectors parallel and perpendicular to the roof. Suppose the roof is
tilted at a $\ds 30^\circ$ angle, as in figure~\xrefn{fig:components
of force}. A vector parallel to the roof is $\ds
\langle-\sqrt3,-1\rangle$, and a vector perpendicular to the roof is
$\ds \langle 1,-\sqrt3\rangle$.  The force vector is ${\bf F}=\langle
0,-10\rangle$. The component of the force directed down the roof is
then
$$\eqalign{
  {\bf F}_1&={{\bf F}\cdot
  \langle-\sqrt3,-1\rangle\over|\langle-\sqrt3,-1\rangle|^2}
  \langle-\sqrt3,-1\rangle
  ={10\over 2}{\langle-\sqrt3,-1\rangle\over2}=
  \langle -5\sqrt3/2,-5/2\rangle \\
}$$
with length 5.  The component of the force directed into the roof is
$$\eqalign{
  {\bf F}_2&={{\bf F}\cdot
  \langle1,-\sqrt3\rangle\over|\langle1,-\sqrt3\rangle|^2}
  \langle1,-\sqrt3\rangle
={10\sqrt3\over 2}{\langle1,-\sqrt3\rangle\over2}=
\langle 5\sqrt3/2,-15/2\rangle \\
}$$
with length $\ds 5\sqrt3$. Thus, a force of 5 pounds is pulling the object
down the roof, while a force of $\ds 5\sqrt3$ pounds is pulling the object
into the roof.
\end{example}

\figure
\vbox{\beginpicture
\normalgraphs
\ninepoint
\setcoordinatesystem units <2truecm,2truecm>
\setplotarea x from 0 to 3.5, y from 0 to 2
\plot 0 0 3.464 2 /
\setplotsymbol ({\twelvepoint.})
\arrow <4pt> [0.35, 1] from 2.5 1.443 to 2.067 1.193
\arrow <4pt> [0.35, 1] from 2.5 1.443 to 2.5 0.443
\arrow <4pt> [0.35, 1] from 2.5 1.443 to 2.933 0.683
\put {${\bf F}$} [t] <0pt,-5pt> at 2.5 0.443
\put {${\bf F}_2$} [tl] <3pt,-3pt> at 2.933 0.683
\put {${\bf F}_1$} [br] <-3pt,3pt> at 2.067 1.193
\endpicture}
\figrdef{fig:components of force}
\endfigure{Components of a force.}

The dot product has some familiar-looking properties that will be
useful later, so we list them here. These may be proved by writing the
vectors in coordinate form and then performing the indicated
calculations; subsequently it can be easier to use the properties
instead of calculating with coordinates.

\begin{theorem}
If ${\bf u}$, ${\bf v}$, and ${\bf w}$ are vectors and $a$ is a real
number, then
\label{thm:dot product properties}

\begin{itemize} % BADBAD
\item{1.} $\ds {\bf u}\cdot{\bf u} = |{\bf u}|^2$
\item{2.} ${\bf u}\cdot{\bf v} = {\bf v}\cdot{\bf u}$
\item{3.} ${\bf u}\cdot({\bf v}+{\bf w}) = 
{\bf u}\cdot{\bf v}+{\bf u}\cdot{\bf w}$
\item{4.} $(a{\bf u})\cdot{\bf v}=a({\bf u}\cdot{\bf v})
={\bf u}\cdot(a{\bf v})$
\end{proof}

\end{itemize}

\begin{exercises}

\begin{exercise} Find $\langle 1,1,1\rangle\cdot\langle 2,-3,4\rangle$.
\begin{answer} $3$
\end{answer}\end{exercise}

\begin{exercise} Find $\langle 1,2,0\rangle\cdot\langle 0,0,57\rangle$.
\begin{answer} $0$
\end{answer}\end{exercise}

\begin{exercise} Find $\langle 3,2,1\rangle\cdot\langle 0,1,0\rangle$.
\begin{answer} $2$
\end{answer}\end{exercise}

\begin{exercise} Find $\langle -1,-2,5\rangle\cdot\langle 1,0,-1 \rangle$.
\begin{answer} $-6$
\end{answer}\end{exercise}

\begin{exercise} Find $\langle 3,4,6\rangle\cdot\langle 2,3,4\rangle$.
\begin{answer} $42$
\end{answer}\end{exercise}

\begin{exercise} Find the cosine of the angle between $\langle 1,2,3\rangle$
and $\langle 1,1,1\rangle$; use a calculator if necessary to find the angle.
\begin{answer} $\ds \sqrt6/\sqrt7$, $\approx 0.39$
\end{answer}\end{exercise}

\begin{exercise} Find the cosine of the angle between $\langle -1, -2,-3\rangle$
and $\langle 5,0,2\rangle$; use a calculator if necessary to find the angle.
\begin{answer} $\ds -11\sqrt{14}\sqrt{29}/406$, $\approx 2.15$
\end{answer}\end{exercise}

\begin{exercise} Find the cosine of the angle between $\langle 47,100,0\rangle$
and $\langle 0,0,5\rangle$; use a calculator if necessary to find the angle.
\begin{answer} $0$, $\pi/2$
\end{answer}\end{exercise}

\begin{exercise} Find the cosine of the angle between $\langle 1,0,1 \rangle$
and $\langle 0,1,1\rangle$; use a calculator if necessary to find the angle.
\begin{answer} $1/2$, $\pi/3$
\end{answer}\end{exercise}

\begin{exercise} Find the cosine of the angle between $\langle 2,0,0\rangle$
and $\langle -1,1,-1\rangle$; use a calculator if necessary to find the angle.
\begin{answer} $\ds -1/\sqrt3$, $\approx 2.19$
\end{answer}\end{exercise}

\begin{exercise} Find the angle between the diagonal of a cube and one of the
edges adjacent to the diagonal.
\begin{answer} $\ds \arccos(1/\sqrt3)\approx 0.96$
\end{answer}\end{exercise}

\begin{exercise} Find the scalar and vector projections of $\langle 1,2,3\rangle$
onto $\langle 1,2,0\rangle$.
\begin{answer} $\ds \sqrt{5}$, $\langle 1,2,0\rangle$.
\end{answer}\end{exercise}

\begin{exercise} Find the scalar and vector projections of $\langle 1,1,1\rangle$
onto $\langle 3,2,1\rangle$.
\begin{answer} $\ds 3\sqrt{14}/7$, $\langle 9/7,6/7,3/7\rangle$.
\end{answer}\end{exercise}

\begin{exercise} A force of 10 pounds is applied to a wagon, directed at an
angle of $\ds 30^\circ$. Find the component of this force pulling the
wagon straight up, and the component pulling it horizontally along
the ground.
\begin{answer} $\langle 0,5\rangle$, $\ds \langle 5\sqrt3,0\rangle$
\end{answer}\end{exercise}

\figure
\vbox{\beginpicture
\normalgraphs
\ninepoint
\setcoordinatesystem units <5truemm,5truemm>
\setplotarea x from 0 to 11, y from 0 to 4
\putrule from 0 0 to 11 0
\putrule from 6 0.5 to 6 2
\putrule from 6 2 to 2 2
\putrule from 2 2 to 2 0.5
\putrule from 2 0.5 to 6 0.5
\circulararc 360 degrees from 3.5 0.5 center at 3 0.5
\circulararc 360 degrees from 5.5 0.5 center at 5 0.5
\plot 6 1.5 8 2.655 /
\setplotsymbol ({\twelvepoint.})
\arrow <4pt> [0.35, 1] from 8.5 2.943 to 10 3.81
\put {${\bf F}$} [bl] <3pt,3pt> at 10 3.81
\endpicture}
\figrdef{fig:pulling a wagon}
\endfigure{Pulling a wagon.}

\begin{exercise} A force of 15 pounds is applied to a wagon, directed at an
angle of $\ds 45^\circ$. Find the component of this force pulling the
wagon straight up, and the component pulling it horizontally along
the ground.
\begin{answer} $\ds \langle 0,15\sqrt2/2\rangle$,$\langle 15\sqrt2/2,0\rangle$
\end{answer}\end{exercise}

\begin{exercise} Use the dot product to find a non-zero vector ${\bf w}$
perpendicular to both ${\bf u}=\langle 1,2,-3\rangle$ and 
${\bf v}=\langle 2,0,1\rangle$.
\begin{answer} Any vector of the form $\langle a, -7a/2, -2a\rangle$
\end{answer}\end{exercise}

\begin{exercise} Let ${\bf x}=\langle 1,1,0 \rangle$ and ${\bf y}=\langle
2,4,2 \rangle$.  Find a unit vector that is perpendicular to both $\bf
x$ and $\bf y$.
\begin{answer} $\ds \langle 1/\sqrt3,-1/\sqrt3,1/\sqrt3\rangle$
\end{answer}\end{exercise}

\begin{exercise} Do the three points $(1,2,0)$, $(-2,1,1)$, and $(0,3,-1)$
form a right triangle?
\begin{answer} No.
\end{answer}\end{exercise}

\begin{exercise} Do the three points $(1,1,1)$, $(2,3,2)$, and $(5,0,-1)$
form a right triangle?
\begin{answer} Yes.
\end{answer}\end{exercise}

\begin{exercise}
Show that $|{\bf A}\cdot{\bf B}|\le|{\bf A}||{\bf B}|$

\begin{exercise} Let $\bf x$ and $\bf y$ be perpendicular vectors.  Use
Theorem~\xrefn{thm:dot product properties} to prove that $\ds |{\bf
  x}|^2+|{\bf y}|^2=|{\bf x}+{\bf y}|^2$.  What is this result better
known as?

\begin{exercise} Prove that the diagonals of a rhombus intersect at right angles. 

\begin{exercise} Suppose that ${\bf z}=|{\bf x}| {\bf y} + |{\bf y}| {\bf x}$
where $\bf x$, $\bf y$, and $\bf z$ are all nonzero vectors.  Prove
that $\bf z$ bisects the angle between $\bf x$ and $\bf y$.

\begin{exercise}
Prove Theorem~\xrefn{thm:dot product properties}.

\end{exercises}

