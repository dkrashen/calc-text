\section{Maxima and minima}{}{}

Suppose a surface given by $f(x,y)$ has a local maximum at
$(x_0,y_0,z_0)$; geometrically, this point on the surface looks like
the top of a hill. If we look at the cross-section in the plane
$y=y_0$, we will see a local maximum on the curve at $(x_0,z_0)$, and
we know from single-variable calculus that ${\partial z\over\partial x}=0$
at this point. Likewise, in the plane $x=x_0$, ${\partial
  z\over\partial y}=0$. So if there is a local maximum at
$(x_0,y_0,z_0)$, both partial derivatives at the point must be zero,
and likewise for a local minimum. Thus, to find local maximum and
minimum points, we need only consider those points at which both
partial derivatives are 0. As in the single-variable case, it is
possible for the derivatives to be 0 at a point that is neither a
maximum or a minimum, so we need to test these points further.

You will recall that in the single variable case, we examined three
methods to identify maximum and minimum points; the most useful is the
second derivative test, though it does not always work. For functions
of two variables there is also a second derivative test; again it is
by far the most useful test, though it doesn't always work.

\begin{theorem} Suppose that the second partial derivatives of $f(x,y)$ are
continuous near $(x_0,y_0)$, and $f_x(x_0,y_0)=f_y(x_0,y_0)=0$.
We denote by $D$ the {\dfont discriminant\index{discriminant}}
$D(x_0,y_0)=f_{xx}(x_0,y_0)f_{yy}(x_0,y_0)-f_{xy}(x_0,y_0)^2$.
If $D>0$ and $f_{xx}(x_0,y_0)<0$ there is a local maximum at $(x_0,y_0)$;
if $D>0$ and $f_{xx}(x_0,y_0)>0$ there is a local minimum at $(x_0,y_0)$;
if $D<0$ there is neither a maximum nor a minimum at $(x_0,y_0)$;
if $D=0$, the test fails.
\end{theorem}

\begin{example} Verify that $f(x,y)=x^2+y^2$ has a minimum at $(0,0)$.

First, we compute all the needed derivatives:
$$f_x=2x \qquad f_y=2y \qquad f_{xx}=2 \qquad f_{yy}=2 \qquad
f_{xy}=0.$$
The derivatives $f_x$ and $f_y$ are zero only at $(0,0)$. Applying the
second derivative test there:
$$D(0,0)=f_{xx}(0,0)f_{yy}(0,0)-f_{xy}(0,0)^2=
2\cdot2-0=4>0,$$
so there is a local minimum at $(0,0)$, and there are no other
possibilities. 
\end{example}

\begin{example} Find all local maxima and minima for $f(x,y)=x^2-y^2$.

The derivatives:
$$f_x=2x \qquad f_y=-2y \qquad f_{xx}=2 \qquad f_{yy}=-2 \qquad
f_{xy}=0.$$
Again there is a single critical point, at $(0,0)$, and
$$D(0,0)=f_{xx}(0,0)f_{yy}(0,0)-f_{xy}(0,0)^2=
2\cdot-2-0=-4<0,$$
so there is neither a maximum nor minimum there, and so there are no
local maxima or minima. The surface is shown in
figure~\xrefn{fig:saddle}. 
\end{example}

\figure
\vbox{\beginpicture
\normalgraphs
\ninepoint
\setcoordinatesystem units <3truecm,3truecm>
\setplotarea x from -1 to 1, y from 0 to 1
\put {\hbox{\epsfxsize8cm\epsfbox{saddle.eps}}} at 0 -0.15
\endpicture}
\figrdef{fig:saddle}
\endfigure{A saddle point, neither a maximum nor a minimum.
(\expandafter\url\expandafter{\liveurl saddle.html}%
AP\endurl)}

\begin{example} Find all local maxima and minima for $f(x,y)=x^4+y^4$.

The derivatives:
$$f_x=4x^3 \qquad f_y=4y^3 \qquad f_{xx}=12x^2 \qquad f_{yy}=12y^2 \qquad
f_{xy}=0.$$
Again there is a single critical point, at $(0,0)$, and
$$D(0,0)=f_{xx}(0,0)f_{yy}(0,0)-f_{xy}(0,0)^2=
0\cdot0-0=0,$$
so we get no information. However, in this case it is easy to see that
there is a minimum at $(0,0)$, because $f(0,0)=0$ and
at all other points $f(x,y)>0$.
\end{example}

\begin{example} Find all local maxima and minima for $f(x,y)=x^3+y^3$.

The derivatives:
$$f_x=3x^2 \qquad f_y=3y^2 \qquad f_{xx}=6x^2 \qquad f_{yy}=6y^2 \qquad
f_{xy}=0.$$
Again there is a single critical point, at $(0,0)$, and
$$D(0,0)=f_{xx}(0,0)f_{yy}(0,0)-f_{xy}(0,0)^2=
0\cdot0-0=0,$$
so we get no information. In this case, a little thought shows there
is neither a maximum nor a minimum at $(0,0)$: when $x$ and $y$ are
both positive, $f(x,y)>0$, and when $x$ and $y$ are
both negative, $f(x,y)<0$, and there are points of both kinds
arbitrarily close to $(0,0)$. Alternately, if we look at the
cross-section when $y=0$, we get $f(x,0)=x^3$, which does not have
either a maximum or minimum at $x=0$.
\end{example}

\begin{example} Suppose a box with no top is to hold a certain volume $V$. Find
the dimensions for the box that result in the minimum surface area.

The area of the box is $A=2hw+2hl+lw$, and the volume is $V=lwh$, so
we can write the area as a function of two variables,
$$A(l,w)={2V\over l}+{2V\over w}+lw.$$
Then
$$A_l=-{2V\over l^2}+w \hbox{\quad and\quad} A_w=-{2V\over w^2}+l.$$
If we set these equal to zero and solve, we find
$\ds w=(2V)^{1/3}$ and $\ds l=(2V)^{1/3}$, and the corresponding
height is $h=V/(2V)^{2/3}$.

The second derivatives are
$$A_{ll}={4V\over l^3}\qquad A_{ww}={4V\over w^3}\qquad
A_{lw}=1,$$
so the discriminant is
$$D={4V\over l^3}{4V\over w^3}-1=4-1=3>0.$$
Since $A_{ll}$ is 2, there is a local minimum at the critical point.
Is this a global minimum? It is, but it is difficult to see this
analytically; physically and graphically it is clear that there is a
minimum, in which case it must be at the single critical point. 
%\expandafter\url\expandafter{\sageurl 2730}Here\endurl\ 
\expandafter\url\expandafter{\liveurl jmol_surface_minimum}Here\endurl\ 
is the graph as rendered by Sage, as an example. Note that we must
choose a value for $V$ in order to graph it.
\end{example}

Recall that when we did single variable global maximum and minimum
problems, the easiest cases were those for which the variable could be
limited to a finite closed interval, for then we simply had to check
all critical values and the endpoints. The previous example is
difficult because there is no finite boundary to the domain of the
problem---both $w$ and $l$ can be in $(0,\infty)$. As in the single
variable case, the problem is often simpler when there is a finite
boundary. 

\begin{theorem} If $f(x,y)$ is continuous on a closed and bounded subset of
$\R^2$, then it has both a maximum and minimum value.
\end{theorem}

As in the case of single variable functions, this means that the
maximum and minimum values must occur at a critical point or on the
boundary; in the two variable case, however, the boundary is a curve,
not merely two endpoints.

\begin{example} The length of the diagonal of a box is to be 1 meter; find the
maximum possible volume.
\label{exam:box diagonal}

If the box is placed with one corner at the origin, and sides along
the axes, the length of the diagonal is $\ds\sqrt{x^2+y^2+z^2}$, and
the volume is
$$V=xyz=xy\sqrt{1-x^2-y^2}.$$
Clearly, $x^2+y^2\le 1$, so the domain we are interested in
is the quarter of the unit disk in the first quadrant.
Computing derivatives:
$$\eqalign{
V_x&={y-2yx^2-y^3\over\sqrt{1-x^2-y^2}} \\
V_y&={x-2xy^2-x^3\over\sqrt{1-x^2-y^2}} \\
}$$
If these are both 0, then $x=0$ or $y=0$, or $x=y=1/\sqrt3$. The boundary of
the domain is composed of three curves: $x=0$ for $y\in[0,1]$; $y=0$
for $x\in[0,1]$; and $x^2+y^2=1$, where $x\ge0$ and $y\ge0$. In all
three cases, the volume  $xy\sqrt{1-x^2-y^2}$ is 0, so the maximum
occurs at the only critical point $(1/\sqrt3,1/\sqrt3,\sqrt3/9)$. See
figure~\xrefn{fig:max volume}.
\end{example}

\figure
\vbox{\beginpicture
\normalgraphs
\ninepoint
\setcoordinatesystem units <3truecm,3truecm>
\setplotarea x from -1 to 1, y from 0 to 1
\put {\hbox{\epsfxsize8cm\epsfbox{max_volume.eps}}} at 0 0
\endpicture}
\figrdef{fig:max volume}
\endfigure{The volume of a box with fixed length diagonal.}

\begin{exercises}

\begin{exercise} Find all local maximum and minimum points of
$f=x^2+4y^2-2x+8y-1$.
\begin{answer} minimum at $(1,-1)$
\end{answer}\end{exercise}

\begin{exercise} Find all local maximum and minimum points of
$f=x^2-y^2+6x-10y+2$.
\begin{answer} none
\end{answer}\end{exercise}

\begin{exercise} Find all local maximum and minimum points of
$f=xy$.
\begin{answer} none
\end{answer}\end{exercise}

\begin{exercise} Find all local maximum and minimum points of
$f=9+4x-y-2x^2-3y^2$.
\begin{answer} maximum at $(1,-1/6)$
\end{answer}\end{exercise}

\begin{exercise} Find all local maximum and minimum points of
$f=x^2+4xy+y^2-6y+1$.
\begin{answer} none
\end{answer}\end{exercise}

\begin{exercise} Find all local maximum and minimum points of
$f=x^2-xy+2y^2-5x+6y-9$.
\begin{answer} minimum at $(2,-1)$
\end{answer}\end{exercise}

% Albert
\begin{exercise} Find the absolute maximum and minimum points of
$f=x^2+3y-3xy$ over the region bounded by
$y=x$, $y=0$, and $x=2$.
\begin{answer} $f(2,2)=-2$, $f(2,0)=4$
\end{answer}\end{exercise}
%/Albert

\begin{exercise} A six-sided rectangular box is to hold $1/2$ cubic meter;
what shape should the box be to minimize surface area?
\begin{answer} a cube $1/\root 3 \of {2}$ on a side
\end{answer}\end{exercise}

\begin{exercise} The post office will accept packages whose combined length
and girth is at most 130 inches (girth is the maximum distance around
the package perpendicular to the length). What is the largest volume
that can be sent in a rectangular box?
\begin{answer} $65/3\times 65/3\times 130/3$
\end{answer}\end{exercise}

\begin{exercise} The bottom of a rectangular box costs twice as much per unit
area as the sides and top. Find the shape for a given volume that will
minimize cost.
\begin{answer} It has a square base, and is one and one half times as tall as wide.
If the volume is $V$ the dimensions are $\root 3 \of {2V/3}\times
\root 3 \of {2V/3}\times \root 3\of {9V/4}$.
\end{answer}\end{exercise}

\begin{exercise} Using the methods of this section, find the shortest
distance from the origin to the plane $x+y+z=10$.
\begin{answer} $\sqrt{100/3}$
\end{answer}\end{exercise}

\begin{exercise} Using the methods of this section, find the shortest
distance from the point $(x_0,y_0,z_0)$ to the plane $ax+by+cz=d$.
You may assume that $c\not=0$; use of Sage or similar software
is recommended.
\begin{answer} $|ax_0+by_0+cz_0-d|/\sqrt{a^2+b^2+c^2}$
\end{answer}\end{exercise}

\begin{exercise} A trough is to be formed by bending up two sides of a long
metal rectangle
so that the cross-section of the trough is an isosceles trapezoid, as
in figure~\xrefn{fig:trough}. If the width of the metal sheet is 2
meters, how should it be bent to maximize the volume of the trough?
\begin{answer} The sides and bottom should all be $2/3$ meter, and the sides
should be bent up at angle $\pi/3$.
\end{answer}\end{exercise}

\begin{exercise} Given the three points $(1,4)$, $(5,2)$, and $(3,-2)$, 
$\ds(x-1)^2+(y-4)^2+(x-5)^2+(y-2)^2+(x-3)^2+(y+2)^2$
is the sum of the squares of the distances from point $(x,y)$ to the
three points. Find $x$ and $y$ so that this quantity is minimized.
\begin{answer} $(3,4/3)$
\end{answer}\end{exercise}

\begin{exercise} Suppose that $f(x,y)=x^2+y^2+kxy$. Find and classify the
  critical points, and discuss how they change when $k$ takes on
  different values.

\begin{exercise} Find the shortest distance from the point $(0,b)$ to the
  parabola $y=x^2$.
\begin{answer} $|b|$ if $b\le1/2$, otherwise $\ds\sqrt{b-1/4}$
\end{answer}\end{exercise}

\begin{exercise} Find the shortest distance from the point $(0,0,b)$ to the
  paraboloid $z=x^2+y^2$.
\begin{answer} $|b|$ if $b\le1/2$, otherwise $\ds\sqrt{b-1/4}$
\end{answer}\end{exercise}

\begin{exercise} Consider the function $f(x,y)=x^3-3x^2y+y^3$.

\begin{itemize} % BADBAD
\item{a.} Show that $(0,0)$ is the only critical point of $f$.

\item{b.} Show that the discriminant test is inconclusive for $f$.  

\item{c.} Determine the cross-sections of $f$ obtained by setting $y=kx$ for
  various values of $k$.

\item{d.} What kind of critical point is $(0,0)$?

\end{itemize}

\begin{exercise} Find the volume of the largest rectangular box with edges
  parallel to the axes that can be inscribed in the ellipsoid
  $2x^2+72y^2+18z^2=288$.
\begin{answer} $\ds 1024/\sqrt3$
\end{answer}\end{exercise}

\end{exercises}

