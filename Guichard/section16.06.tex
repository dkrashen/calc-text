\section{Vector Functions for Surfaces}{}{}

We have dealt extensively with vector equations for curves, 
${\bf r}(t)=\langle x(t),y(t),z(t)\rangle$. A similar technique
can be used to represent surfaces in a way that is more general than
the equations for surfaces we have used so far. Recall that
when we use ${\bf r}(t)$ to represent a curve, we imagine the vector 
${\bf r}(t)$ with its tail at the origin, and then we follow the head
of the arrow as $t$ changes. The vector ``draws'' the curve through
space as $t$ varies.

Suppose we instead have a vector function of two variables,
$${\bf r}(u,v)=\langle x(u,v),y(u,v),z(u,v)\rangle.$$ As both $u$ and
$v$ vary, we again imagine the vector ${\bf r}(u,v)$ with its tail at
the origin, and its head sweeps out a surface in space. A useful
analogy is the technology of CRT video screens, in which an electron
gun fires electrons in the direction of the screen. The gun's
direction sweeps horizontally and vertically to ``paint'' the screen
with the desired image. In practice, the gun moves horizontally
through an entire line, then moves vertically to the next line and
repeats the operation. In the same way, it can be useful to imagine
fixing a value of $v$ and letting ${\bf r}(u,v)$ sweep out a curve as
$u$ changes. Then $v$ can change a bit, and ${\bf r}(u,v)$ sweeps out
a new curve very close to the first. Put enough of these curves
together and they form a surface.

\begin{example} Consider the function ${\bf r}(u,v)=\langle v\cos u,v\sin u,
v\rangle$.  For a fixed value of $v$, as $u$ varies from 0 to $2\pi$,
this traces a circle of radius $v$ at height $v$ above the
$x$-$y$ plane. Put lots and lots of these together,and they form a
cone, as in figure~\xrefn{fig:parametric cone}.
\end{example}

\figure
\vbox{\beginpicture
\normalgraphs
\ninepoint
\setcoordinatesystem units <3truecm,3truecm>
\setplotarea x from -1 to 1, y from 0 to 1
\put {\hbox{\epsfxsize6cm\epsfbox{cone_level_curves.eps}}} at 0 0
\put {\hbox{\epsfxsize6cm\epsfbox{parametric_cone.eps}}} at 2.5 0
\endpicture}
\figrdef{fig:parametric cone}
\endfigure{Tracing a surface.}

\begin{example} Let ${\bf r}=\langle v\cos u, v\sin u, u\rangle$. If $v$ is
constant, the resulting curve is a helix (as in
figure~\xrefn{fig:helixes}). If $u$ is constant, the resulting curve
is a straight line at height $u$ in the direction $u$ radians from the
positive $x$ axis. Note in figure~\xrefn{fig:helical ramp} how the
helixes and the lines both paint the same surface in a different way.
\end{example}

\figure
\vbox{\beginpicture
\normalgraphs
\ninepoint
\setcoordinatesystem units <2.8truecm,2.8truecm>
\setplotarea x from -1 to 1, y from 0 to 1
\put {\hbox{\epsfxsize4cm\epsfbox{helical_ramp.eps}}} at 0 0
\put {\hbox{\epsfxsize4cm\epsfbox{helical_ramp_2.eps}}} at 2 0
\put {\hbox{\epsfxsize4cm\epsfbox{helical_ramp_3.eps}}} at 4 0
\endpicture}
\figrdef{fig:helical ramp}
\endfigure{Tracing a surface.
(\expandafter\url\expandafter{\liveurl helical_ramp.html}%
AP\endurl)}

This technique allows us to represent many more surfaces than
previously. 

\begin{example} The curve given by 
$${\bf r}=\langle (2+\cos(3u/2))\cos u,
(2+\cos(3u/2))\sin u, \sin(3u/2)\rangle$$
 is called a trefoil knot.
Recall that from the vector equation of the curve we can compute the
unit tangent $\bf T$, the unit normal $\bf N$, and the 
binormal\index{binormal} vector ${\bf B}={\bf T}\times{\bf N}$; you may
  want to review section~\xrefn{sec:arc length 3D}. The binormal 
is
perpendicular to both $\bf T$ and $\bf N$; one way to interpret this
is that ${\bf N}$ and ${\bf B}$ define a plane perpendicular to $\bf
T$, that is, perpendicular to the curve; since ${\bf N}$ and ${\bf B}$
are perpendicular to each other, they can function just as $\bf i$ and
$\bf j$ do for the $x$-$y$ plane. So, for example, ${\bf c}(v)={\bf
  N}\cos v+{\bf B}\sin v$ is a vector equation for a unit circle in a plane
perpendicular to the curve described by $\bf r$, except that the usual
interpretation of $\bf c$ would put its center at the origin. We can
fix that simply by adding $\bf c$ to the original $\bf r$:
let ${\bf f}={\bf r}(u) +{\bf c}(v)$. For a
fixed $u$ this draws a circle around the point ${\bf r}(u)$; as $u$
varies we get a sequence of such circles around the curve $\bf r$,
that is, a tube of radius 1 with $\bf r$ at its center. We can easily
change the radius; for example ${\bf r}(u) +a{\bf c}(v)$
gives the tube radius $a$; we can make the radius vary as we
move along the curve with ${\bf r}(u) +g(u){\bf c}(v)$,
where $g(u)$ is a function of $u$.
As shown in
figure~\xrefn{fig:trefoil}, it is hard to see that the plain knot is
knotted; the tube makes the structure apparent. Of course, there is
nothing special about the trefoil knot in this example; we can put a
tube around (almost) any curve in the same way.
\end{example}

\figure
\vbox{\beginpicture
\normalgraphs
\ninepoint
\setcoordinatesystem units <2.5truecm,2.5truecm>
\setplotarea x from -1 to 1, y from 0 to 1
\put {\hbox{\epsfxsize4.17cm\epsfbox{trefoil.eps}}} at 0 0
\put {\hbox{\epsfxsize5cm\epsfbox{trefoil_tube.eps}}} at 2.5 0
\put {\hbox{\epsfxsize5cm\epsfbox{trefoil_sausage.eps}}} at 5 0
\endpicture}
\figrdef{fig:trefoil}
\endfigure{Tubes around a trefoil knot, with radius $1/2$ and
$3\cos(u)/4$.
(\expandafter\url\expandafter{\liveurl trefoil.html}%
AP\endurl)}

We have previously examined surfaces given in the form 
$f(x,y)$. It is sometimes useful to represent such surfaces in the
more general vector form, which is quite easy:
${\bf r}(u,v)=\langle u,v,f(u,v)\rangle$. The names of the variables
are not important of course; instead of disguising $x$ and $y$, we
could simply write ${\bf r}(x,y)=\langle x,y,f(x,y)\rangle$. 

We have also previously dealt with surfaces that are not functions of
$x$ and $y$; many of these are easy to represent in vector form.
One common type of surface that cannot be represented as $z=f(x,y)$ is
a surface given by an equation involving only $x$ and $y$. For
example,
$x+y=1$ and $y=x^2$ are ``vertical'' surfaces. For every point $(x,y)$
in the plane that satisfies the equation, the point $(x,y,z)$ is on
the surface, for 
every value of $z$. Thus, a corresponding vector form for the surface
is something like $\langle f(u),g(u),v\rangle$; for example, $x+y=1$
becomes $\langle u,1-u,v\rangle$ and $y=x^2$ becomes $\langle
u,u^2,v\rangle$. 

Yet another sort of example is the sphere, say $x^2+y^2+z^2=1$. This
cannot be written in the form $z=f(x,y)$, but it is easy to write in
vector form; indeed this particular surface is much like the cone,
since it has circular cross-sections, or we can think of it as a tube
around a portion of the $z$-axis, with a radius that varies depending
on where along the axis we are. One vector expression for the sphere
is $\langle \sqrt{1-v^2}\cos u,\sqrt{1-v^2}\sin u, v\rangle$---this
emphasizes the tube structure, as it is naturally viewed as drawing a
circle of radius $\sqrt{1-v^2}$ around the $z$-axis at height $v$.
We could also take a cue from spherical coordinates, and write
$\langle \sin u\cos v,\sin u\sin v,\cos u\rangle$, where in effect
$u$ and $v$ are $\phi$ and $\theta$ in disguise. 

It is quite simple in Sage to plot any surface for which you have a
vector representation. Using different vector functions sometimes
gives different looking plots, because Sage in effect draws the
surface by holding one variable constant and then the other. For
example, you might have noticed in figure~\xrefn{fig:helical ramp}
that the curves in the two right-hand graphs are superimposed on the
left-hand graph; the graph of the surface is just the combination of
the two sets of curves, with the spaces filled in with color. 

Here's a simple but striking example: the plane $x+y+z=1$ can be
represented quite naturally as $\langle u,v,1-u-v\rangle$. But we
could also think of painting the same plane by choosing a particular
point on the plane, say $(1,0,0)$, and then drawing circles or
ellipses (or any of a number of other curves) as if that point were
the origin in the plane. For example, $\langle 1-v\cos u-v\sin u,v\sin
u,v\cos u\rangle$ is one such vector function. Note that while it may
not be obvious where this came from, it is quite easy to see that
the sum of the $x$, $y$, and $z$ components of the vector is always
1. Computer renderings of the plane using these two functions are shown
in figure~\xrefn{fig:two plane parameterizations}.

\figure
\vbox{\beginpicture
\normalgraphs
\ninepoint
\setcoordinatesystem units <2.5truecm,2.5truecm>
\setplotarea x from -1 to 1, y from 0 to 1
\put {\hbox{\epsfxsize7cm\epsfbox{plane1.eps}}} at 0 0
\put {\hbox{\epsfxsize5cm\epsfbox{plane2.eps}}} at 3 0
\endpicture}
\figrdef{fig:two plane parameterizations}
\endfigure{Two representations of the same plane.
(\expandafter\url\expandafter{\liveurl two_planes.html}%
AP\endurl)}

Suppose we know that a plane contains a particular point
$(x_0,y_0,z_0)$ and that two vectors ${\bf u}=\langle
u_0,u_1,u_2\rangle$ and ${\bf v}=\langle
v_0,v_1,v_2\rangle$ are parallel to the plane but not to each
other. We know how to get an equation for the plane in the form
$ax+by+cz=d$, by first computing ${\bf u}\times{\bf v}$. It's even
easier to get a vector equation:
$${\bf r}(u,v) = \langle x_0,y_0,z_0\rangle + u{\bf u} + v{\bf v}.$$
The first vector gets to the point $(x_0,y_0,z_0)$ and then by varying
$u$ and $v$, $u{\bf u} + v{\bf v}$ gets to every point in the plane.

Returning to $x+y+z=1$, the points $(1,0,0)$, $(0,1,0)$, and $(0,0,1)$
are all on the plane. By subtracting coordinates we see that
$\langle -1,0,1\rangle$ and $\langle -1,1,0\rangle$ are parallel to
the plane, so a third vector form for this plane is
$$\langle 1,0,0\rangle + u\langle -1,0,1\rangle + v\langle
-1,1,0\rangle
= \langle 1-u-v,v,u\rangle.$$
This is clearly quite similar to the first form we found.

We have already seen (section~\xrefn{sec:surface area 3D})
how to find the area of a surface when it is
defined in the form $f(x,y)$. Finding the area when the surface is
given as a vector function is very similar. Looking at the plots of
surfaces we have just seen, it is evident that the two sets of curves
that fill out the surface divide it into a grid, and that the spaces
in the grid are approximately parallelograms. As before this is the
key: we can write down the area of a typical little parallelogram and
add them all up with an integral.

Suppose we want to approximate the area of the surface ${\bf r}(u,v)$
near ${\bf r}(u_0,v_0)$. The functions ${\bf r}(u,v_0)$ and ${\bf
  r}(u_0,v)$ define two curves that intersect at ${\bf
  r}(u_0,v_0)$. The derivatives of $\bf r$ give us vectors tangent to
these two curves: ${\bf r}_u(u_0,v_0)$ and ${\bf r}_v(u_0,v_0)$, and
then ${\bf r}_u(u_0,v_0)\,du$ and ${\bf r}_v(u_0,v_0)\,dv$
are two small tangent vectors, whose lengths can be used as the
lengths of the sides of an approximating parallelogram. Finally, the
area of this parallelogram is $|{\bf r}_u\times{\bf r}_v|\,du\,dv$ and
so the total surface area is
$$\int_a^b\int_c^d |{\bf r}_u\times{\bf r}_v|\,du\,dv.$$

\begin{example} We find the area of the surface $\langle v\cos u,v\sin
u,u\rangle$ for $0\le u \le \pi$ and  $0\le v\le 1$; this is a portion
of the helical surface in figure~\xrefn{fig:helical ramp}.
We compute ${\bf r}_u = \langle -v\sin u,v\cos u,1\rangle$
and ${\bf r}_v = \langle \cos u,\sin u,0\rangle$.
The cross product of these two vectors is $\langle \sin u,-\cos
u,v\rangle$ with length $\sqrt{1+v^2}$, and the surface area is
$$\int_0^\pi\int_0^1 \sqrt{1+v^2}\,dv\,du={\pi\sqrt2\over2}+
{\pi\ln(\sqrt2+1)\over 2}.$$
\end{example}

\begin{exercises}

\begin{exercise} Describe or sketch the surface with the given vector function.
\begin{itemize} % BADBAD

  \item{a.} ${\bf r}(u,v) = \langle u+v,3-v,1+4u+5v\rangle$
  \item{b.} ${\bf r}(u,v) = \langle 2\sin u, 3\cos u, v\rangle$ 
  \item{c.} ${\bf r}(s,t) = \langle s,t,t^2-s^2\rangle$
  \item{d.} ${\bf r}(s,t) = \langle s\sin 2t, s^2, s\cos 2t\rangle$

\end{itemize}

\begin{exercise} Find a parametric representation, ${\bf r}(u,v)$, for the surface.
\begin{itemize} % BADBAD
  \item{a.} The plane that passes through the point $(1,2,-3)$ and is parallel
  to the vectors $\langle 1,1,-1\rangle$ and $\langle 1,-1,1\rangle$.

  \item{b.} The lower half of the ellipsoid $2x^2+4y^2+z^2=1$.

  \item{c.} The part of the sphere of radius 4 centered at the origin that lies
  between the planes $z=-2$ and $z=2$.
\end{itemize}

\begin{exercise} Find the area of the portion of $x+2y+4z=10$ in the first
octant.
\begin{answer} $25\sqrt{21}/4$
\end{answer}\end{exercise}

\begin{exercise} Find the area of the portion of $2x+4y+z=0$
inside $x^2+y^2=1$.
\begin{answer} $\pi\sqrt{21}$
\end{answer}\end{exercise}

\begin{exercise} Find the area of $z=x^2+y^2$ that lies below $z=1$.
\begin{answer} $\pi(5\sqrt5-1)/6$
\end{answer}\end{exercise}

\begin{exercise} Find the area of $z=\sqrt{x^2+y^2}$ that lies below $z=2$.
\begin{answer} $4\pi\sqrt2$
\end{answer}\end{exercise}

\begin{exercise} Find the area of the portion of $x^2+y^2+z^2=a^2$ that lies
in the first octant.
\begin{answer} $\pi a^2/2$
\end{answer}\end{exercise}

\begin{exercise} Find the area of the portion of $x^2+y^2+z^2=a^2$ that lies
above $x^2+y^2\le b^2$.
\begin{answer} $2\pi a(a-\sqrt{a^2-b^2})$
\end{answer}\end{exercise}

\begin{exercise} Find the area of $z=x^2-y^2$ that lies inside $x^2+y^2=a^2$.
\begin{answer} $\pi((1+4a^2)^{3/2}-1)/6$
\end{answer}\end{exercise}

\begin{exercise} Find the area of $z=xy$ that lies inside $x^2+y^2=a^2$.
\begin{answer} $2\pi((1+a^2)^{3/2}-1)/3$
\end{answer}\end{exercise}

\begin{exercise} Find the area of $x^2+y^2+z^2=a^2$ 
that lies above the interior of the circle given in polar coordinates
by $r=a\cos \theta$.
\begin{answer} $\pi a^2-2a^2$
\end{answer}\end{exercise}

\begin{exercise} Find the area of the cone $z=k\sqrt{x^2+y^2}$
that lies above the interior of the circle given in polar coordinates
by $r=a\cos \theta$.
\begin{answer} $\pi a^2\sqrt{1+k^2}/4$
\end{answer}\end{exercise}

\begin{exercise} Find the area of the plane $z=ax+by+c$ that lies over a
region $D$ with area $A$.
\begin{answer} $A\sqrt{1+a^2+b^2}$
\end{answer}\end{exercise}

\begin{exercise} Find the area of the cone $z=k\sqrt{x^2+y^2}$ that lies over a
region $D$ with area $A$.
\begin{answer} $A\sqrt{k^2+1}$
\end{answer}\end{exercise}

\begin{exercise} Find the area of the cylinder $x^2+z^2=a^2$ that lies inside
the cylinder $x^2+y^2=a^2$.
\begin{answer} $8a^2$
\end{answer}\end{exercise}

\begin{exercise} The surface $f(x,y)$ can be represented with the vector
function $\langle x,y,f(x,y)\rangle$. Set up the surface area integral using
this vector function and compare to the integral of
section~\xrefn{sec:surface area 3D}. 

\end{exercises}

