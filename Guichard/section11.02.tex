\section{Series}{}{}
\nobreak
While much more can be said about sequences, we now turn to our
principal interest, series. Recall that a series, roughly speaking, is
the sum of a sequence: if $\ds\{a_n\}_{n=0}^\infty$ is a sequence then the
associated series is
$$\sum_{i=0}^\infty a_n=a_0+a_1+a_2+\cdots$$
Associated with a series is a second sequence, called the {\dfont sequence of
  partial sums\index{sequence!of partial sums}\/} 
$\ds\{s_n\}_{n=0}^\infty$:
$$s_n=\sum_{i=0}^n a_i.$$
So
$$s_0=a_0,\quad s_1=a_0+a_1,\quad s_2=a_0+a_1+a_2,\quad \ldots$$
A series converges\index{series!convergent}\index{convergent series} 
if the sequence of partial sums converges, and otherwise the series 
diverges\index{series!divergent}\index{divergent series}.

\begin{example}
If $\ds a_n=kx^n$, $\ds\sum_{n=0}^\infty a_n$ is called a 
{\dfont geometric series\index{geometric series}\index{series!geometric}\/}.
A typical partial sum is
$$s_n=k+kx+kx^2+kx^3+\cdots+kx^n=k(1+x+x^2+x^3+\cdots+x^n).$$
We note that
$$\eqalign{
  s_n(1-x)&=k(1+x+x^2+x^3+\cdots+x^n)(1-x) \\
  &=k(1+x+x^2+x^3+\cdots+x^n)1-k(1+x+x^2+x^3+\cdots+x^{n-1}+x^n)x \\
  &=k(1+x+x^2+x^3+\cdots+x^n-x-x^2-x^3-\cdots-x^n-x^{n+1}) \\
  &=k(1-x^{n+1}) \\
}$$
so
$$\eqalign{
  s_n(1-x)&=k(1-x^{n+1}) \\
  s_n&=k{1-x^{n+1}\over 1-x}. \\
}$$
If $|x|<1$, $\ds\lim_{n\to\infty}x^n=0$ so
$$
  \lim_{n\to\infty}s_n=\lim_{n\to\infty}k{1-x^{n+1}\over 1-x}=
  k{1\over 1-x}.
$$ 
Thus, when $|x|<1$ the geometric series converges to $k/(1-x)$. When, for
  example, $k=1$ and $x=1/2$:
$$
  s_n={1-(1/2)^{n+1}\over 1-1/2}={2^{n+1}-1\over 2^n}=2-{1\over 2^n}
  \hbox{\quad and\quad } \sum_{n=0}^\infty {1\over 2^n} = 
  {1\over 1-1/2} = 2.
$$
We began the chapter with the series
$$\sum_{n=1}^\infty {1\over 2^n},$$
namely, the geometric series without the first term $1$. Each partial
sum of this series is 1 less than the corresponding partial sum for 
the geometric series, so of course the limit is also one less than the
value of the geometric series, that is,
$$\sum_{n=1}^\infty {1\over 2^n}=1.$$
\vskip-10pt\end{example}

It is not hard to see that the following theorem follows from
theorem~\xrefn{thm:properties of sequences}. 

\begin{theorem} \relax\label{thm:series are linear}
Suppose that $\sum a_n$ and $\sum b_n$ are convergent series,
and $c$ is a constant. Then

\begin{itemize} % BADBAD
\item{1.} $\ds\sum ca_n$ is convergent and $\ds\sum ca_n=c\sum a_n$
\item{2.} $\ds\sum (a_n+b_n)$ is convergent and 
$\ds\sum (a_n+b_n)=\sum a_n+\sum b_n$.
\end{itemize}
\end{proof}

The two parts of this theorem are subtly different. Suppose that
$\sum a_n$ diverges; does $\sum ca_n$ also diverge? Yes: suppose
instead that $\sum ca_n$ converges; then by the theorem,
$\sum (1/c)ca_n$ converges, but this is the same as $\sum a_n$,
which by assumption diverges. Hence $\sum ca_n$ also diverges. Note
that we are applying the theorem with $a_n$ replaced by $ca_n$ and $c$
replaced by $(1/c)$.

Now suppose that $\sum a_n$ and $\sum b_n$ diverge; does
$\sum (a_n+b_n)$ also diverge? Now the answer is no: Let $a_n=1$ and
$b_n=-1$, so certainly $\sum a_n$ and $\sum b_n$ diverge. But
$\sum (a_n+b_n)=\sum(1+-1)=\sum 0 = 0$. Of course, sometimes 
$\sum (a_n+b_n)$ will also diverge, for example, if $a_n=b_n=1$, then
$\sum (a_n+b_n)=\sum(1+1)=\sum 2$ diverges.

In general, the sequence of partial sums $\ds s_n$ is harder to understand
and analyze than the sequence of terms $\ds a_n$, and it is difficult
to determine whether series converge and if so to what. Sometimes
things are relatively simple, starting with the following.

\begin{theorem} If $\sum a_n$ converges then $\ds\lim_{n\to\infty}a_n=0$.
\begin{proof} Since $\sum a_n$ converges, $\ds\lim_{n\to\infty}s_n=L$ and 
$\ds\lim_{n\to\infty}s_{n-1}=L$, because this really says the same
thing but ``renumbers'' the terms. By
theorem~\xrefn{thm:properties of sequences}, 
$$
  \lim_{n\to\infty} (s_{n}-s_{n-1})=
  \lim_{n\to\infty} s_{n}-\lim_{n\to\infty}s_{n-1}=L-L=0.
$$
But
$$
  s_{n}-s_{n-1}=(a_0+a_1+a_2+\cdots+a_n)-(a_0+a_1+a_2+\cdots+a_{n-1})
  =a_n,
$$
so as desired $\ds\lim_{n\to\infty}a_n=0$.
\end{proof}
\label{thm:divergence test}

This theorem presents an easy divergence\index{divergence test} 
test: if given a series $\sum
a_n$ the limit $\ds\lim_{n\to\infty}a_n$ does not exist or has a value
other than zero, the series diverges. Note well that the converse is
{\em not\/} true: If $\ds\lim_{n\to\infty}a_n=0$ then the series does
not necessarily converge.

\begin{example}
Show that $\ds\sum_{n=1}^\infty {n\over n+1}$ diverges.
\par\nobreak\ssk\noindent
We compute the limit:
$$\lim _{n\to\infty}{n\over n+1}=1\not=0.$$
Looking at the first few terms perhaps makes it clear that the series
has no chance of converging:
$${1\over2}+{2\over3}+{3\over4}+{4\over5}+\cdots$$
will just get larger and larger; indeed, after a bit longer the series
starts to look very much like $\cdots+1+1+1+1+\cdots$, and of course
if we add up enough 1's we can make the sum as large as we desire.
\end{example}

\begin{example}
Show that $\ds\sum_{n=1}^\infty {1\over n}$ diverges.
\par\nobreak\ssk\noindent
Here the theorem does not apply: $\ds\lim _{n\to\infty} 1/n=0$, so it
looks like perhaps the series converges. Indeed, if you have the
fortitude (or the software) to add up the first 1000 terms you will find that
$$\sum_{n=1}^{1000} {1\over n}\approx 7.49,$$
so it might be reasonable to speculate that the series converges to
something in the neighborhood of 10. But in fact the partial sums do go
to infinity; they just get big very, very slowly. Consider the
following:
\msk
\hbox to \hsize{$\ds 1+{1\over 2}+{1\over 3}+{1\over 4} > 
1+{1\over 2}+{1\over 4}+{1\over
  4} = 1+{1\over 2}+{1\over 2}$\hfill}
\msk
\hbox to \hsize{$\ds 1+{1\over 2}+{1\over 3}+{1\over 4}+
{1\over 5}+{1\over 6}+{1\over
    7}+{1\over 8} > 
1+{1\over 2}+{1\over 4}+{1\over 4}+{1\over 8}+{1\over 8}+{1\over
    8}+{1\over 8} = 1+{1\over 2}+{1\over 2}+{1\over 2}$\hfill}
\msk
\hbox to \hsize{$\ds 1+{1\over 2}+{1\over 3}+\cdots+{1\over16}>
1+{1\over 2}+{1\over 4}+{1\over 4}+{1\over 8}+\cdots+{1\over
  8}+{1\over16}+\cdots +{1\over16} =1+{1\over 2}+{1\over 2}+{1\over
  2}+{1\over 2}$\hfill}

\msk\noindent
and so on. By swallowing up more and more terms we can always manage
to add at least another $1/2$ to the sum, and by adding enough of
these we can make the partial sums as big as we like. In fact, it's
not hard to see from this pattern that
$$1+{1\over 2}+{1\over 3}+\cdots+{1\over 2^n} > 1+{n\over 2},$$
so to make sure the sum is over 100, for example, we'd add
up terms until we get to around $\ds 1/2^{198}$, that is,
about $\ds 4\cdot 10^{59}$ terms. This series, $\sum (1/n)$, is called the
{\dfont harmonic series\index{series!harmonic}\index{harmonic series}\/}.
\end{example}

\begin{exercises}

\begin{exercise} Explain why $\ds\sum_{n=1}^\infty {n^2\over 2n^2+1}$
diverges.
\begin{answer} $\ds\lim_{n\to\infty} n^2/(2n^2+1)=1/2$
\end{answer}\end{exercise}

\begin{exercise} Explain why $\ds\sum_{n=1}^\infty {5\over 2^{1/n}+14}$
diverges.
\begin{answer} $\ds\lim_{n\to\infty} 5/(2^{1/n}+14)=1/3$
\end{answer}\end{exercise}

\begin{exercise} Explain why $\ds\sum_{n=1}^\infty {3\over n}$
diverges.
\begin{answer} $\sum_{n=1}^\infty {1\over n}$ diverges, so $\ds\sum_{n=1}^\infty 3{1\over n}$ diverges
\end{answer}\end{exercise}

\twocol

\begin{exercise} Compute $\ds\sum_{n=0}^\infty {4\over (-3)^n}- {3\over 3^n}$. 
\begin{answer} $-3/2$
\end{answer}\end{exercise}

\begin{exercise} Compute $\ds\sum_{n=0}^\infty {3\over 2^n}+ {4\over 5^n}$. 
\begin{answer} $11$
\end{answer}\end{exercise}

\begin{exercise} Compute $\ds\sum_{n=0}^\infty {4^{n+1}\over 5^n}$.
\begin{answer} $20$
\end{answer}\end{exercise}

\begin{exercise} Compute $\ds\sum_{n=0}^\infty {3^{n+1}\over 7^{n+1}}$.
\begin{answer} $3/4$
\end{answer}\end{exercise}

\begin{exercise} Compute $\ds\sum_{n=1}^\infty \left({3\over 5}\right)^n$.
\begin{answer} $3/2$
\end{answer}\end{exercise}

\begin{exercise} Compute $\ds\sum_{n=1}^\infty {3^n\over 5^{n+1}}$.
\begin{answer} $3/10$
\end{answer}\end{exercise}

\endtwocol

\end{exercises}


