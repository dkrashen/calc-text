\section{Polar Coordinates}{}{}
\nobreak
Coordinate systems are tools that let us use algebraic methods to
understand geometry. While the {\dfont rectangular\index{rectangular
    coordinates}\index{coordinates!rectangular}\/} (also called
{\dfont Cartesian\index{Cartesian
    coordinates}\index{coordinates!Cartesian}\/}) coordinates that we
have been using are the most common, some problems are easier to
analyze in alternate coordinate systems.

A coordinate system is a scheme that allows us to identify any point
in the plane or in three-dimensional space by a set of numbers. In
rectangular coordinates these numbers are interpreted, roughly
speaking, as the lengths of the sides of a rectangle. 
In {\dfont polar\index{polar coordinates}\index{coordinates!polar}
coordinates\/} a 
point in the plane is identified by a pair of numbers $(r,\theta)$.
The number $\theta$ measures the angle between the positive
$x$-axis and a ray that goes through the point,
as shown in figure~\xrefn{fig:polar coordinates example}; the number
$r$ measures the distance from the origin to the
point. Figure~\xrefn{fig:polar coordinates example} shows the point with
rectangular coordinates $\ds (1,\sqrt3)$ and polar coordinates 
$(2,\pi/3)$, 2 units from the origin and $\pi/3$ radians from the
positive $x$-axis.

\figure
\vbox{\beginpicture
\normalgraphs
\ninepoint
\setcoordinatesystem units <15truemm,15truemm>
\setplotarea x from 0 to 1.25, y from 0 to 1.9
\axis left ticks length <2pt> withvalues {$\sqrt3$} / at 1.728 / /
\axis bottom ticks length <2pt> withvalues {$1$} / at 1 / /
\arrow <4pt> [0.35, 1] from 0 0 to 1.2 2.078
\put {$(2,\pi/3)$} [l] <3pt,0pt> at 1 1.73
\put {$\bullet$} at 1 1.728
\circulararc 60 degrees from 0.2 0 center at 0 0
\endpicture}
\figrdef{fig:polar coordinates example}
\endfigure{Polar coordinates of the point $\ds (1,\sqrt3)$.}

Just as we describe curves in the plane using equations involving $x$
and $y$, so can we describe curves using equations involving $r$ and
$\theta$. Most common are equations of the form $r=f(\theta)$. 

\begin{example} Graph the curve given by $r=2$. All points with $r=2$ are at
distance 2 from the origin, so $r=2$ describes the circle of radius 2
with center at the origin.
\end{example}

\begin{example} Graph the curve given by $r=1+\cos\theta$. We first consider
$y=1+\cos x$, as in figure~\xrefn{fig:cardioid}. As $\theta$ goes
through the values in $[0,2\pi]$, the value of $r$ tracks the value of
$y$, forming the ``cardioid\index{cardioid}'' 
shape of  figure~\xrefn{fig:cardioid}.
For example, when $\theta=\pi/2$, $r=1+\cos(\pi/2)=1$, so we graph the
point at distance 1 from the origin along the positive $y$-axis, which
is at an angle of $\pi/2$ from the positive $x$-axis. When
$\theta=7\pi/4$, $\ds r=1+\cos(7\pi/4)=1+\sqrt2/2\approx 1.71$, and the
corresponding point appears in the fourth quadrant. This illustrates
one of the potential benefits of using polar coordinates: the equation
for this curve in rectangular coordinates would be quite complicated.
\end{example}

\figure
\vbox{\beginpicture
\normalgraphs
\ninepoint
\setcoordinatesystem units <12truemm,12truemm>
\setplotarea x from 0 to 6.2832, y from 0 to 2.2
\axis left ticks length <2pt> numbered from 0 to 2 by 1 /
\axis bottom ticks length <2pt> withvalues 
  {$\pi/2$} {$\pi$} {$3\pi/2$} {$2\pi$} / at 1.57 3.14 4.7 6.28 / /
\multiput {$\bullet$} at 1.57 1 5.5 1.71 /
\setquadratic
\plot
0.000 2.000 0.105 1.995 0.209 1.978 0.314 1.951 0.419 1.914 
0.524 1.866 0.628 1.809 0.733 1.743 0.838 1.669 0.942 1.588 
1.047 1.500 1.152 1.407 1.257 1.309 1.361 1.208 1.466 1.105 
1.571 1.000 1.676 0.895 1.780 0.792 1.885 0.691 1.990 0.593 
2.094 0.500 2.199 0.412 2.304 0.331 2.409 0.257 2.513 0.191 
2.618 0.134 2.723 0.086 2.827 0.049 2.932 0.022 3.037 0.005 
3.142 0.000 3.246 0.005 3.351 0.022 3.456 0.049 3.560 0.086 
3.665 0.134 3.770 0.191 3.875 0.257 3.979 0.331 4.084 0.412 
4.189 0.500 4.294 0.593 4.398 0.691 4.503 0.792 4.608 0.895 
4.712 1.000 4.817 1.105 4.922 1.208 5.027 1.309 5.131 1.407 
5.236 1.500 5.341 1.588 5.445 1.669 5.550 1.743 5.655 1.809 
5.760 1.866 5.864 1.914 5.969 1.951 6.074 1.978 6.178 1.995 
6.283 2.000 /
\setcoordinatesystem units <12truemm,12truemm> point at -8 -1
\setplotarea x from -1 to 2.5, y from -1.5 to 1.5
\axis left shiftedto x=0 /
\axis bottom shiftedto y=0 /
\put {$\bullet$} at 0 1
\put {$\bullet$} at 1.207 -1.207
\setquadratic
\plot 2.000 0.000 1.984 0.208 1.935 0.411 1.856 0.603 1.748 0.778 
1.616 0.933 1.464 1.063 1.295 1.166 1.117 1.240 0.933 1.285 
0.750 1.299 0.572 1.285 0.405 1.245 0.251 1.182 0.115 1.098 
-0.000 1.000 -0.094 0.891 -0.165 0.775 -0.214 0.657 -0.241 0.542 
-0.250 0.433 -0.242 0.333 -0.221 0.246 -0.191 0.172 -0.155 0.112 
-0.116 0.067 -0.079 0.035 -0.047 0.015 -0.021 0.005 -0.005 0.001 
0.000 0.000 -0.005 -0.001 -0.021 -0.005 -0.047 -0.015 -0.079 -0.035 
-0.116 -0.067 -0.155 -0.112 -0.191 -0.172 -0.221 -0.246 -0.242 -0.333 
-0.250 -0.433 -0.241 -0.542 -0.214 -0.657 -0.165 -0.775 -0.094 -0.891 
0.000 -1.000 0.115 -1.098 0.251 -1.182 0.405 -1.245 0.572 -1.285 
0.750 -1.299 0.933 -1.285 1.117 -1.240 1.295 -1.166 1.464 -1.063 
1.616 -0.933 1.748 -0.778 1.856 -0.603 1.935 -0.411 1.984 -0.208 
2.000 0.000 /
\setdashes
\arrow <4pt> [0.35, 1] from 0 0 to 1.207 -1.207
\endpicture}
\figrdef{fig:cardioid}
\endfigure{A cardioid: $y=1+\cos x$ on the left, $r=1+\cos\theta$ on
  the right.}

Each point in the plane is associated with exactly one pair of numbers
in the rectangular coordinate system; each point is associated with an
infinite number of pairs in polar coordinates. In the cardioid
example, we considered only the range $0\le \theta\le2\pi$, and
already there was a duplicate: $(2,0)$ and $(2,2\pi)$ are the same
point. Indeed, every value of $\theta$ outside the interval $[0,2\pi)$
duplicates a point on the curve $r=1+\cos\theta$ when
$0\le\theta<2\pi$. We can even make sense of polar coordinates like
$(-2,\pi/4)$: go to the direction $\pi/4$ and then move a distance 2
in the opposite direction; see figure~\xrefn{fig:negative r
coordinate}. As usual, a negative angle $\theta$ means an angle
measured clockwise from the positive $x$-axis. The point in
figure~\xrefn{fig:negative r coordinate} also has coordinates
$(2,5\pi/4)$ and $(2,-3\pi/4)$.

\figure
\vbox{\beginpicture
\normalgraphs
\eightpoint
\setcoordinatesystem units <10truemm,10truemm>
\setplotarea x from -2 to 2, y from -2 to 2
\axis left shiftedto x=0 ticks length <2pt> 
  withvalues $-2$ $-1$ {} $1$ $2$ / quantity 5 /
\axis bottom shiftedto y=0 ticks length <2pt> 
  withvalues $-2$ $-1$ {} $1$ $2$ / quantity 5 /
\put {$\pi/4$} [bl] <3pt,-4pt> at 0.4 0.3
\put {$\bullet$} at -1.4142 -1.4142
\circulararc 45 degrees from 0.4 0 center at 0 0
\setdashes\arrow <4pt> [0.35, 1] from 1 1 to -1.4142 -1.4142
\endpicture}
\figrdef{fig:negative r coordinate}
\endfigure{The point $(-2,\pi/4)=(2,5\pi/4)=(2,-3\pi/4)$ in polar coordinates.}

The relationship\index{coordinates!converting rectangular to polar}
between rectangular and polar coordinates is quite easy to
understand. The point with polar coordinates $(r,\theta)$ has
rectangular coordinates $x=r\cos\theta$ and $y=r\sin\theta$; this
follows immediately from the definition of the sine and cosine
functions. Using figure~\xrefn{fig:negative r coordinate} as an
example, the point shown has rectangular coordinates 
$\ds x=(-2)\cos(\pi/4)=-\sqrt2\approx 1.4142$ and 
$\ds y=(-2)\sin(\pi/4)=-\sqrt2$.  This makes it very easy to convert
equations from rectangular to polar coordinates.

\begin{example} Find the equation of the line $y=3x+2$ in polar
coordinates. We merely substitute: $r\sin\theta=3r\cos\theta+2$, or 
$\ds r= {2\over \sin\theta-3\cos\theta}$.
\end{example}

\begin{example} Find the equation of the circle $\ds (x-1/2)^2+y^2=1/4$ in polar
coordinates. Again substituting:
$\ds (r\cos\theta-1/2)^2+r^2\sin^2\theta=1/4$. A bit of algebra turns this
into $r=\cos(t)$. You should try plotting a few $(r,\theta)$ values to
convince yourself that this makes sense.
\end{example}

\begin{example} Graph the polar equation $r=\theta$. Here the distance from
the origin exactly matches the angle, so a bit of thought makes it
clear that when $\theta\ge0$ we get the spiral of 
Archimedes\index{spiral of Archimedes} in 
figure~\xrefn{fig:spiral of Archimedes}. When $\theta<0$, $r$ is also
negative, and so the full graph is the right hand picture in the
figure.
\end{example}

\figure
\vbox{\beginpicture
\normalgraphs
\eightpoint
\setcoordinatesystem units <5truemm,5truemm>
\setplotarea x from -3.4 to 6.5, y from -4.8 to 2
\axis left shiftedto x=0 /
\axis bottom shiftedto y=0 /
\multiput {$\bullet$} at 0.54 0.84 0 1.57 -3.14 0 6.28 0 /
\put {$(2\pi,2\pi)$} [b] <0pt,3pt> at 6.28 0
\put {$(\pi,\pi)$} [br] <-3pt,0pt> at -3.14 0
\put {$(\pi/2,\pi/2)$} [br] <-2pt,4pt> at 0 1.57
\put {$(1,1)$} [bl] <3pt,0pt> at 0.54 0.84
\plot 0.000 0.000 0.104 0.011 0.205 0.044 0.299 0.097 0.383 0.170
0.453 0.262 0.508 0.369 0.545 0.490 0.561 0.623 0.554 0.762
0.524 0.907 0.469 1.052 0.388 1.195 0.283 1.332 0.153 1.458
0.000 1.571 -0.175 1.666 -0.370 1.741 -0.582 1.793 -0.809 1.818
-1.047 1.814 -1.293 1.779 -1.542 1.712 -1.790 1.612 -2.033 1.477
-2.267 1.309 -2.487 1.107 -2.689 0.874 -2.868 0.610 -3.020 0.317
-3.142 0.000 -3.229 -0.339 -3.278 -0.697 -3.287 -1.068 -3.253 -1.448
-3.174 -1.833 -3.050 -2.216 -2.879 -2.593 -2.663 -2.957 -2.401 -3.304
-2.094 -3.628 -1.746 -3.922 -1.359 -4.183 -0.936 -4.405 -0.482 -4.582
0.000 -4.712 0.504 -4.791 1.023 -4.814 1.553 -4.781 2.087 -4.688
2.618 -4.534 3.139 -4.321 3.644 -4.047 4.125 -3.714 4.575 -3.324
4.988 -2.880 5.357 -2.385 5.677 -1.845 5.941 -1.263 6.145 -0.646
6.283 0.000 /
\setcoordinatesystem units <5truemm,5truemm> point at -18 0
\setplotarea x from -6.5 to 6.5, y from -4.8 to 2
\axis left shiftedto x=0 /
\axis bottom shiftedto y=0 /
\multiput {$\bullet$} at -0.54 0.84 0 1.57 3.14 0 -6.28 0 /
\put {$(-2\pi,-2\pi)$} [b] <0pt,3pt> at -6.28 0
\put {$(-\pi,-\pi)$} [bl] <3pt,2pt> at 3.14 0
\put {$(-\pi/2,-\pi/2)$} [br] <-2pt,4pt> at 0 1.57
\put {$(-1,-1)$} [t] <0pt,0pt> at -1.5 -1
\plot -6.283 0.000 -6.145 -0.646 -5.941 -1.263 -5.677 -1.845 -5.357 -2.385
-4.988 -2.880 -4.575 -3.324 -4.125 -3.714 -3.644 -4.047 -3.139 -4.321
-2.618 -4.534 -2.087 -4.688 -1.553 -4.781 -1.023 -4.814 -0.504 -4.791
0.000 -4.712 0.482 -4.582 0.936 -4.405 1.359 -4.183 1.746 -3.922
2.094 -3.628 2.401 -3.304 2.663 -2.957 2.879 -2.593 3.050 -2.216
3.174 -1.833 3.253 -1.448 3.287 -1.068 3.278 -0.697 3.229 -0.339
3.142 0.000 3.020 0.317 2.868 0.610 2.689 0.874 2.487 1.107
2.267 1.309 2.033 1.477 1.790 1.612 1.542 1.712 1.293 1.779
1.047 1.814 0.809 1.818 0.582 1.793 0.370 1.741 0.175 1.666
0.000 1.571 -0.153 1.458 -0.283 1.332 -0.388 1.195 -0.469 1.052
-0.524 0.907 -0.554 0.762 -0.561 0.623 -0.545 0.490 -0.508 0.369
-0.453 0.262 -0.383 0.170 -0.299 0.097 -0.205 0.044 -0.104 0.011
0.000 0.000 0.104 0.011 0.205 0.044 0.299 0.097 0.383 0.170
0.453 0.262 0.508 0.369 0.545 0.490 0.561 0.623 0.554 0.762
0.524 0.907 0.469 1.052 0.388 1.195 0.283 1.332 0.153 1.458
0.000 1.571 -0.175 1.666 -0.370 1.741 -0.582 1.793 -0.809 1.818
-1.047 1.814 -1.293 1.779 -1.542 1.712 -1.790 1.612 -2.033 1.477
-2.267 1.309 -2.487 1.107 -2.689 0.874 -2.868 0.610 -3.020 0.317
-3.142 0.000 -3.229 -0.339 -3.278 -0.697 -3.287 -1.068 -3.253 -1.448
-3.174 -1.833 -3.050 -2.216 -2.879 -2.593 -2.663 -2.957 -2.401 -3.304
-2.094 -3.628 -1.746 -3.922 -1.359 -4.183 -0.936 -4.405 -0.482 -4.582
0.000 -4.712 0.504 -4.791 1.023 -4.814 1.553 -4.781 2.087 -4.688
2.618 -4.534 3.139 -4.321 3.644 -4.047 4.125 -3.714 4.575 -3.324
4.988 -2.880 5.357 -2.385 5.677 -1.845 5.941 -1.263 6.145 -0.646
6.283 0.000 /
\setdashes <2pt>
\setlinear
\plot -2 -0.9 -0.54 0.84 /
%\altarrow <4pt,5pt> [0.35, 1] from -2 -0.9 to -0.54 0.84
\endpicture}
\figrdef{fig:spiral of Archimedes}
\endfigure{The spiral of Archimedes and the full graph of $r=\theta$.}

Converting polar equations to rectangular equations can be somewhat
trickier, and graphing polar equations directly is also not always easy.

\begin{example} Graph $r=2\sin\theta$. Because the sine is periodic, we know
that we will get the entire curve for values of $\theta$ in
$[0,2\pi)$. As $\theta$ runs from 0 to $\pi/2$, $r$ increases from 0
to 2. Then as $\theta$ continues to $\pi$, $r$ decreases again to
0. When $\theta$ runs from $\pi$ to $2\pi$, $r$ is negative, and it
is not hard to see that the first part of the curve is simply traced
out again, so in fact we get the whole curve for values of $\theta$
in $[0,\pi)$. Thus, the curve looks something like
figure~\xrefn{fig:circle from polar equation}. Now, this suggests
that the curve could possibly be a circle, and if it is, it would
have to be the circle $\ds x^2+(y-1)^2=1$. Having made this guess, we
can easily check it. First we substitute for $x$ and $y$ to get
$\ds (r\cos\theta)^2+(r\sin\theta-1)^2=1$; expanding and simplifying
does indeed turn this into $r=2\sin\theta$.
\end{example}

\figure
\vbox{\beginpicture
\normalgraphs
\ninepoint
\setcoordinatesystem units <15truemm,15truemm>
\setplotarea x from -1 to 1, y from 0 to 2.2
\axis left shiftedto x=0 ticks length <2pt> numbered from 1 to 1 by 1 /
\axis bottom ticks length <2pt> numbered from -1 to 1 by 1 /
\circulararc 360 degrees from 1 1 center at 0 1
\endpicture}
\figrdef{fig:circle from polar equation}
\endfigure{Graph of $r=2\sin\theta$.}

\begin{exercises}
% Almost all from Keisler

\begin{exercise} Plot these polar coordinate points on one graph:
$(2,\pi/3)$, $(-3,\pi/2)$, $(-2,-\pi/4)$, $(1/2,\pi)$, $(1,4\pi/3)$, 
$(0,3\pi/2)$.

\noindent Find an equation in polar coordinates that has the same
graph as the given equation in rectangular coordinates.

\twocol

\begin{exercise} $\ds y=3x$
\begin{answer} $\ds \theta=\arctan(3)$
\end{answer}\end{exercise}

\begin{exercise} $\ds y=-4$
\begin{answer} $\ds r=-4\csc\theta$
\end{answer}\end{exercise}

\begin{exercise} $\ds xy^2=1$
\begin{answer} $\ds r=\sec\theta\csc^2\theta$
\end{answer}\end{exercise}

\begin{exercise} $\ds x^2+y^2=5$
\begin{answer} $\ds r=\sqrt{5}$
\end{answer}\end{exercise}

\begin{exercise} $\ds y=x^3$
\begin{answer} $\ds r^2=\sin\theta\sec^3\theta$
\end{answer}\end{exercise}

\begin{exercise} $\ds y=\sin x$
\begin{answer} $\ds r\sin\theta=\sin(r\cos\theta)$
\end{answer}\end{exercise}

\begin{exercise} $\ds y=5x+2$
\begin{answer} $\ds r=2/(\sin\theta-5\cos\theta)$
\end{answer}\end{exercise}

\begin{exercise} $\ds x=2$
\begin{answer} $\ds r=2\sec\theta$
\end{answer}\end{exercise}

\begin{exercise} $\ds y=x^2+1$
\begin{answer} $\ds 0=r^2\cos^2\theta-r\sin\theta+1$
\end{answer}\end{exercise}

\begin{exercise} $\ds y=3x^2-2x$
\begin{answer} $\ds 0=3r^2\cos^2\theta-2r\cos\theta-r\sin\theta$
\end{answer}\end{exercise}

\begin{exercise} $\ds y=x^2+y^2$
\begin{answer} $\ds r=\sin\theta$
\end{answer}\end{exercise}

\endtwocol
\bsk
\noindent Sketch the curve.
\twocol

\begin{exercise} $\ds r=\cos\theta$

\begin{exercise} $\ds r=\sin(\theta+\pi/4)$

\begin{exercise} $\ds r=-\sec\theta$

\begin{exercise} $\ds r=\theta/2$, $\theta\ge0$

\begin{exercise} $\ds r=1+\theta^1/\pi^2$

\begin{exercise} $\ds r=\cot\theta\csc\theta$

\begin{exercise} $\ds r={1\over\sin\theta+\cos\theta}$

\begin{exercise} $\ds r^2=-2\sec\theta\csc\theta$

\endtwocol

\bigbreak
\noindent Find an equation in rectangular coordinates that has the same
graph as the given equation in polar coordinates.

\twocol

\begin{exercise} $\ds r=\sin(3\theta)$
\begin{answer} $\ds (x^2+y^2)^2=4x^2y-(x^2+y^2)y$
\end{answer}\end{exercise}

\begin{exercise} $\ds r=\sin^2\theta$
\begin{answer} $\ds (x^2+y^2)^{3/2}=y^2$
\end{answer}\end{exercise}

\begin{exercise} $\ds r=\sec\theta\csc\theta$
\begin{answer} $\ds x^2+y^2=x^2y^2$
\end{answer}\end{exercise}

\begin{exercise} $\ds r=\tan\theta$
\begin{answer} $\ds x^4+x^2y^2=y^2$
\end{answer}\end{exercise}

\endtwocol

\end{exercises}
