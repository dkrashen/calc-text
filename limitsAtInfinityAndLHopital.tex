\chapter{Limits at Infinity and L'Hopital's rule}

\section{Limits at Infinity}


Consider the function:
\[
f(x) = \frac{6x-9}{x-1}
\]
\begin{marginfigure}[0in]
\begin{tikzpicture}
	\begin{axis}[
            domain=1:4,
            ymax=20,
            ymin=-10,
            samples=100,
            axis lines =middle, xlabel=$x$, ylabel=$y$,
            every axis y label/.style={at=(current axis.above origin),anchor=south},
            every axis x label/.style={at=(current axis.right of origin),anchor=west}
          ]
	  \addplot [very thick, penColor, smooth, domain=(0:.9)] {(6*x-9)/(x-1)};
          \addplot [very thick, penColor, smooth, domain=(1.1:3)] {(6*x-9)/(x-1)};
          \addplot [textColor, dashed] plot coordinates {(1,-10) (1,20)};
        \end{axis}
\end{tikzpicture}
\caption{A plot of $f(x)=\protect\frac{6x-9}{x-1}$.}
\label{plot:(6x-9)/(x-1)}
\end{marginfigure}
As $x$ approaches infinity, it seems like $f(x)$ approaches a specific
value. This is a \textit{limit at infinity}.

\begin{definition}\label{def:limitAtInfty}\index{limit!at infinity}
If $f(x)$ becomes arbitrarily close to a specific value $L$ by making
$x$ sufficiently large, we write
\[
\lim_{x\to \infty} f(x) = L
\]
and we say, the \textbf{limit at infinity} of $f(x)$ is $L$.  

If $f(x)$ becomes
arbitrarily close to a specific value $L$ by making $x$ sufficiently
large and negative, we write
\[
\lim_{x\to -\infty} f(x) = L
\]
and we say, the \textbf{limit at negative infinity} of $f(x)$ is $L$.  
\end{definition}

\begin{example} Compute
\[
\lim_{x\to\infty} \frac{6x-9}{x-1}.
\]
\end{example}


\begin{solution}
Write
\begin{align*}
\lim_{x\to\infty}\frac{6x-9}{x-1} &= \lim_{x\to\infty}\frac{6x-9}{x-1} \frac{1/x}{1/x}\\
&=\lim_{x\to\infty}\frac{\frac{6x}{x} - \frac{9}{x}}{\frac{x}{x} - \frac{1}{x}}\\
&= \lim_{x\to\infty} \frac{6}{1}\\
&= 6.
\end{align*}
\end{solution}

Sometimes one must be careful, consider this example.

\begin{example}
Compute
\[
\lim_{x\to -\infty} \frac{x+1}{\sqrt{x^2}}
\]
\end{example}

\begin{solution}
In this case we multiply the numerator and denominator by $-1/x$,
which is a positive number as since $x\to -\infty$, $x$ is a negative
number.
\begin{align*}
\lim_{x\to -\infty} \frac{x+1}{\sqrt{x^2}} &= \lim_{x\to -\infty} \frac{x+1}{\sqrt{x^2}} \cdot \frac{-1/x}{-1/x}\\
&= \lim_{x\to -\infty} \frac{-1-1/x}{\sqrt{x^2/x^2}}\\
&= -1.
\end{align*}
\end{solution}


Here is a somewhat different example of a limit at infinity.

\begin{example}
Compute
\[
\lim_{x\to \infty} \frac{\sin(7x)}{x}+4.
\]
\end{example}

\begin{marginfigure}[0in]
\begin{tikzpicture}
	\begin{axis}[
            domain=2:20,
            ymax=5,
            ymin=3,
            samples=100,
            axis lines =middle, xlabel=$x$, ylabel=$y$,
            every axis y label/.style={at=(current axis.above origin),anchor=south},
            every axis x label/.style={at=(current axis.right of origin),anchor=west}
          ]
	  \addplot [very thick, penColor, smooth] {(1/x) * sin(deg(7*x))+4};
        \end{axis}
\end{tikzpicture}
\caption{A plot of $f(x)=\frac{\sin(7x)}{x}+4$.}
\label{plot:sin7x/x+4}
\end{marginfigure}

\begin{solution}
We can bound our function
\[
-1/x + 4 \le \frac{\sin(7x)}{x}+4 \le 1/x + 4.
\]
Since 
\[
\lim_{x\to \infty} -1/x + 4 = 4 = \lim_{x\to \infty}1/x + 4
\] 
we conclude by the Squeeze Theorem, Theorem~\ref{theorem:squeeze},
$\lim_{x\to\infty}\frac{\sin(7x)}{x}+4 = 4$.
\end{solution}






\begin{definition}\label{def:horiz asymptote}\index{asymptote!horizontal}\index{horizontal asymptote}
If  
\[
\lim_{x\to \infty} f(x) = L \qquad\text{or}\qquad \lim_{x\to -\infty} f(x) = L,
\]
then the line $y=L$ is a \textbf{horizontal asymptote} of $f(x)$.
\end{definition}

\begin{example} 
Give the horizontal asymptotes of
\[
f(x) = \frac{6x-9}{x-1}
\]
\end{example}

\begin{solution}
From our previous work, we see that $\lim_{x\to \infty} f(x) = 6$, and
upon further inspection, we see that $\lim_{x\to -\infty} f(x) =
6$. Hence the horizontal asymptote of $f(x)$ is the line $y=6$.
\end{solution}


It is a common misconception that a function cannot cross an
asymptote. As the next example shows, a function can cross an
asymptote, and in this case this occurs an infinite number of times!

\begin{example}
Give a horizontal asymptote of
\[
f(x) = \frac{\sin(7x)}{x}+4.
\]
\end{example}

\begin{solution}
Again from previous work, we see that $\lim_{x\to \infty} f(x) =
4$. Hence $y=4$ is a horizontal asymptote of $f(x)$.
\end{solution}


We conclude with an infinite limit at infinity.

\begin{example}
Compute
\[
\lim_{x\to \infty} \ln(x)
\]
\end{example}
\begin{marginfigure}[0in]
\begin{tikzpicture}
	\begin{axis}[
            domain=0:20,
            ymax=4,
            ymin=-1,
            samples=100,
            axis lines =middle, xlabel=$x$, ylabel=$y$,
            every axis y label/.style={at=(current axis.above origin),anchor=south},
            every axis x label/.style={at=(current axis.right of origin),anchor=west}
          ]
	  \addplot [very thick, penColor, smooth] {ln(x)};
        \end{axis}
\end{tikzpicture}
\caption{A plot of $f(x)=\ln(x)$.}
\label{plot:lnx}
\end{marginfigure}

\begin{solution}
The function $\ln(x)$ grows very slowly, and seems like it may have a
horizontal asymptote, see Figure~\ref{plot:lnx}. However, if we
consider the definition of the natural log
\[
\ln(x) = y \qquad \Leftrightarrow\qquad e^y =x
\]
Since we need to raise $e$ to higher and higher values to obtain
larger numbers, we see that $\ln(x)$ is unbounded, and hence
$\lim_{x\to\infty}\ln(x)=\infty$.
\end{solution}


\begin{exercises}

\noindent Compute the limits.
\twocol
\begin{exercise}
$\lim_{x\to \infty} \frac{1}{x}$
\begin{answer}
$0$
\end{answer}
\end{exercise}

\begin{exercise}
$\lim_{x\to \infty} \frac{-x}{\sqrt{4+x^2}}$
\begin{answer}
$-1$
\end{answer}
\end{exercise}

\begin{exercise}
$\lim_{x\to \infty} \frac{2x^2-x+1}{4x^2-3x-1}$
\begin{answer}
$\frac{1}{2}$
\end{answer}
\end{exercise}

\begin{exercise}
$\lim_{x\to -\infty} \frac{3x+7}{\sqrt{x^2}}$
\begin{answer}
$-3$
\end{answer}
\end{exercise}

\begin{exercise}
$\lim_{x\to -\infty} \frac{2x+7}{\sqrt{x^2+2x-1}}$
\begin{answer}
$-2$
\end{answer}
\end{exercise}

\begin{exercise}
$\lim_{x\to -\infty} \frac{x^3-4}{3x^2+4x-1}$
\begin{answer}
$-\infty$
\end{answer}
\end{exercise}


\begin{exercise}
$\lim_{x\to \infty} \left(\frac{4}{x}+\pi\right)$
\begin{answer}
$\pi$
\end{answer}
\end{exercise}

\begin{exercise}
$\lim_{x\to \infty} \frac{\cos(x)}{\ln(x)}$
\begin{answer}
$0$
\end{answer}
\end{exercise}

\begin{exercise}
$\lim_{x\to \infty} \frac{\sin\left(x^7\right)}{\sqrt{x}}$
\begin{answer}
$0$
\end{answer}
\end{exercise}

\begin{exercise}
$\lim_{x\to \infty} \left(17 + \frac{32}{x} - \frac{\left(\sin(x/2)\right)^2}{x^3}\right)$
\begin{answer}
$17$
\end{answer}
\end{exercise}

\endtwocol

\begin{exercise}
Suppose a population of feral cats on a certain college campus $t$
years from now is approximated by
\[
p(t) = \frac{1000}{5+ 2e^{-0.1 t}}.
\]
Approximately how many feral cats are on campus 10 years from now? 50
years from now? 100 years from now? 1000 years from now? What do you
notice about the prediction---is this realistic?
\begin{answer}
After 10 years, $\approx 174$ cats; after 50 years, $\approx 199$
cats; after 100 years, $\approx 200$ cats; after 1000 years, $\approx
200$ cats; in the sense that the population of cats cannot grow
indefinitely this is somewhat realistic.
\end{answer}
\end{exercise}

\begin{exercise}
The amplitude of an oscillating spring is given by
\[
a(t) = \frac{\sin(t)}{t}.
\]
What happens to the amplitude of the oscillation over a long period of
time?
\begin{answer}
The amplitude goes to zero. 
\end{answer}
\end{exercise}
\end{exercises}






\section{Limits at Infinity}


Consider the function:
\[
f(x) = \frac{6x-9}{x-1}
\]
\begin{marginfigure}[0in]
\begin{tikzpicture}
	\begin{axis}[
            domain=1:4,
            ymax=20,
            ymin=-10,
            samples=100,
            axis lines =middle, xlabel=$x$, ylabel=$y$,
            every axis y label/.style={at=(current axis.above origin),anchor=south},
            every axis x label/.style={at=(current axis.right of origin),anchor=west}
          ]
	  \addplot [very thick, penColor, smooth, domain=(0:.9)] {(6*x-9)/(x-1)};
          \addplot [very thick, penColor, smooth, domain=(1.1:3)] {(6*x-9)/(x-1)};
          \addplot [textColor, dashed] plot coordinates {(1,-10) (1,20)};
        \end{axis}
\end{tikzpicture}
\caption{A plot of $f(x)=\protect\frac{6x-9}{x-1}$.}
\label{plot:(6x-9)/(x-1)}
\end{marginfigure}
As $x$ approaches infinity, it seems like $f(x)$ approaches a specific
value. This is a \textit{limit at infinity}.

\begin{definition}\label{def:limitAtInfty}\index{limit!at infinity}
If $f(x)$ becomes arbitrarily close to a specific value $L$ by making
$x$ sufficiently large, we write
\[
\lim_{x\to \infty} f(x) = L
\]
and we say, the \textbf{limit at infinity} of $f(x)$ is $L$.  

If $f(x)$ becomes
arbitrarily close to a specific value $L$ by making $x$ sufficiently
large and negative, we write
\[
\lim_{x\to -\infty} f(x) = L
\]
and we say, the \textbf{limit at negative infinity} of $f(x)$ is $L$.  
\end{definition}

\begin{example} Compute
\[
\lim_{x\to\infty} \frac{6x-9}{x-1}.
\]
\end{example}


\begin{solution}
Write
\begin{align*}
\lim_{x\to\infty}\frac{6x-9}{x-1} &= \lim_{x\to\infty}\frac{6x-9}{x-1} \frac{1/x}{1/x}\\
&=\lim_{x\to\infty}\frac{\frac{6x}{x} - \frac{9}{x}}{\frac{x}{x} - \frac{1}{x}}\\
&= \lim_{x\to\infty} \frac{6}{1}\\
&= 6.
\end{align*}
\end{solution}

Sometimes one must be careful, consider this example.

\begin{example}
Compute
\[
\lim_{x\to -\infty} \frac{x+1}{\sqrt{x^2}}
\]
\end{example}

\begin{solution}
In this case we multiply the numerator and denominator by $-1/x$,
which is a positive number as since $x\to -\infty$, $x$ is a negative
number.
\begin{align*}
\lim_{x\to -\infty} \frac{x+1}{\sqrt{x^2}} &= \lim_{x\to -\infty} \frac{x+1}{\sqrt{x^2}} \cdot \frac{-1/x}{-1/x}\\
&= \lim_{x\to -\infty} \frac{-1-1/x}{\sqrt{x^2/x^2}}\\
&= -1.
\end{align*}
\end{solution}


Here is a somewhat different example of a limit at infinity.

\begin{example}
Compute
\[
\lim_{x\to \infty} \frac{\sin(7x)}{x}+4.
\]
\end{example}

\begin{marginfigure}[0in]
\begin{tikzpicture}
	\begin{axis}[
            domain=2:20,
            ymax=5,
            ymin=3,
            samples=100,
            axis lines =middle, xlabel=$x$, ylabel=$y$,
            every axis y label/.style={at=(current axis.above origin),anchor=south},
            every axis x label/.style={at=(current axis.right of origin),anchor=west}
          ]
	  \addplot [very thick, penColor, smooth] {(1/x) * sin(deg(7*x))+4};
        \end{axis}
\end{tikzpicture}
\caption{A plot of $f(x)=\frac{\sin(7x)}{x}+4$.}
\label{plot:sin7x/x+4}
\end{marginfigure}

\begin{solution}
We can bound our function
\[
-1/x + 4 \le \frac{\sin(7x)}{x}+4 \le 1/x + 4.
\]
Since 
\[
\lim_{x\to \infty} -1/x + 4 = 4 = \lim_{x\to \infty}1/x + 4
\] 
we conclude by the Squeeze Theorem, Theorem~\ref{theorem:squeeze},
$\lim_{x\to\infty}\frac{\sin(7x)}{x}+4 = 4$.
\end{solution}






\begin{definition}\label{def:horiz asymptote}\index{asymptote!horizontal}\index{horizontal asymptote}
If  
\[
\lim_{x\to \infty} f(x) = L \qquad\text{or}\qquad \lim_{x\to -\infty} f(x) = L,
\]
then the line $y=L$ is a \textbf{horizontal asymptote} of $f(x)$.
\end{definition}

\begin{example} 
Give the horizontal asymptotes of
\[
f(x) = \frac{6x-9}{x-1}
\]
\end{example}

\begin{solution}
From our previous work, we see that $\lim_{x\to \infty} f(x) = 6$, and
upon further inspection, we see that $\lim_{x\to -\infty} f(x) =
6$. Hence the horizontal asymptote of $f(x)$ is the line $y=6$.
\end{solution}


It is a common misconception that a function cannot cross an
asymptote. As the next example shows, a function can cross an
asymptote, and in this case this occurs an infinite number of times!

\begin{example}
Give a horizontal asymptote of
\[
f(x) = \frac{\sin(7x)}{x}+4.
\]
\end{example}

\begin{solution}
Again from previous work, we see that $\lim_{x\to \infty} f(x) =
4$. Hence $y=4$ is a horizontal asymptote of $f(x)$.
\end{solution}


We conclude with an infinite limit at infinity.

\begin{example}
Compute
\[
\lim_{x\to \infty} \ln(x)
\]
\end{example}
\begin{marginfigure}[0in]
\begin{tikzpicture}
	\begin{axis}[
            domain=0:20,
            ymax=4,
            ymin=-1,
            samples=100,
            axis lines =middle, xlabel=$x$, ylabel=$y$,
            every axis y label/.style={at=(current axis.above origin),anchor=south},
            every axis x label/.style={at=(current axis.right of origin),anchor=west}
          ]
	  \addplot [very thick, penColor, smooth] {ln(x)};
        \end{axis}
\end{tikzpicture}
\caption{A plot of $f(x)=\ln(x)$.}
\label{plot:lnx}
\end{marginfigure}

\begin{solution}
The function $\ln(x)$ grows very slowly, and seems like it may have a
horizontal asymptote, see Figure~\ref{plot:lnx}. However, if we
consider the definition of the natural log
\[
\ln(x) = y \qquad \Leftrightarrow\qquad e^y =x
\]
Since we need to raise $e$ to higher and higher values to obtain
larger numbers, we see that $\ln(x)$ is unbounded, and hence
$\lim_{x\to\infty}\ln(x)=\infty$.
\end{solution}


\begin{exercises}

\noindent Compute the limits.
\twocol
\begin{exercise}
$\lim_{x\to \infty} \frac{1}{x}$
\begin{answer}
$0$
\end{answer}
\end{exercise}

\begin{exercise}
$\lim_{x\to \infty} \frac{-x}{\sqrt{4+x^2}}$
\begin{answer}
$-1$
\end{answer}
\end{exercise}

\begin{exercise}
$\lim_{x\to \infty} \frac{2x^2-x+1}{4x^2-3x-1}$
\begin{answer}
$\frac{1}{2}$
\end{answer}
\end{exercise}

\begin{exercise}
$\lim_{x\to -\infty} \frac{3x+7}{\sqrt{x^2}}$
\begin{answer}
$-3$
\end{answer}
\end{exercise}

\begin{exercise}
$\lim_{x\to -\infty} \frac{2x+7}{\sqrt{x^2+2x-1}}$
\begin{answer}
$-2$
\end{answer}
\end{exercise}

\begin{exercise}
$\lim_{x\to -\infty} \frac{x^3-4}{3x^2+4x-1}$
\begin{answer}
$-\infty$
\end{answer}
\end{exercise}


\begin{exercise}
$\lim_{x\to \infty} \left(\frac{4}{x}+\pi\right)$
\begin{answer}
$\pi$
\end{answer}
\end{exercise}

\begin{exercise}
$\lim_{x\to \infty} \frac{\cos(x)}{\ln(x)}$
\begin{answer}
$0$
\end{answer}
\end{exercise}

\begin{exercise}
$\lim_{x\to \infty} \frac{\sin\left(x^7\right)}{\sqrt{x}}$
\begin{answer}
$0$
\end{answer}
\end{exercise}

\begin{exercise}
$\lim_{x\to \infty} \left(17 + \frac{32}{x} - \frac{\left(\sin(x/2)\right)^2}{x^3}\right)$
\begin{answer}
$17$
\end{answer}
\end{exercise}

\endtwocol

\begin{exercise}
Suppose a population of feral cats on a certain college campus $t$
years from now is approximated by
\[
p(t) = \frac{1000}{5+ 2e^{-0.1 t}}.
\]
Approximately how many feral cats are on campus 10 years from now? 50
years from now? 100 years from now? 1000 years from now? What do you
notice about the prediction---is this realistic?
\begin{answer}
After 10 years, $\approx 174$ cats; after 50 years, $\approx 199$
cats; after 100 years, $\approx 200$ cats; after 1000 years, $\approx
200$ cats; in the sense that the population of cats cannot grow
indefinitely this is somewhat realistic.
\end{answer}
\end{exercise}

\begin{exercise}
The amplitude of an oscillating spring is given by
\[
a(t) = \frac{\sin(t)}{t}.
\]
What happens to the amplitude of the oscillation over a long period of
time?
\begin{answer}
The amplitude goes to zero. 
\end{answer}
\end{exercise}
\end{exercises}


\section{L'H\^{o}pital's Rule}


Derivatives allow us to take problems that were once difficult to
solve and convert them to problems that are easier to solve. Let us
consider l'H\^{o}pital's rule:

\marginnote[1in]{L'H\^opital's rule applies even when $\lim_{x\to a}f(x) =
  \pm \infty$ and $\lim_{x\to a}g(x) = \mp \infty$. See Example~\ref{example:xlnx infty}.}
\begin{mainTheorem}[L'H\^opital's Rule]\index{l'H\^opital's Rule} 
Let $f(x)$ and $g(x)$ be functions that are differentiable near $a$.  If
\[
\lim_{x \to a} f(x) = \lim_{x \to a}g(x) = 0 \qquad \text{or} \pm \infty,
\]
and $\lim_{x \to a} \frac{f'(x)}{g'(x)}$ exists, and $g'(x) \neq 0$
for all $x$ near $a$, then 
\[
\lim_{x \to a} \frac{f(x)}{g(x)} = \lim_{x \to a} \frac{f'(x)}{g'(x)}.
\]
\end{mainTheorem}
This theorem is somewhat difficult to prove, in part because it
incorporates so many different possibilities, so we will not prove it
here. 

\break

L'H\^{o}pital's rule allows us to investigate limits of
\textit{indeterminate form}.

\begin{definition}[List of Indeterminate Forms]\index{indeterminate form}\hfil
\begin{itemize}
\item[\textbf{0/0}] This refers to a limit of the form $\lim_{x\to a}
  \frac{f(x)}{g(x)}$ where $f(x)\to 0$ and $g(x)\to 0$ as $x\to a$.
\item[\textbf{$\pmb\infty$/$\pmb\infty$}] This refers to a limit of the form $\lim_{x\to a}
  \frac{f(x)}{g(x)}$ where $f(x)\to \infty$ and $g(x)\to \infty$ as $x\to a$.
\item[\textbf{0\,$\pmb{\cdot\infty}$}] This refers to a limit of the form $\lim_{x\to a}
  \left(f(x)\cdot g(x)\right)$ where $f(x)\to 0$ and $g(x)\to \infty$ as $x\to a$.
\item[\textbf{$\pmb\infty$--$\pmb\infty$}] This refers to a limit of the form $\lim_{x\to a}\left(
  f(x)-g(x)\right)$ where $f(x)\to \infty$ and $g(x)\to \infty$ as $x\to a$.

\item[\textbf{1$^{\pmb\infty}$}] This refers to a limit of the form $\lim_{x\to a}
  f(x)^{g(x)}$ where $f(x)\to 1$ and $g(x)\to \infty$ as $x\to a$.
\item[\textbf{0$^\text{0}$}] This refers to a limit of the form $\lim_{x\to a}
  f(x)^{g(x)}$ where $f(x)\to 0$ and $g(x)\to 0$ as $x\to a$.
\item[\textbf{$\pmb\infty^\text{0}$}] This refers to a limit of the form $\lim_{x\to a}
  f(x)^{g(x)}$ where $f(x)\to \infty$ and $g(x)\to 0$ as $x\to a$.
\end{itemize}
In each of these cases, the value of the limit is \textbf{not} immediately
obvious. Hence, a careful analysis is required!
\end{definition}

Our first example is the computation of a limit that was somewhat
difficult before, see Example~\ref{example:sinx/x}. Note, this is an
example of the indeterminate form $0/0$.

\begin{example}[0/0]\label{example:sinx/x-lhopital}
Compute
\[
\lim_{x\to 0} \frac{\sin(x)}{x}.
\]
\end{example}

\begin{solution}
Set $f(x) = \sin(x)$ and $g(x) = x$.  Since both $f(x)$ and $g(x)$ are
differentiable functions at $0$, and 
\[
\lim_{x \to 0} f(x) = \lim_{x \to 0}g(x) = 0,
\]
this situation is ripe for l'H\^opital's Rule. Now
\[
f'(x) = \cos(x) \qquad\text{and}\qquad g'(x) = 1.
\] 
L'H\^opital's rule tells us that 
\[
\lim_{x \to 0} \frac{\sin(x)}{x} = \lim_{x \to 0} \frac{\cos(x)}{1} = 1.
\]
\end{solution}


\begin{marginfigure}[-1in]
\begin{tikzpicture}
	\begin{axis}[
            xmin=-1.6,xmax=1.6,ymin=-1.5,ymax=1.5,
            axis lines=center,
            xtick={-1.57, 0, 1.57},
            xticklabels={$-\pi/2$, $0$, $\pi/2$},
            ytick={-1,1},
            %ticks=none,
            %width=3in,
            %height=2in,
            unit vector ratio*=1 1 1,
            xlabel=$x$, ylabel=$y$,
            every axis y label/.style={at=(current axis.above origin),anchor=south},
            every axis x label/.style={at=(current axis.right of origin),anchor=west},
          ]        
          \addplot [very thick, penColor, samples=100,smooth, domain=(-1.6:1.6)] {sin(deg(x))};
          \addplot [very thick, penColor2] {x};
          \node at (axis cs:1,.6) [penColor] {$f(x)$};
          \node at (axis cs:-1,-1.2) [penColor2] {$g(x)$};
        \end{axis}
\end{tikzpicture}
\caption{A plot of $f(x)=\sin(x)$ and $g(x) = x$. Note how the tangent
  lines for each curve are coincident at $x=0$.}
\label{example:sinx and x}
\end{marginfigure}


From this example, we gain an intuitive feeling for why l'H\^opital's
rule is true: If two functions are both $0$ when $x=a$, and if their
tangent lines have the same slope, then the functions coincide as $x$
approaches $a$. See Figure~\ref{example:sinx and x}. 




Our next set of examples will run through the remaining indeterminate
forms one is likely to encounter.

\begin{example}[$\pmb\infty$/$\pmb\infty$] Compute 
\[
\lim_{x\to \pi/2+} \frac{\sec(x)}{\tan(x)}.
\]
\end{example}

\begin{solution}
Set $f(x) = \sec(x)$ and $g(x) = \tan(x)$. Both $f(x)$ and $g(x)$
are differentiable near $\pi/2$. Additionally,
\[
\lim_{x \to \pi/2+} f(x) = \lim_{x \to \pi/2+}g(x) = -\infty.
\]
This situation is ripe for l'H\^opital's Rule. Now 
\[
f'(x) = \sec(x)\tan(x) \qquad\text{and}\qquad g'(x) = \sec^2(x).
\]
L'H\^opital's rule tells us that 
\[
\lim_{x\to \pi/2+} \frac{\sec(x)}{\tan(x)} = \lim_{x\to \pi/2+}
\frac{\sec(x)\tan(x)}{\sec^2(x)} = \lim_{x\to \pi/2+} \sin(x) =
1.
\]
\end{solution}



\begin{example}[0\,$\pmb{\cdot\infty}$]\label{example:xlnx infty} 
Compute 
\[
\lim_{x\to 0+} x\ln x.
\]
\end{example}

\begin{solution}
This doesn't appear to be suitable for l'H\^opital's Rule. As $x$
approaches zero, $\ln x$ goes to $-\infty$, so the product looks like
\[
(\text{something very small})\cdot (\text{something very large and
  negative}).
\] 
This product could be anything---a careful analysis is required.
Write
\[
x\ln x = \frac{\ln x}{x^{-1}}.
\]
Set $f(x) = \ln(x)$ and $g(x) = x^{-1}$.  Since both functions are differentiable near zero and 
\[
\lim_{x\to 0+} \ln(x) = -\infty\qquad\text{and}\qquad \lim_{x\to 0+} x^{-1} = \infty,
\]
we may apply l'H\^opital's rule. Write
\[
f'(x) = x^{-1}\qquad \text{and}\qquad g'(x) = -x^{-2},
\]
so
\[
\lim_{x\to 0+} x\ln x = \lim_{x\to 0+} \frac{\ln x}{x^{-1}} = \lim_{x\to 0+} \frac{x^{-1}}{-x^{-2}} =\lim_{x\to 0+} -x = 0.
\]
One way to interpret this is that since $\lim_{x\to 0^+}x\ln x = 0$,
the function $x$ approaches zero much faster than $\ln x$ approaches
$-\infty$.
\end{solution}

\subsection*{Indeterminate Forms Involving Subtraction}

There are two basic cases here, we'll do an example of each.

\begin{example}[$\pmb\infty$--$\pmb\infty$]
Compute
\[
\lim_{x\to 0} \left(\cot(x) - \csc(x)\right).
\]
\end{example}

\begin{solution}
Here we simply need to write each term as a fraction,
\begin{align*}
\lim_{x\to 0} \left(\cot(x) - \csc(x)\right) &= \lim_{x\to 0} \left(\frac{\cos(x)}{\sin(x)} - \frac{1}{\sin(x)}\right)\\
&= \lim_{x\to 0} \frac{\cos(x)-1}{\sin(x)} 
\end{align*}
Setting $f(x) = \cos(x)-1$ and $g(x)=\sin(x)$, both functions are differentiable near zero and 
\[
\lim_{x\to 0}(\cos(x)-1)=\lim_{x\to 0}\sin(x) = 0.
\]
We may now apply l'H\^opital's rule. Write
\[
f'(x) = -\sin(x)\qquad \text{and}\qquad g'(x) = \cos(x),
\]
so
\[
\lim_{x\to 0} \left(\cot(x) - \csc(x)\right) = \lim_{x\to 0} \frac{\cos(x)-1}{\sin(x)} = \lim_{x\to 0} \frac{-\sin(x)}{\cos(x)} =0.
\]
\end{solution}


Sometimes one must be slightly more clever. 

\begin{example}[$\pmb\infty$--$\pmb\infty$]
Compute
\[
\lim_{x\to\infty}\left(\sqrt{x^2+x}-x\right).
\]
\end{example}

\begin{solution}
Again, this doesn't appear to be suitable for l'H\^opital's Rule. A bit of algebraic manipulation will help. Write
\begin{align*}
\lim_{x\to\infty}\left(\sqrt{x^2+x}-x\right) &= \lim_{x\to\infty}\left(x\left(\sqrt{1+1/x}-1\right)\right)\\
&=\lim_{x\to\infty}\frac{\sqrt{1+1/x}-1}{x^{-1}}
\end{align*}
Now set $f(x) = \sqrt{1+1/x}-1$, $g(x) = x^{-1}$. Since both
  functions are differentiable for large values of $x$ and 
\[
\lim_{x\to\infty} (\sqrt{1+1/x}-1) = \lim_{x\to\infty}x^{-1} = 0, 
\]
we may apply l'H\^opital's rule. Write
\[
f'(x) = (1/2)(1+1/x)^{-1/2}\cdot(-x^{-2}) \qquad \text{and}\qquad g'(x) = -x^{-2}
\]
so
\begin{align*}
\lim_{x\to\infty}\left(\sqrt{x^2+x}-x\right) &= \lim_{x\to\infty}\frac{\sqrt{1+1/x}-1}{x^{-1}} \\
&= \lim_{x\to\infty}\frac{(1/2)(1+1/x)^{-1/2}\cdot(-x^{-2})}{-x^{-2}} \\
&= \lim_{x\to\infty} \frac{1}{2\sqrt{1+1/x}}\\
&= \frac{1}{2}.
\end{align*}
\end{solution}


\subsection*{Exponential Indeterminate Forms}

There is a standard trick for dealing with the indeterminate forms
\[
1^\infty,\qquad 0^0,\qquad \infty^0.
\]
Given $u(x)$ and $v(x)$ such that
\[
\lim_{x\to a}u(x)^{v(x)}
\]
falls into one of the categories described above, rewrite as
\[
\lim_{x\to a}e^{v(x)\ln(u(x))}
\]
and then examine the limit of the exponent
\[
\lim_{x\to a} v(x)\ln(u(x)) = \lim_{x\to a} \frac{\ln(u(x))}{v(x)^{-1}}
\]
using l'H\^opital's rule.  Since these forms are all very similar, we
will only give a single example.


\begin{example}[1$^{\pmb\infty}$]
Compute
\[
\lim_{x\to \infty}\left(1 + \frac{1}{x}\right)^x.
\]
\end{example}

\begin{solution}
Write
\[
\lim_{x\to \infty}\left(1 + \frac{1}{x}\right)^x = \lim_{x\to \infty}e^{x\ln\left(1 + \frac{1}{x}\right)}.
\]
So now look at the limit of the exponent
\[
\lim_{x\to\infty} x\ln\left(1 + \frac{1}{x}\right) = \lim_{x\to\infty} \frac{\ln\left(1 + \frac{1}{x}\right)}{x^{-1}}.
\]
Setting $f(x) = \ln\left(1 + \frac{1}{x}\right)$ and $g(x) = x^{-1}$,
both functions are differentiable for large values of $x$ and
\[
\lim_{x\to \infty}\ln\left(1 + \frac{1}{x}\right)=\lim_{x\to \infty}x^{-1} = 0.
\]
We may now apply l'H\^opital's rule. Write
\[
f'(x) = \frac{-x^{-2}}{1 + \frac{1}{x}}\qquad\text{and}\qquad g'(x) = -x^{-2},
\]
so
\begin{align*}
\lim_{x\to\infty} \frac{\ln\left(1 + \frac{1}{x}\right)}{x^{-1}} &= \lim_{x\to\infty} \frac{\frac{-x^{-2}}{1 + \frac{1}{x}}}{-x^{-2}} \\
&=\lim_{x\to\infty} \frac{1}{1 + \frac{1}{x}}\\
&=1.
\end{align*}
Hence, 
\[
\lim_{x\to \infty}\left(1 + \frac{1}{x}\right)^x = \lim_{x\to \infty}e^{x\ln\left(1 + \frac{1}{x}\right)} =e^{1} = e.
\]
\end{solution}











% Most from Keisler
\begin{exercises}

\noindent Compute the limits.

\twocol

\begin{exercise} $\lim_{x\to 0} {\cos x -1\over \sin x}$
\begin{answer} $0$
\end{answer}\end{exercise}

\begin{exercise} $\lim_{x\to \infty} {e^x\over x^3}$
\begin{answer} $\infty$
\end{answer}\end{exercise}

\begin{exercise} $\lim_{x\to \infty} \sqrt{x^2+x}-\sqrt{x^2-x}$
\begin{answer} $1$
\end{answer}\end{exercise}

\begin{exercise} $\lim_{x\to \infty} {\ln x\over x}$
\begin{answer} $0$
\end{answer}\end{exercise}

\begin{exercise} $\lim_{x\to \infty} {\ln x\over \sqrt{x}}$
\begin{answer} $0$
\end{answer}\end{exercise}

\begin{exercise} $\lim_{x\to\infty} {e^x + e^{-x}\over e^x -e^{-x}}$
\begin{answer} 1
\end{answer}\end{exercise}

\begin{exercise} $\lim_{x\to0}{\sqrt{9+x}-3\over x}$
\begin{answer} $1/6$
\end{answer}\end{exercise}

\begin{exercise} $\lim_{t\to1+}{(1/t)-1\over t^2-2t+1}$
\begin{answer} $-\infty$
\end{answer}\end{exercise}

\begin{exercise} $\lim_{x\to2}{2-\sqrt{x+2}\over 4-x^2}$
\begin{answer} $1/16$
\end{answer}\end{exercise}

\begin{exercise} $\lim_{t\to\infty}{t+5-2/t-1/t^3\over 3t+12-1/t^2}$
\begin{answer} $1/3$
\end{answer}\end{exercise}

\begin{exercise} $\lim_{y\to\infty}{\sqrt{y+1}+\sqrt{y-1}\over y}$
\begin{answer} $0$
\end{answer}\end{exercise}

\begin{exercise} $\lim_{x\to1}\frac{\sqrt{x}-1}{\sqrt[3]{x}-1}$
\begin{answer} $3/2$
\end{answer}\end{exercise}

\begin{exercise} $\lim_{x\to0}{(1-x)^{1/4}-1\over x}$
\begin{answer} $-1/4$
\end{answer}\end{exercise}

\begin{exercise} $\lim_{t\to 0}{\left(t+{1\over t}\right)((4-t)^{3/2}-8)}$
\begin{answer} $-3$
\end{answer}\end{exercise}

\begin{exercise} $\lim_{t\to 0+}\left({1\over t}+{1\over\sqrt{t}}\right)
(\sqrt{t+1}-1)$
\begin{answer} $1/2$
\end{answer}\end{exercise}

\begin{exercise} $\lim_{x\to 0}{x^2\over\sqrt{2x+1}-1}$
\begin{answer} $0$
\end{answer}\end{exercise}

\begin{exercise} $\lim_{u\to 1}{(u-1)^3\over (1/u)-u^2+3/u-3}$
\begin{answer} $0$
\end{answer}\end{exercise}

\begin{exercise} $\lim_{x\to 0}{2+(1/x)\over 3-(2/x)}$
\begin{answer} $-1/2$
\end{answer}\end{exercise}

\begin{exercise} $\lim_{x\to 0+}{1+5/\sqrt{x}\over 2+1/\sqrt{x}}$
\begin{answer} $5$
\end{answer}\end{exercise}

\begin{exercise} $\lim_{x\to 0+}{3+x^{-1/2}+x^{-1}\over 2+4x^{-1/2}}$
\begin{answer} $\infty$
\end{answer}\end{exercise}

\begin{exercise} $\lim_{x\to\infty}{x+x^{1/2}+x^{1/3}\over x^{2/3}+x^{1/4}}$
\begin{answer} $\infty$
\end{answer}\end{exercise}

\begin{exercise} $\lim_{t\to\infty}
{1-\sqrt{t\over t+1}\over 2-\sqrt{4t+1\over t+2}}$
\begin{answer} $2/7$
\end{answer}\end{exercise}

\begin{exercise} $\lim_{t\to\infty}{1-{t\over t-1}\over 1-\sqrt{t\over t-1}}$
\begin{answer} $2$
\end{answer}\end{exercise}

\begin{exercise} $\lim_{x\to-\infty}{x+x^{-1}\over 1+\sqrt{1-x}}$
\begin{answer} $-\infty$
\end{answer}\end{exercise}



\begin{exercise} $\lim_{x\to\pi/2}{\cos x\over (\pi/2)-x}$
\begin{answer} $1$
\end{answer}\end{exercise}

\begin{exercise} $\lim_{x\to0}{e^x-1\over x}$
\begin{answer} $1$
\end{answer}\end{exercise}

\begin{exercise} $\lim_{x\to0}{x^2\over e^x-x-1}$
\begin{answer} $2$
\end{answer}\end{exercise}

\begin{exercise} $\lim_{x\to1}{\ln x\over x-1}$
\begin{answer} $1$
\end{answer}\end{exercise}

\begin{exercise} $\lim_{x\to0}{\ln(x^2+1)\over x}$
\begin{answer} $0$
\end{answer}\end{exercise}

\begin{exercise} $\lim_{x\to1}{x\ln x\over x^2-1}$
\begin{answer} $1/2$
\end{answer}\end{exercise}

\begin{exercise} $\lim_{x\to0}{\sin(2x)\over\ln(x+1)}$
\begin{answer} $2$
\end{answer}\end{exercise}

\begin{exercise} $\lim_{x\to1}{x^{1/4}-1\over x}$
\begin{answer} $0$
\end{answer}\end{exercise}

\begin{exercise} $\lim_{x\to1+}{\sqrt{x}\over x-1}$
\begin{answer} $\infty$
\end{answer}\end{exercise}

\begin{exercise} $\lim_{x\to1}{\sqrt{x}-1\over x-1}$
\begin{answer} $1/2$
\end{answer}\end{exercise}

\begin{exercise} $\lim_{x\to\infty}{x^{-1}+x^{-1/2}\over x+x^{-1/2}}$
\begin{answer} $0$
\end{answer}\end{exercise}

\begin{exercise} $\lim_{x\to\infty}{x+x^{-2}\over 2x+x^{-2}}$
\begin{answer} $1/2$
\end{answer}\end{exercise}

\begin{exercise} $\lim_{x\to\infty}{5+x^{-1}\over 1+2x^{-1}}$
\begin{answer} $5$
\end{answer}\end{exercise}

\begin{exercise} $\lim_{x\to\infty}{4x\over\sqrt{2x^2+1}}$
\begin{answer} $2\sqrt2$
\end{answer}\end{exercise}

\begin{exercise} $\lim_{x\to0}{3x^2+x+2\over x-4}$
\begin{answer} $-1/2$
\end{answer}\end{exercise}

\begin{exercise} $\lim_{x\to0}{\sqrt{x+1}-1\over \sqrt{x+4}-2}$
\begin{answer} $2$
\end{answer}\end{exercise}

\begin{exercise} $\lim_{x\to0}{\sqrt{x+1}-1\over \sqrt{x+2}-2}$
\begin{answer} $0$
\end{answer}\end{exercise}

\begin{exercise} $\lim_{x\to0+}{\sqrt{x+1}+1\over\sqrt{x+1}-1}$
\begin{answer} $\infty$
\end{answer}\end{exercise}

\begin{exercise} $\lim_{x\to0}{\sqrt{x^2+1}-1\over\sqrt{x+1}-1}$
\begin{answer} $0$
\end{answer}\end{exercise}

\begin{exercise} $\lim_{x\to\infty}{(x+5)\left({1\over 2x}+{1\over x+2}\right)}$
\begin{answer} $3/2$
\end{answer}\end{exercise}

\begin{exercise} $\lim_{x\to0+}{(x+5)\left({1\over 2x}+{1\over x+2}\right)}$
\begin{answer} $\infty$
\end{answer}\end{exercise}

\begin{exercise} $\lim_{x\to1}{(x+5)\left({1\over 2x}+{1\over x+2}\right)}$
\begin{answer} $5$
\end{answer}\end{exercise}

\begin{exercise} $\lim_{x\to2}{x^3-6x-2\over x^3+4}$
\begin{answer} $-1/2$
\end{answer}\end{exercise}

\begin{exercise} $\lim_{x\to2}{x^3-6x-2\over x^3-4x}$
\begin{answer} does not exist
\end{answer}\end{exercise}

\begin{exercise} $\lim_{x\to1+}{x^3+4x+8\over 2x^3-2}$
\begin{answer} $\infty$
\end{answer}\end{exercise}
\endtwocol
\end{exercises}





