\chapter{The Derivatives of Trigonometric Functions}



\section{The Derivatives of Trigonometric Functions}

Up until this point of the course we have been largely ignoring a large
class of functions---those involving $\sin(x)$ and $\cos(x)$. It is
now time to visit our two friends who concern themselves periodically
with triangles and circles.

\begin{theorem}[The Derivative of sin(\textit{x})]\index{derivative!of sine}\label{theorem:deriv sin}
\[
\ddx \sin(x) = \cos(x).
\]
\end{theorem}
\marginnote{
\begin{align*}
\lim_{h\to 0}\frac{\cos(h)-1}{h} &= \lim_{h\to 0}\left(\frac{\cos(h)-1}{h}\cdot\frac{\cos(h)+1}{\cos(h)+1}\right)\\
&=\lim_{h\to 0}\frac{\cos^2(h)-1}{h(\cos(h)+1)}\\
&=\lim_{h\to 0}\frac{-\sin^2(h)}{h(\cos(h)+1)}\\
&=-\lim_{h\to 0}\left(\frac{\sin(h)}{h}\cdot\frac{\sin(h)}{(\cos(h)+1)}\right)\\
&= -1 \cdot \frac{0}{2} = 0.
\end{align*}
}
\begin{proof}
Using the definition of the derivative, write
\begin{align*}
\ddx \sin(x) &= \lim_{h\to0} \frac{\sin(x+h)-\sin(x)}{h} \\
&= \lim_{h\to0} \frac{\sin(x)\cos(h)+\sin(h)\cos(x)-\sin(x)}{h}  & \text{Trig Identity.}\\
&= \lim_{h\to0} \left(\frac{\sin(x)\cos(h)-\sin(x)}{h} + \frac{\sin(h)\cos{x}}{h} \right)\\
&=\lim_{h\to0} \left(\sin (x)\frac{\cos(h) - 1}{h}+\cos(x)\frac{\sin(h)}{h}\right) \\
&=\sin(x) \cdot 0 + \cos(x) \cdot 1 = \cos x. & \text{See Example~\ref{example:sinx/x}.}
\end{align*}
\end{proof}

Consider the following geometric interpretation of the derivative of
$\sin(\theta)$.  
%\begin{figure}

\begin{tikzpicture}
	\begin{axis}[
            xmin=-.1,xmax=1.1,ymin=-.1,ymax=1.1,
            axis lines=center,
            ticks=none,
            width=5in,
            unit vector ratio*=1 1 1,
            xlabel=$x$, ylabel=$y$,
            every axis y label/.style={at=(current axis.above origin),anchor=south},
            every axis x label/.style={at=(current axis.right of origin),anchor=west},
          ]        
          \addplot [very thick, textColor!30!background, smooth, domain=(-.2:.2+pi/2)] ({cos(deg(x))},{sin(deg(x))});
          \addplot [textColor,very thick] plot coordinates {(0,0) (.766,.643)}; %% 40 degrees
          \addplot [textColor,very thick] plot coordinates {(0,0) (.766,0)}; %% bottom
          \addplot [very thick, penColor2!30!background] {(x-.766)*(-.766/.643)+.643};
          \addplot [textColor,dashed] plot coordinates {(0,0) (.766-.196,.643+1-.766)}; %% 40+16.98 degrees          

          %% \addplot [textColor!20!background] plot coordinates {(.766,.643) (1,.839)}; %% hyp
          %% \addplot [textColor!20!background] plot coordinates {(1,.643) (1,.839)}; %% side
          %% \addplot [textColor!20!background] plot coordinates {(.766,.643) (1,.643)}; %% bottom
          %% \addplot [textColor!20!background,smooth, domain=(0:40)] ({.05*cos(x)+.766},{.05*sin(x)+.643}); %% angle
          %% \node at (axis cs:.84,.670) [textColor!20!background] {\footnotesize$\theta$};
          
          %% \addplot [textColor!20!background] plot coordinates {(.766,.643) (.766,.839)}; %% side
          %% \addplot [textColor!20!background] plot coordinates {(.766,.839) (1,.839)}; %% bottom
          %% \addplot [textColor!20!background,smooth, domain=(180:220)] ({.05*cos(x)+1},{.05*sin(x)+.839}); %% angle
          %% \node at (axis cs:.926,.812) [textColor!20!background] {\footnotesize$\theta$};
          
          \draw[rotate around={30:(.5,.5)}] (.7,.7) rectangle (.25,.25);

          %\draw[textColor, rotate around={45:(.5,.5)}] (.5,.5) rectangle (.2,.2);

          \addplot [penColor4,very thick] plot coordinates {(.766,.643) (.766,.643+1-.766)}; %% side
          \addplot [textColor,very thick] plot coordinates {(.766,.643+1-.766) (.766-.196,.643+1-.766)}; %% top
          \addplot [textColor,smooth, domain=(90:130)] ({.05*cos(x)+.766},{.05*sin(x)+.643}); %% angle
          \addplot [very thick, textColor] plot coordinates {(.766-.196,.643+1-.766) (.766,.643)}; %% hyp
          \node at (axis cs:.739,.717) [textColor] {\footnotesize$\theta$};
          
          \node at (axis cs:.668,.877) [anchor=south] {\footnotesize$h\sin(\theta)$};
          \node at (axis cs:.766,.76) [anchor=west] {\footnotesize$h\cos(\theta)$};
          \node at (axis cs:.65,.78) [anchor=west] {\footnotesize$\approx h$};

          \addplot [very thick, penColor] plot coordinates {(.766,0) (.766,.643)}; %% sin theta          
          
          \addplot [textColor, smooth, domain=(0:40)] ({.15*cos(x)},{.15*sin(x)});
          \addplot [textColor, smooth, domain=(40:56.90)] ({.17*cos(x)},{.17*sin(x)});
          \addplot [textColor, smooth, domain=(40:56.90)] ({.185*cos(x)},{.185*sin(x)});
          \node at (axis cs:.15,.07) [anchor=west] {$\theta$};
          \node at (axis cs:.15,.17) {$h$};
          \node at (axis cs:.766,.322) [anchor=east] {$\sin(\theta)$};
          \node at (axis cs:.383,0) [anchor=north] {$\cos(\theta)$};
        \end{axis}
\end{tikzpicture}
%\label{figure:geo-interp sinx/x}
%\end{figure}

Here we see that increasing $\theta$ by a ``small amount'' $h$,
increases $\sin(\theta)$ by approximately $h\cos(\theta)$. Hence,
\[
\frac{\Delta y}{\Delta \theta}\approx \frac{h\cos(\theta)}{h} =
\cos(\theta).
\]

With this said, the derivative of a function measures the slope of the
plot of a function.  If we examine the graphs of the sine and cosine
side by side, it should be that the latter appears to accurately
describe the slope of the former, and indeed this is true, see
Figure~\ref{figure:sin/cos}.
\begin{figure*}
\begin{tikzpicture}
	\begin{axis}[
            xmin=-6.75,xmax=6.75,ymin=-1.5,ymax=1.5,
            axis lines=center,
            xtick={-6.28, -4.71, -3.14, -1.57, 0, 1.57, 3.142, 4.71, 6.28},
            xticklabels={$-2\pi$,$-3\pi/2$,$-\pi$, $-\pi/2$, $0$, $\pi/2$, $\pi$, $3\pi/2$, $2\pi$},
            ytick={-1,1},
            %ticks=none,
            width=9in,
            height=2in,
            unit vector ratio*=1 1 1,
            xlabel=$x$, ylabel=$y$,
            every axis y label/.style={at=(current axis.above origin),anchor=south},
            every axis x label/.style={at=(current axis.right of origin),anchor=west},
          ]        
          \addplot [very thick, penColor, samples=100,smooth, domain=(-6.75:6.75)] {sin(deg(x))};
          \addplot [very thick, penColor2, samples=100,smooth, domain=(-6.75:6.75)] {cos(deg(x))};
          \node at (axis cs:3.14,.75) [penColor] {$f(x)$};
          \node at (axis cs:-1.57,.75) [penColor2] {$f'(x)$};
        \end{axis}
\end{tikzpicture}
\caption{Here we see a plot of $f(x)=\sin(x)$ and its derivative
  $f'(x)=\cos(x)$. One can readily see that $\cos(x)$ is positive when
  $\sin(x)$ is increasing, and that $\cos(x)$ is negative when
  $\sin(x)$ is decreasing.}
\label{figure:sin/cos}
\end{figure*}

Of course, now that we know the derivative of the sine, we can compute
derivatives of more complicated functions involving the sine.

%\break

\begin{theorem}[The Derivative of cos(\textit{x})]\index{derivative!of cosine}
\[
\ddx \cos(x) = -\sin(x).
\]
\end{theorem}

\begin{proof}
Recall that
\begin{align*}
\cos(x) &= \sin\left(x+\frac{\pi}{2}\right), \\
\sin(x) &= -\cos\left(x+\frac{\pi}{2}\right).
\end{align*}
Now:
\begin{align*}
\ddx \cos(x) &= \ddx \sin\left(x+\frac{\pi}{2}\right)\\
&=\cos\left(x+\frac{\pi}{2}\right)\cdot 1 \\
&= -\sin(x).
\end{align*}
\end{proof}

Next we have:

\begin{theorem}[The Derivative of tan(\textit{x})]\index{derivative!of tangent}
\[
\ddx \tan(x) = \sec^2(x).
\]
\end{theorem}

\begin{proof}
We'll rewrite $\tan(x)$ as $\frac{\sin(x)}{\cos(x)}$ and use the quotient rule. Write
\begin{align*}
\ddx\tan(x) &= \ddx\frac{\sin(x)}{\cos(x)}\\
&=\frac{\cos^2(x) + \sin^2(x)}{\cos^2(x)}\\
&=\frac{1}{\cos^2(x)}\\
&=\sec^2(x).
\end{align*}
\end{proof}

Finally, we have

\begin{theorem}[The Derivative of sec(\textit{x})]\index{derivative!of secant}
\[
\ddx \sec(x) = \sec(x)\tan(x).
\]
\end{theorem}

\begin{proof}
We'll rewrite $\sec(x)$ as $(\cos(x))^{-1}$ and use the power rule and the chain rule. Write
\begin{align*}
\ddx \sec(x) &= \ddx(\cos (x))^{-1}\\
&=-1(\cos(x))^{-2}(-\sin(x)) \\
&= \frac{\sin(x)}{\cos^2(x)} \\
&= \sec(x)\tan(x).
\end{align*}
\end{proof}


The derivatives of the cotangent and cosecant are similar and left as
exercises. 

Putting this all together, we have:

\begin{mainTheorem}[The Derivatives of Trigonometric Functions] \hfil
\begin{itemize}
\item $\ddx \sin(x) = \cos(x)$.
\item $\ddx \cos(x) = -\sin(x)$.
\item $\ddx \tan(x) = \sec^2(x)$.
\item $\ddx \sec(x) = \sec(x)\tan(x)$.
\item $\ddx \csc(x) = -\csc(x)\cot(x)$.
\item $\ddx \cot(x) = -\csc^2(x)$.
\end{itemize}
\end{mainTheorem}


\begin{warning}
When working with derivatives of trigonometric functions, we suggest
you use \textbf{radians} for angle measure. For example, while
\[
\sin\left((90^\circ\right)^2) = \sin\left(\left(\frac{\pi}{2}\right)^2\right),
\]
one must be careful with derivatives as
\[
\left. \ddx \sin\left(x^2\right)\right|_{x=90^\circ} \ne \underbrace{2\cdot 90\cdot \cos(90^2)}_{\text{incorrect}}
\]
Alternatively, one could think of $x^\circ$ as meaning
$\frac{x\cdot\pi}{180}$, as then $90^\circ = \frac{90\cdot\pi}{180} =
\frac{\pi}{2}$. In this case
\[
2\cdot 90^\circ\cdot \cos((90^\circ)^2) = 2\cdot \frac{\pi}{2}\cdot\cos\left(\left(\frac{\pi}{2}\right)^2\right).
\]
\end{warning}



\begin{exercises}
Find the derivatives of the following functions.

\twocol

\begin{exercise} $\sin^2(\sqrt{x})$
\begin{answer} $\sin(\sqrt{x})\cos(\sqrt{x})/\sqrt{x}$
\end{answer}\end{exercise}

\begin{exercise} $\sqrt{x}\sin(x)$
\begin{answer} ${\sin(x)\over2\sqrt x}+\sqrt{x}\cos(x)$
\end{answer}\end{exercise}

\begin{exercise} ${1\over \sin(x)}$
\begin{answer} $ -{\cos(x)\over \sin^2(x)}$
\end{answer}\end{exercise}

\begin{exercise} ${x^2+x\over \sin(x)}$
\begin{answer} ${(2x +1)\sin(x) - (x^2+x)\cos(x) \over \sin^2 (x)}$
\end{answer}\end{exercise}

\begin{exercise} $\sqrt{1-\sin^2(x)}$
\begin{answer} ${-\sin(x)\cos(x)\over \sqrt{1-\sin^2(x)}}$
\end{answer}\end{exercise}

\begin{exercise} $\sin (x)\cos(x)$
\begin{answer} $\cos^2(x)-\sin^2(x)$
\end{answer}\end{exercise}

\begin{exercise} $\sin(\cos(x))$
\begin{answer} $-\sin(x)\cos(\cos(x))$
\end{answer}\end{exercise}

\begin{exercise} $\sqrt{x\tan(x)}$
\begin{answer} ${\tan(x)+x\sec^2(x)\over2\sqrt{x\tan(x)}}$
\end{answer}\end{exercise}

\begin{exercise} $\tan(x)/(1+\sin(x))$
\begin{answer} ${\sec^2(x)(1+\sin(x))-\tan(x) \cos(x)\over (1+\sin(x))^2}$
\end{answer}\end{exercise}

\begin{exercise} $\cot(x)$
\begin{answer} $ -\csc^2(x)$
\end{answer}\end{exercise}

\begin{exercise} $\csc(x)$
\begin{answer} $ -\csc(x)\cot(x)$
\end{answer}\end{exercise}

\begin{exercise} $x^3 \sin (23x^2 )$
\begin{answer} $3x^2\sin(23x^2)+46x^4\cos(23x^2)$
\end{answer}\end{exercise}

\begin{exercise} $\sin ^2(x) + \cos ^2(x)$
 \begin{answer} $0$
\end{answer}\end{exercise}

\begin{exercise}  $\sin (\cos (6x) )$
 \begin{answer} $-6\cos(\cos(6x))\sin(6x)$
\end{answer}\end{exercise}

\endtwocol

\begin{exercise} Compute ${d\over d\theta}{\sec(\theta)\over 1+\sec(\theta)}$.
 \begin{answer} $\sin(\theta)/(\cos(\theta)+1)^2$
\end{answer}\end{exercise}

\begin{exercise} Compute ${d\over dt}t^5 \cos (6t)$.
\begin{answer} $5t^4\cos(6t)-6t^5\sin(6t)$
\end{answer}\end{exercise}

\begin{exercise} Compute ${d\over dt}{t^3 \sin (3t)\over\cos (2t)}$.
\begin{answer} $3t^2(\sin(3t)+t\cos(3t))/\cos(2t)+2t^3\sin(3t)\sin(2t)/\cos^2(2t)$
\end{answer}\end{exercise}

\begin{exercise} Find all points on the graph of
$f(x)=\sin^2(x)$ at which the tangent line is horizontal.
\begin{answer} $n\pi/2$, any integer $n$
\end{answer}\end{exercise}

\begin{exercise} Find all points on the graph of $f(x) = 2\sin(x) -
\sin^2(x)$ at which the tangent line is horizontal.
\begin{answer} $\pi/2+n\pi$, any integer $n$
\end{answer}\end{exercise}

\begin{exercise} Find an
 equation for the tangent line to $\sin^2(x)$ at 
$x=\pi/3$.
\begin{answer} $\sqrt3x/2+3/4-\sqrt3\pi/6$
\end{answer}\end{exercise}

\begin{exercise} Find an equation for the tangent line to $\sec^2(x)$
at $x=\pi/3$.
\begin{answer} $8\sqrt3x+4-8\sqrt3\pi/3$
\end{answer}\end{exercise}

\begin{exercise} Find an equation for the tangent line to $\cos^2(x) -
\sin^2(4x)$ at $x=\pi/6$.
\begin{answer} $3\sqrt3x/2-\sqrt3\pi/4$
\end{answer}\end{exercise}

\begin{exercise} Find the points on the curve $y= x+ 2\cos(x)$ that have a
horizontal tangent line.
\begin{answer} $\pi/6+2n\pi$, $5\pi/6+2n\pi$, any integer $n$
\end{answer}\end{exercise}

\end{exercises}

